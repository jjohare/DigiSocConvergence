This chapter identifies an intersectional space across the described technologies, and proposes a valuable and novel software stack, which can enable exploration  and product development. It is useful to briefly look at some of the potential applications which might benefit from value and trust exchange within an global shared social space.

\section{Summary TL;DR}
\textbf{The points below are the tl;dr for the whole book. They do not currently correspond to the order in the book, but will do at some point
\begin{itemize}
\item There may be an inflection point in the organisational topology of the internet because of trust abuses by the incumbent providers. This is calling itself Web3, but the moniker is fraught with problems and somewhat meaningless. The drivers are real.
\item `The Metaverse' is coming, in some form, at some point. Everyone is positioning in case it's ``soon''.
\item It's not at all clear what it is, or if people want it, but the emergent narrative seems to be ``digital society'' and that obviously should not be dismissed lightly.
\item From a business perspective metaverse is the worst of the remote collaboration tool-kits, and undermines flow, productivity, and interpersonal trust.
\item It's probably technology for technologies sake at this time, but it's a strong attempt, and the investment is real.
\item Excluding Facebook/Meta a lot of the investment is coming from the Web3 speculative bubble, who have a parallel and intersectional metaverse narrative. 
\item Industry has noted the risk, and failures of Meta in this domain, and have latched onto "open metaverse" as a narrative, to de-risk their interest.
\item The current open metaverse is muddy and confused.
\item Anything from a multi-million pound XR studio screen, to a speech audio system, can be a metaverse interface.
\item Constraints (metaverse schema) which are crafted in response to the use case, the interface, and the audience, is where challenges are most profound. This is an opportunity.
\item Trust, accessibility, governance, and safeguarding, are far harder in these real-time distributed social systems.
\item There's a bunch of start-ups and options, interesting research, and some adoption, but scaling is very hard in the face of tepid consumer interest.
\item Industry seem to be missing the point; that open metaverse should mean open source metaverse. There are some options, but they are under developed. This is an opportunity.
\item There is genuine, undeniable interest in digital scarcity. The ownership of digital goods seems natural to younger, digitally native users.
\item This is serviced already by various (gaming) platforms, but they are all isolated ecosystems.
\item Uniting these attempts, with portable (transferable) ``goods'' across the digital universe (metaverse) likely requires a global digital ledger (blockchain).
\item Crypto is igniting imagination on this topic, and is seeing adoption both inside out outside of the metaverse contexts.
\item Crypto is a nightmare; rife with scams, poor technology choices, limited life, and incorrect assumptions.
\item The only thing blockchain/crypto can do well is ``money like networks'', which is a cornerstone of human interaction, and the killer application.
\item Representations of dollars and pounds etc money can ride securely on top of such networks as stablecoins, and this is getting easier to integrate, though there are risks.
\item It's unclear which technology will win, if any, but since the tools exist now they can be integrated now.
\item Legislative and cultural headwinds are significant. There might be no opportunity here in the end, though ``rough game theory'' supports the attempt.
\item Money, identity, and thereby trust, can already be mediated by the Bitcoin network, even without using Bitcoin the asset. This is an opportunity.
\item Digital ownership is nearly possible at scale, low cost, and minimal environmental impact, with the Bitcoin network. This is an opportunity.
\item A topologically flat, inclusive, permissionless, open metaverse, with economically empowered ML and AI actors, which can mediate governance issues, transparently, according to well constructed custom schemas, between cryptographically verifiable economic users (human or AI) is an opportunity.
\item New open source machine learning tooling potentially removes many of the problems with accessibility, creativity, language barriers, safeguarding, and governance. This is a huge opportunity.
\end{itemize}
}


\section{Software stack}
This section needs building out to describe the stack and the choices made, but can be seen in Figure \ref{fig:pyramind} and Figure \ref{fig:highlevelstack}.

\begin{figure*}[ht]\centering 	\includegraphics[width=\linewidth]{pyramid}
	\caption{Pyramid showing the components for sats, stablecoins on lightning, asssets, and trust}
	\label{fig:pyramind}
\end{figure*}

\begin{figure*}[ht]\centering 	\includegraphics[width=\linewidth]{highlevelstack}
	\caption{High level overview showing the components for sats, stablecoins on lightning, asssets, and trust}
	\label{fig:highlevelstack}
\end{figure*}

\section{Pathway \& telepresence}
Designing open federated metaverse from a 25 year research foundation
There are serious and under discussed natural social constraints on group behaviours, and these translate into social VR. For instance the ideal meeting size is 6, and this is naturally established in work settings. This has not translated into a metaverse setting where dozens of people routinely crash across one another. In the context of supporting a creative “backstage” world where set planning, production shots, etc can be discussed we believe we have solutions which will get the best out of distributed teams of film-makers.
Leveraging the world's most powerful decentralised computing network to create 
scale and security without high cost
The Bitcoin network is more than just a speculative money like asset, it is the most secure distributed computing system ever built. We can jump on the back of this at almost no cost to enable scale for transfer of value, trust, and digital assets of provenance.
Cryptographically assured end points
With the cryptography tools provided through integration of the Bitcoin network we can also use non-blockchain based secure messaging, and identity proofs. 
Micro transactions in collaborative spaces
New tooling the space allows fractions of a pound or dollar to be exchanged between parties across the world. This means that work can be paid “by the second” both inside and outside of the metaverse. This radically improves creative microtask workflows.
World leading open source machine learning and bot architectures
By integrating Stablity AI tools for image generation, video processing, natural language, and speech to text / text to speech we hope to reduce friction within the backstage worlds.
Creating a narrative arrow from a remote director/producer/DP, through a VP screen into a shoot, and back into a persistent metaverse shared with the public
By linking across these new systems with world class telepresence research we hope to use a single digital context to support senior stakeholders, creatives, technical teams, and the wider public.
New paths to monetisation and digital ownership
This unified digital back end is optimised for flows of money, trust, and digital objects. This is a new area for VP.
Current workstreams:
\begin{itemize}
\item Storyboarding with text2img and dreambooth to add talent and costume ideas before meeting up, as demonstrated in this document \cite{ruiz2022dreambooth}.
\item Collaborative, self hosted, high speed, low detail, economically and cryptographically enabled set design spaces, with near instant language translation (speech to text an speech to speech). Micropayment for cheap international labour. Technology agnostic. Use the screen, audio only, compressed video dial-in, headsets, tablet rendering: (this book).
\item High end telepresence \cite{Roberts2015, OHare2018, Fairchild2017, OHare2016} into the studio/shoot from the virtual set, allowing high value stakeholders to be `present` on set as virtual collaborants with spatial descrimination allowing directional queues. This involved real time human capture like moveAI or the expensive rigs with DSLRs.
\item Novel render pipeline for fast turnaround of final look and feel, taking the rough scene and applying img2img ML with the kind of interframe consistency we are starting to see from the video projects \cite{anonymous2023phenaki}.
\item Text to model pipeline for interactively building key elements with senior stakeholders, pushed from post ideation the the  pre-shoot Unreal content creation \cite{poole2022dreamfusion}.
\item All assets switch over to Unreal metaverse and become consistent (optimised) digital set which can be visited by stakeholders, funders, VIPs etc. Public can visit later for a fee? Digital assets can be bought from the set.
\end{itemize}

\section{Money in metaverses}
\subsection{ML actors and blockchain based bots}
Stablity AI is an open source imitative to bring ML/AL capabilities to the world. This is a hugely exciting emergent area and much more will be developed here.
\subsection{AI economic actors in mixed reality}
AI actors can now be trusted visually \cite{nightingale2022ai}. We have some thinking on this which links the external web to our proposed metaverse \href{https://melbotz.github.io/melbot_20220713/}{using tooling} developed by Melvin Carvelho. He is working on economically empowered bots which leverage Nostr and RGB to perform functions within our metaverse, and outside in the WWW, as well as interacting economically through trusted cryptography with other bots, anywhere, and human participants, anywhere. This is incredibly powerful and is assured by the Bitcoin security model. Imagine being able to interact with a bot flower seller representing all the real world florists it had found. In the metaverse you could handle the flowers and take advice and guidance from the bot agent, then it would be able to take your money to buy you flowers to send to a real world address, and later find you to tell you when it's delivered. These possibilities are endless. The AI chat element, the AI translation of images on websites to 3D assets in the Metaverse are difficult but possible challenges, but the secure movement of money from the local context in the metaverse to the real world is within reach using these bots, and they are completely autonomous and distributed.
\section{Our socialisation best practice}
\subsubsection{Identity}
We will base our identity and object management on Nostr public/private key pairs. The public key of these will generate LNBits economic wallets which will interact with RGB and Lightning globally.
\subsubsection{Webs of trust}
Webs of trust will be built within worlds using economically costly (but private) social rating systems, between any actor, human or AI. It should be too costly to attack an individual aggressively. This implies an increased weighting for scores issued in short time periods. Poorly behaving AI's will eventually be excluded through lack of funds.
\subsubsection{Integration of 'good' actor AI entities}
Gratitude practice should be encouraged between AI actors to foster trust and wellbeing in human observers. ``It's nice to be nice'' should be incentivised between all parties''. This could include tipping and trust nudging through the social rating system. Great AI behaviour would result in economically powerful entities.
\subsection{Emulation of important social cues}
\href{https://www.cleverclassroomsdesign.co.uk/general-5}{Classroom layout}
\subsubsection{Behaviour incentives, arbitration, and penalties}
Collapses of trust and abuse will trigger flags from ML based oversight, which will create situational records and payloads of involved parties to unlock with their nostr private keys. ML red flagged actors will be finacially penalised but have access to human arbitration using their copy of the data blob. Nothing will be stored except by the end users.
\subsection{Federations of webs of trust and economics}
Web2/3 enabled \href{https://bitbots.org}{`Bitbots'} by Melvin Carvalho, with their own money supply, should be able to carry data payloads between virtual worlds across all networks. This is an RGB/Nostr/Melbot problem space and requires more investigation. 
\section{Security evaluation}
As part of developing our stack we will penetration test the deployment as detailed using \href{https://hexway.io/}{Hexway}
\section{Potential applications }
\begin{itemize}
\item Art / NFT galleries with instant sales
\end{itemize}

This application allows artists and content creator communities to
display and sell NFT and fungible art to global consumer audiences,
instantly.

\begin{itemize}
\item
  Large scale conference center

  \begin{itemize}
  \item
    Academic conferences
  \item
    Political conference
  \item
    Commercial expo
  \end{itemize}
\end{itemize}

In a hypothetical virtual conference centre a true marketplace of ideas
could be enacted, with participants being paid directly by their
audience based on the proximity to the presentation.

\begin{itemize}
\item
  Group entertainment

  \begin{itemize}
  \item Global social puzzle gaming with prizes
  \item
    Music festivals and gigs - Pay live artists and DJs in real time
    depending on location within the extended landscape of the venue.
    Split to music producer a portion of the value
  \item
    Mixed reality theatre
  \item
    murder mystery
  \item
    Mixed reality live immersive MMORG games
  \item
    Bingo and mass participation gameshows
  \item
    Immersive brand storytelling metaverses
  \item
    Escape rooms
  \end{itemize}
\item
  Debating townhall meetings (with voting etc)
\item
  Mixed reality information metaverse

  \begin{itemize}
  \item
    AR based city tours with collectibles
  \item
    AR based collectibles for trails and heritage (museums, libraries)
    with location specific donations.
  \end{itemize}
\item
  Retail applications

  \begin{itemize}
  \item
    Proxy for physical market
  \item
    AR home delivery market interface within physical marketplaces
  \end{itemize}
\item
  Global course / Education provision
    \begin{itemize}
  \item
    Explore the universe as a group of spaceship or planet characters
  \item
    Explore biology and physics at a microscopic and nanoscopic level
  \end{itemize}
\item
  Micro tasking marketplace
\item
  Code bounty marketplace
\item
  Micro remittance role sharing (business PA / reception etc)
\item
  Careers fair with credential passing
\item
  Auctions in mixed reality
\item
  eSports and live sports
\item
  Gambling, betting markets, and financial leverage markets
\end{itemize}

\subsection{Global cybersec course delivery}
Isolating and building out one example here:
\begin{itemize}
\item Elements for the infrastructure: Economic layer, asset layer, content interface, user management, data storage, microsites loaded in Wolvin and webm, accessibility schema, network security, backups, secure messaging. Deployable framework with high modularity. Some more ossified elements for surity, some less so for malleability and open opportunity. Figure \ref{fig:globalclassroom}.
\item Course delivery in XR, how to we develop a platform, marketplace, framework for open contribution.
\item WebXR, Vircadia, any snap in metaverse middleware that is free and open source (action to compare the two). 
\item Define an interface schema for bolting in any commercial or FOSS metaverse engine.
\item VR marketplace (outside the scope of the VR engine) without a trusted third party.
\item Cryptographically managed learning deliverables (coursework as NFT). 
\item Secure messaging and group messaging using cryptographic keys. Check this stuff with the distributed computing science people in the group (action on John)
\item work toward an exemplar MVP which is then "in the wild"
\item Platform for educators
\item Define scheme, documentation, best practice, interfaces, functional objects, pedagogy, accessibility, multi-language. 
\item Define user management system for educators and client learners.
\item Identify the pain points which current FOSS elements which need development time/money
\item separate the UI/engine from the graphical assets, and the educational / pedagogical components, accessibility, and the value and asset transfer layers.
\item Desktop systems are the primary target (low end system)
\item define schema for accessibility. Colour, subtitles, immersion concerns which can be applied to metaverse rooms through API?
\item Start to define the hybrid presentation model we favour. Avatars? Micro sites? A combination of the two? Balance of guided vs unguided experience. Do we need to test the correct way to do delivery? Is there prior art we can draw on? I feel I should know. Is this part of the research that's being done here?
\item Big work package on schema vs key and user management to enforce rules in spaces. Only participants who have provably paid should have access to learning material, the ability to input into the assessment system, and the tokenised learning outcome `NFT' or proof.
\item Proof that XR system improve learning outcomes. Also that the proposed systems for micro-transactions and user and schema management give additional headroom for teaching.
\end{itemize}

Notes on build-out
The world database in the shared rooms in the metaverse is the global object master,  educational materials, videos,  audio content and branded objects are fungible tokens authentically proved by rgb client side validation between parties,  only validated ones will be persisted in shared rooms like conferences and classes according to the room schema. That allows educators to monetise their content.  That can work on lightning.  NFT objects between parties like content crafted by participants (coursework, homework) are not on lightning and will attract main chain fees but are rarer. User authentication and communication will be through nostr.

\begin{figure*}[ht]\centering % Using \begin{figure*} makes the figure take up the entire width of the page
	\includegraphics[width=\linewidth]{globalclassroom}
	\caption{Functional elements for infrastructure.}
	\label{fig:globalclassroom}
\end{figure*}

\begin{figure*}[ht]\centering % Using \begin{figure*} makes the figure take up the entire width of the page
	\includegraphics[width=\linewidth]{systemc4}
	\caption{Client server C4 diagrams.}
	\label{fig:globalclassroom}
\end{figure*}
