This chapter identifies an intersectional space across the described technologies, and proposes a valuable and novel software stack, which can enable exploration  and product development. It is useful to briefly look recap the book and the conclusions we have drawn so far.

\section{Summary TL;DR}
\textbf{The points below are the tl;dr for the whole book. They do not currently correspond to the order in the book, but will do at some point
\begin{itemize}
\item There may be an inflection point in the organisational topology of the internet because of trust abuses by the incumbent providers. This is calling itself Web3, but the moniker is fraught with problems and somewhat meaningless. The drivers are real.
\item `The Metaverse' is coming, in some form, at some point. Everyone is positioning in case it's ``soon''.
\item It's not at all clear what it is, or if people want it, but the emergent narrative seems to be ``digital society'' and that obviously should not be dismissed lightly.
\item From a business perspective metaverse is the worst of the remote collaboration tool-kits, and undermines flow, productivity, and interpersonal trust.
\item It's probably technology for technologies sake at this time, but it's a strong attempt, and the investment is real.
\item Excluding Facebook/Meta a lot of the investment is coming from the Web3 speculative bubble, who have a parallel and intersectional metaverse narrative. 
\item Industry has noted the risk, and failures of Meta in this domain, and have latched onto "open metaverse" as a narrative, to de-risk their interest.
\item The current open metaverse is muddy and confused.
\item Anything from a multi-million pound XR studio screen, to a speech audio system, can be a metaverse interface.
\item Constraints (metaverse schema) which are crafted in response to the use case, the interface, and the audience, is where challenges are most profound. This is an opportunity.
\item Trust, accessibility, governance, and safeguarding, are far harder in these real-time distributed social systems.
\item There's a bunch of start-ups and options, interesting research, and some adoption, but scaling is very hard in the face of tepid consumer interest.
\item Industry seem to be missing the point; that open metaverse should mean open source metaverse. There are some options, but they are under developed. This is an opportunity.
\item There is genuine, undeniable interest in digital scarcity. The ownership of digital goods seems natural to younger, digitally native users.
\item This is serviced already by various (gaming) platforms, but they are all isolated ecosystems.
\item Uniting these attempts, with portable (transferable) ``goods'' across the digital universe (metaverse) likely requires a global digital ledger (blockchain).
\item Crypto is igniting imagination on this topic, and is seeing adoption both inside out outside of the metaverse contexts.
\item Crypto is a nightmare; rife with scams, poor technology choices, limited life, and incorrect assumptions.
\item The only thing blockchain/crypto can do well is ``money like networks'', which is a cornerstone of human interaction, and the killer application.
\item Representations of dollars and pounds etc money can ride securely on top of such networks as stablecoins, and this is getting easier to integrate, though there are risks.
\item It's unclear which technology will win, if any, but since the tools exist now they can be integrated now.
\item Legislative and cultural headwinds are significant. There might be no opportunity here in the end, though ``rough game theory'' supports the attempt.
\item Money, identity, and thereby trust, can already be mediated by the Bitcoin network, even without using Bitcoin the asset. This is an opportunity.
\item Digital ownership is nearly possible at scale, low cost, and minimal environmental impact, with the Bitcoin network. This is an opportunity.
\item A topologically flat, inclusive, permissionless, open metaverse, with economically empowered ML and AI actors, which can mediate governance issues, transparently, according to well constructed custom schemas, between cryptographically verifiable economic users (human or AI) is an opportunity.
\item New open source machine learning tooling potentially removes many of the problems with accessibility, creativity, language barriers, safeguarding, and governance. This is a huge opportunity.
\end{itemize}
}


\section{Software stack}
This section needs building out to describe the stack and the choices made, but can be seen in Figure \ref{fig:pyramind} and Figure \ref{fig:highlevelstack}.

\begin{figure*}[ht]\centering 	\includegraphics[width=\linewidth]{pyramid}
	\caption{Pyramid showing the components for sats, stablecoins on lightning, asssets, and trust}
	\label{fig:pyramind}
\end{figure*}

\begin{figure*}[ht]\centering 	\includegraphics[width=\linewidth]{highlevelstack}
	\caption{High level overview showing the components for sats, stablecoins on lightning, asssets, and trust}
	\label{fig:highlevelstack}
\end{figure*}

\section{In camera VFX \& telepresence}
Designing open federated metaverse from a 25 year research foundation
There are serious and under discussed natural social constraints on group behaviours, and these translate into social VR. For instance the ideal meeting size is 6, and this is naturally established in work settings. This has not translated into a metaverse setting where dozens of people routinely crash across one another. In the context of supporting a creative “backstage” world where set planning, production shots, etc can be discussed we believe we have solutions which will get the best out of distributed teams of film-makers.
Leveraging the world's most powerful decentralised computing network to create 
scale and security without high cost
The Bitcoin network is more than just a speculative money like asset, it is the most secure distributed computing system ever built. We can jump on the back of this at almost no cost to enable scale for transfer of value, trust, and digital assets of provenance.
Cryptographically assured end points
With the cryptography tools provided through integration of the Bitcoin network we can also use non-blockchain based secure messaging, and identity proofs. 
Micro transactions in collaborative spaces
New tooling the space allows fractions of a pound or dollar to be exchanged between parties across the world. This means that work can be paid “by the second” both inside and outside of the metaverse. This radically improves creative microtask workflows.
World leading open source machine learning and bot architectures
By integrating Stablity AI tools for image generation, video processing, natural language, and speech to text / text to speech we hope to reduce friction within the backstage worlds.
Creating a narrative arrow from a remote director/producer/DP, through a VP screen into a shoot, and back into a persistent metaverse shared with the public
By linking across these new systems with world class telepresence research we hope to use a single digital context to support senior stakeholders, creatives, technical teams, and the wider public.
New paths to monetisation and digital ownership
This unified digital back end is optimised for flows of money, trust, and digital objects. This is a new area for VP.
Current workstreams:
\begin{itemize}
\item Storyboarding with text2img and dreambooth to add talent and costume ideas before meeting up, as demonstrated in this document \cite{ruiz2022dreambooth}.
\item Collaborative, self hosted, high speed, low detail, economically and cryptographically enabled set design spaces, with near instant language translation (speech to text an speech to speech). Micropayment for cheap international labour. Technology agnostic. Use the screen, audio only, compressed video dial-in, headsets, tablet rendering: (this book).
\item High end telepresence \cite{Roberts2015, OHare2018, Fairchild2017, OHare2016} into the studio/shoot from the virtual set, allowing high value stakeholders to be `present` on set as virtual collaborants with spatial descrimination allowing directional queues. This involved real time human capture like moveAI or the expensive rigs with DSLRs.
\item Novel render pipeline for fast turnaround of final look and feel, taking the rough scene and applying img2img ML with the kind of interframe consistency we are starting to see from the video projects \cite{anonymous2023phenaki}.
\item Text to model pipeline for interactively building key elements with senior stakeholders, pushed from post ideation the the  pre-shoot Unreal content creation \cite{poole2022dreamfusion}.
\item All assets switch over to Unreal metaverse and become consistent (optimised) digital set which can be visited by stakeholders, funders, VIPs etc. Public can visit later for a fee? Digital assets can be bought from the set.
\end{itemize}

\section{Accessible metaverse for pre-viz}
Pre-visualization (or "pre-viz") is a process in which a rough simulation of a visual effect or scene is created prior to its actual production. In the context of LED wall virtual production, pre-viz refers to the creation of a 3D representation of a virtual environment, including the placement of cameras, actors, and other elements, that is then used to plan and test the visual effects and lighting for a live-action scene that will eventually be shot in front of an LED wall.\par
The pre-viz process allows filmmakers and visual effects artists to experiment with different camera angles, lighting, and visual effects before committing to a final version. This helps to save time and resources during actual production by reducing the need for multiple takes or re-shoots. Additionally, it allows the filmmakers to see how the final product will look before committing to it, which can help to avoid costly mistakes or changes down the line.\par
The LED wall virtual production process typically involves using a combination of 3D animation software, motion capture technology, and real-time rendering to create a virtual environment that accurately reflects the physical environment in which the scene will be shot. The pre-viz process is then used to plan and test the various visual effects, lighting, and camera angles that will be used in the final production.\par 
Our collaborative software stack is potentially ideally suited to some of this pre-viz work, especially when combined with the power of machine learning, and live linked into Unreal so that changes by stakeholders enter the pre-production pipeline in a seamless way.
\section{Novel VP render pipeline}
Putting the ML image generation on the end of a real-time tracked camera render pipeline might remove the need for detail in set building. To describe how this might work, the set designer, DP, director, etc will be able to ideate in a headset based metaverse of the set design, dropping very basic chairs, windows, light sources whatever. There is -no need- then to create a scene in detail. If the interframe consistency (img2img) can deliver then the output on the VP screen can simply inherit the artistic style from the text prompts, and render production quality from the basic building blocks. Everyone in the set (or just DP/director) could then switch in headset to the final output and ideate (verbally) to create the look and feel (lens, bokeh, light, artistic style etc). This isn’t ready yet as the frames need to generate much faster (100x), but it’s very likely coming in months not years. This ``next level pre-vis'' is being trailed in the Vircadia collaborative environment described in this book, and can be seen illustrated in Figure \ref{fig:vircadiasd}.\par
\begin{figure}[ht]\centering 	\includegraphics[width=\linewidth]{vircadiasd}
	\caption{Top panel is a screen grab from Vircadia and the bottom panel is a quick pass through img2img from Stable Diffusion.}
	\label{fig:vircadiasd}
\end{figure}

This can be done now through the use of camera robots. A scene can be built in basic outline, the camera tracks can be encoded into the robot, and the scene can be rapidly post rendered by Stability with high inter frame consistency.\par
With the help of AI projects such as \href{https://nv-tlabs.github.io/LION/}{LION} it may be possible to pass simple geometry and instructions to ML systems which can create complex textured geometry back into the scene.
\begin{figure}[ht]\centering 	\includegraphics[width=\linewidth]{robotvp}
	\caption{Robot VP}
	\label{fig:robotvp}
\end{figure}

\section{Money in metaverses}
\subsubsection{Global collaboration and remuneration}
In the book ``Ghosts of my life'' \cite{fisher2014ghosts} Fisher asserts that there has been a slowing, even a `cancellation' of creative progress in developed societies, their art, and their media. His contention is that neoliberalism itself is to blame. He says\\
\textit{``It is the contention of this book that 21st-century culture is marked by the same anachronism and inertia which afflicted Sapphire and Steel in their final adventure. But this status has been buried, interred behind a superficial frenzy of ‘newness’, of perpetual movement. The ‘jumbling up of time’, the montaging of earlier eras, has ceased to be worthy of comment; it is now so prevalent that it is no longer even noticed.''}

It is the feeling of the authors of this book that the creative and inspirational efforts of the whole world may be needed to heal these deep wounds. It is possible that by connecting creatives with very different global perspectives, directly into `Western' production pipelines, that we will be able to see the shape of this potential.
\subsection{ML actors and blockchain based bots}
Stablity AI is an open source imitative to bring ML/AL capabilities to the world. This is a hugely exciting emergent area and much more will be developed here.
\subsection{AI economic actors in mixed reality}
AI actors can now be trusted visually \cite{nightingale2022ai}. We have some thinking on this which links the external web to our proposed metaverse. There is work in the community working on economically empowered bots which leverage Nostr and RGB to perform functions within our metaverse, and outside in the WWW, as well as interacting economically through trusted cryptography with other bots, anywhere, and human participants, anywhere. This is incredibly powerful and is assured by the Bitcoin security model. Imagine being able to interact with a bot flower seller representing all the real world florists it had found. In the metaverse you could handle the flowers and take advice and guidance from the bot agent, then it would be able to take your money to buy you flowers to send to a real world address, and later find you to tell you when it's delivered. These possibilities are endless. The AI chat element, the AI translation of images on websites to 3D assets in the Metaverse are difficult but possible challenges, but the secure movement of money from the local context in the metaverse to the real world is within reach using these bots, and they are completely autonomous and distributed.
\section{Our socialisation best practice}
\subsubsection{Identity}
We will base our identity and object management on Nostr public/private key pairs. The public key of these enable lightning based exchange of value globally. %Additional work will be needed to allow objects generated through RGB to be passed between federated worlds.
we plan to operate Nostr in multiple modes. Linking flossverse ``rooms'' will be a Nostr bot to bot system within the private relay mesh. This can also synchronise large amounts of data by leveraging torrents \href{https://iris.to/#/settings}{negotiated by Nostr}. Human to human text chat across and within instances is two 'types' kind of private nostr tag within the private relays mesh. External connectivity to web and nostr apps is just the public relay tags outbound. We don't need to store data external to the flossverse system, though access is obviously possible through the same torrent network.
\subsubsection{Webs of trust}
Webs of trust will be built within worlds using economically costly (but private) social rating systems, between any actor, human or AI. It should be too costly to attack an individual aggressively. This implies an increased weighting for scores issued in short time periods. Poorly behaving AI's will eventually be excluded through lack of funds.
\subsubsection{Integration of 'good' actor AI entities}
Gratitude practice should be encouraged between AI actors to foster trust and wellbeing in human observers. ``It's nice to be nice'' should be incentivised between all parties''. This could include tipping and trust nudging through the social rating system. Great AI behaviour would result in economically powerful entities.
\subsection{Emulation of important social cues}
\href{https://www.cleverclassroomsdesign.co.uk/general-5}{Classroom layout}
\subsubsection{Behaviour incentives, arbitration, and penalties}
Collapses of trust and abuse will trigger flags from ML based oversight, which will create situational records and payloads of involved parties to unlock with their nostr private keys. ML red flagged actors will be finacially penalised but have access to human arbitration using their copy of the data blob. Nothing will be stored except by the end users.
\subsection{Federations of webs of trust and economics}
Nostr is developing fast enough to provide global bridges between metaverse instances.
\section{Security evaluation}
As part of developing our stack we will penetration test the deployment as detailed using \href{https://hexway.io/}{Hexway}

%\input{10_otherusecases}
\section{notes for later}
Notes on build-out
The world database in the shared rooms in the metaverse is the global object master,  educational materials, videos,  audio content and branded objects are fungible tokens authentically proved by rgb client side validation between parties,  only validated ones will be persisted in shared rooms like conferences and classes according to the room schema. That allows educators to monetise their content.  That can work on lightning.  NFT objects between parties like content crafted by participants (coursework, homework) are not on lightning and will attract main chain fees but are rarer. User authentication and communication will be through nostr.

\begin{figure*}[ht]\centering % Using \begin{figure*} makes the figure take up the entire width of the page
	\includegraphics[width=\linewidth]{globalclassroom}
	\caption{Functional elements for infrastructure.}
	\label{fig:globalclassroom}
\end{figure*}

\begin{figure*}[ht]\centering % Using \begin{figure*} makes the figure take up the entire width of the page
	\includegraphics[width=\linewidth]{systemc4}
	\caption{Client server C4 diagrams.}
	\label{fig:globalclassroom}
\end{figure*}
