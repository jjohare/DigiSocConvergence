\section{Introduction and Problem Definition}

This chapter identifies an intersectional space across the described technologies, and proposes a valuable and novel software stack, which can enable exploration  and product development. It is useful to briefly look at the Venn disgram we began with, and recap the book and the conclusions we have drawn so far.

\begin{figure*}[ht]\centering % Using 
	\includegraphics[width=\linewidth]{landscapevenn}
	\caption{Another look at the diagram of intersections.}
	\label{fig:landscapevenn}
\end{figure*}


\section{Introduction and Problem Definition}
\subsection{Overview of the Metaverse and Digital Society}
The concept of the Metaverse has gained significant attention, with various stakeholders positioning themselves to capitalize on its potential. While it remains unclear exactly what form the Metaverse will take or whether people truly desire it, it is evident that digital society holds considerable promise. We see advantage less in social metaverse, and more in solving business to business technical use cases where professionals with visual technical problems, or training requirements, gather in collaborative spaces.

\subsection{Challenges in Trust, Accessibility, Governance, and Safeguarding}
The Metaverse faces numerous challenges, including poor adoption rates, overstated market need, and a lack of genuine digital society use cases. Meanwhile trust abuses by incumbent providers have led to potential inflection points in the organization of the wider internet. Moreover, emerging markets and less developed nations face barriers to entry due to inadequate identification, banking infrastructure, and computing power. There is an opportunity to build pervasive digital spaces with a different and more open foundation, learning from these lessons.

\subsection{The Need for Modular Open-Source Solutions}
Developing a topologically flat, inclusive, permissionless, federated, and open Metaverse is essential to address these challenges. By using open-source AI tooling and large language models, it is possible to improve creativity, safeguarding, and governance, while breaking down language barriers and accessibility challenges. Implementing secure, trusted, and task-appropriate solutions can promote collaboration and innovation across various industries.

\subsection{Technical problem definition}
Problems are
\begin{itemize}
\item evergreen telecollaboration around technical issues
\item exchange of good, services, money within systems, without friction
\item identity management within virtual spaces
\item access to information in the extrinsic world from within the tool
\item federation of instances without overhead (scaling)
\item seamless access to personal information within and without the collaborative system
\item ability to take advantage of supporting smart support agents (bots, etc) throughout
\item governance, trust, safeguarding
\end{itemize}

\section{Proposed Layered Framework}
\subsection{Layer 1: Bitcoin, Lightning, and Nostr protocols}
Distributed financial tooling and digital assets, have ignited imagination and adoption within and outside of the Metaverse context. A global ledger could unite isolated digital ecosystems and enable the transfer of portable `goods' across digital society. An open-source Metaverse should emphasize the development and adoption of open protocols and data formats. The Nostr protocol, for instance, might link and federate mixed reality spaces, providing identity assurances and mediating data synchronization while maintaining reasonably strong cryptography. This also allows integration with the legacy web through ubiquitous web sockets.
\subsection{Layer 2: Modular human computer interface}
Collaborative global networks for training, research, biomedical, and creative industries can be developed using immersive and accessible environments. Engaging with ideas from diverse cultural backgrounds can enrich the overall user experience.
\subsection{Layer 3: Natural Language AI and Generative ML Integration}
Integrating AI and machine learning into the Metaverse can promote supported creativity and augmented intelligence. By incorporating generative ML technologies, users can ideate in simple immersive spaces while instantly creating scenes that can be stylized using verbal commands in real-time.
\section{Modular Open-Source System Components}
\subsection{Infrastructure Components: Blockchain, Networking, and Storage Solutions}
Blockchain technology and crypto, despite their issues, have the potential to revolutionize the way digital society operates by enabling "money-like networks," which are a cornerstone of human interaction. Representations of traditional currencies can ride securely on top of these networks as stablecoins, opening up global collaborative working practices, especially for emerging markets.

\subsection{Interoperability Components: APIs, SDKs, and Data Standards}
Industry players have noted the risk and failures associated with closed systems like Meta and are embracing the "open Metaverse" narrative to de-risk their interests. To enable a truly open and interoperable Metaverse, it is crucial to develop open-source APIs, SDKs, and data standards that allow different platforms to communicate and exchange information.

\subsection{User Experience Components: UI/UX, AI Assistants, and Generative Content Creation}
To create a more inclusive and accessible Metaverse, user experience components like UI/UX design, AI assistants, and generative content creation should be tailored to a wide range of users. The integration of AI and machine learning technologies, such as GPT-4, can facilitate more seamless interactions and creative content generation, fostering a more engaging and immersive Metaverse experience.

\section{System Implementation}
\subsection{Scalable and Federated Network Solutions}
To ensure scalability and optimal performance, the open-source system should rely on distributed and federated networks, which can better manage latency and performance issues.
\subsection{Open Protocols and APIs}
Promoting open protocols, APIs, and data standards is essential to enable seamless communication and interaction between different platforms and systems within the Metaverse.
\subsection{Integration of Nostr Protocol}
The Nostr protocol can potentially be used to link and federate small and useful mixed reality spaces, providing identity assurances, mediating data synchronization, and facilitating machine-to-machine communication while maintaining reasonably strong cryptography throughout the network.

\section{Application case studies}
As we have seen in the `collaborative mixed reality' chapter, these tools are best deployed where some human conversational cues (pointing, looking etc) are required in the context of a shared task, which is mostly visual in nature. This is a surprisingly small amount of tasks, though we have seen that the emergence of AI means that increasingly natural language AI can streamline communication, while visual generative ML can suggest design alternatives or improvements based on existing data and user preferences. This is very likely to expand the use space and this section will attempt to explain how as the case studies are explained.\par 
We will employ the acronym for collaborative virtual environment (CVE) from this stage, and it's going to come up a lot. There will be far less references in this section for brevity.
\subsection{Classic use cases}
Small teams working on product, architectural, or industrial design can benefit from CVEs that allow them to visualize, modify, and iterate on 3D models in real-time. 
\subsection{Virtual training and simulation}
CVEs can facilitate skill development and training in various industries, such as healthcare, military, aviation, and emergency response. Trainees can practice procedures in a virtual environment, with natural language AI providing instructions, explanations, or feedback, and visual generative ML potentially customizing scenarios to adapt to each user's learning curve.
\subsection{Remote teleconferencing}
In situations where face-to-face communication is not feasible, CVEs can enable remote teams to work together on shared visual tasks like planning events, brainstorming ideas, or reviewing documents. Natural language AI can transcribe and analyse spoken conversations, providing real-time translations or summaries, while visual generative ML can create visual aids or dynamically update shared documents. This may especially be useful in complex multinational legal and/or negotiation applications, though very clearly the risks of using assisting ML tooling increases. 
\subsection{Virtual art \& media collaboration}
Artists, animators, and multimedia professionals can collaborate in CVEs to create and develop their projects, such as films, animations, or video games. Natural language AI can help in storyboarding, scriptwriting, or character development, while visual generative ML can generate new visuals or adapt existing assets based on user input and style preferences.
\subsection{Data visualization and analysis}
Small teams working with large datasets can use CVEs to visually explore and analyze data in a more intuitive and engaging way. Natural language AI can help users query and interact with the data using conversational interfaces, while visual generative ML can generate new visualizations based on patterns and trends identified in the data.
\subsection{Education and virtual classrooms}
    Educators can leverage CVEs to create immersive learning experiences that engage students in collaborative activities, such as group projects, problem-solving, or scientific experiments. Natural language AI can facilitate communication, provide personalized tutoring, or assess student progress, while visual generative ML can create customized educational content based on individual needs and interests.
\subsection{Virtual labs and scientific research}
Researchers can use CVEs to conduct experiments, visualize complex data, or simulate real-world conditions in a controlled environment. Natural language AI can assist in interpreting results, automating lab protocols, or identifying research gaps, while visual generative ML can generate predictions or models based on existing data to support hypothesis testing and decision-making.


\subsection{Media and entertainment}


\subsection{Biomedical}
Collaborative Virtual Environments (CVEs) have immense potential in the fields of chemical and medical molecular modeling. By incorporating natural language AI and visual generative machine learning, these environments can revolutionize the way scientists and researchers approach complex chemical and biological problems. Here are some specific use cases:

    Drug design and discovery:
    CVEs can enable researchers to collaboratively visualize and manipulate 3D molecular structures in real-time, identifying potential drug candidates and understanding protein-ligand interactions. Natural language AI can help users interact with the molecular data, while visual generative ML can predict potential binding sites, energetics, or toxicity profiles based on existing knowledge.

    Protein structure prediction and modeling:
    Small teams can work together to predict protein structures, visualize folding patterns, and model protein-protein or protein-nucleic acid interactions. Natural language AI can assist in annotating and explaining the structural features, while visual generative ML can generate new structural hypotheses based on sequence alignments, homology modeling, and experimental data.

    Molecular dynamics simulations:
    CVEs can facilitate collaboration on complex molecular dynamics simulations, allowing researchers to analyze and visualize trajectories, energetics, and conformational changes. Natural language AI can help users navigate through simulation data and identify relevant patterns, while visual generative ML can create new conformations or predict the effects of mutations on protein stability and function.

    Cheminformatics and QSAR modeling:
    Researchers can leverage CVEs to develop and validate Quantitative Structure-Activity Relationship (QSAR) models, which predict the biological activity of chemical compounds based on their structural properties. Natural language AI can facilitate the exploration and interpretation of chemical descriptors, while visual generative ML can suggest new compounds with desired properties or optimize existing molecular scaffolds.

    Metabolic pathway modeling:
    Small teams can work together to build and analyze metabolic pathways, integrating experimental data and computational models to understand the underlying mechanisms and predict metabolic fluxes. Natural language AI can assist in annotating and explaining pathway components, while visual generative ML can generate new pathway hypotheses or predict the effects of genetic or environmental perturbations.

    Biomolecular visualization and virtual reality:
    CVEs can offer immersive, interactive experiences for exploring biomolecular structures and dynamics, enhancing researchers' understanding of complex biological systems. Natural language AI can provide contextual information or guide users through molecular landscapes, while visual generative ML can create new visualizations or adapt existing ones based on user preferences and insights.

    Collaborative molecular docking and virtual screening:
    Small teams can use CVEs to perform collaborative molecular docking and virtual screening, which involve predicting the binding of small molecules to target proteins. Natural language AI can help users refine docking parameters and analyze results, while visual generative ML can generate alternative poses or suggest new compounds for screening based on user feedback and existing data.
    Choose a suitable mixed reality platform: Select a platform that allows the creation of simple, accessible shared mixed reality environments. Consider open-source options like Mozilla Hubs or JanusVR, which offer customizable and collaborative virtual spaces.

    Integrate open-source biomed software: Incorporate open-source biomed software such as PyMOL, Chimera, or VMD for molecular visualization and analysis. These tools can be integrated into the mixed reality environment for real-time interaction, allowing students and instructors to collaboratively visualize and manipulate molecular structures.

    Leverage AI and machine learning: Integrate AI and ML algorithms like those found in DeepChem, RDKit, or Open Babel to aid in the discovery and optimization of novel compounds. These tools can help predict molecular properties, perform virtual screening, and optimize lead compounds for drug development. By incorporating AI and ML, students can learn how to apply these cutting-edge techniques to real-world problems in biomedicine.

    Establish a distributed proof system: Utilize a distributed proof system like the Nostr protocol to federate the small virtual classroom environments. This will allow for seamless collaboration among students and faculty while maintaining security and data integrity.

    Create digital objects for interaction: Use digital objects such as 3D molecular models, virtual lab equipment, and interactive simulations to create an immersive learning experience. These digital objects can be shared and manipulated in real-time, promoting collaborative learning and problem-solving.

    Implement accessible interfaces: Ensure that the virtual classroom environment is accessible to all students, including those with disabilities. Utilize AI-driven tools like StabilityAI to help with language barriers, safeguarding, and governance, enabling a more inclusive learning experience.

    Foster collaboration and communication: Encourage students and faculty to collaborate on projects, share ideas, and ask questions in real-time using voice chat, text chat, or other communication tools integrated into the mixed reality environment.

    Provide training and support: Offer training sessions and support materials to help students and faculty become familiar with the mixed reality environment, the integrated biomed software, and AI/ML tools.

    Monitor progress and adjust as needed: Regularly review student progress, gather feedback, and adjust the virtual classroom environment as needed to ensure an effective and engaging learning experience.
\subsection{Collaborative Design and Prototyping}
Utilizing open-source systems and AI-assisted tools can enable more efficient and creative collaboration in design and prototyping processes. Teams from diverse cultural backgrounds can work together seamlessly, creating a rich pool of ideas and innovations.

\subsection{Training, Simulation, and Education}
The modular open-source system can be applied to various training, simulation, and education scenarios. By integrating AI and generative ML technologies, these tools can provide personalized learning experiences and create realistic simulations that cater to different learning styles and requirements.

\subsection{Remote Collaboration and Teleconferencing}
As remote work becomes more prevalent, the Metaverse can provide a more engaging and immersive platform for collaboration and teleconferencing. The open-source system can be adapted to serve various industries, making remote collaboration more efficient and inclusive.

\subsection{Chemical and Medical Molecular Modeling}
In fields like chemical and medical molecular modeling, the integration of AI and generative ML technologies can significantly improve collaboration and innovation. Teams can work together in immersive environments to visualize complex molecular structures, benefiting from real-time AI-generated visuals and natural language processing.

\subsection{Creative Industries and Generative Art}
The combination of AI, ML, and open-source systems can revolutionize the creative industries by offering new avenues for generative art, content creation, and collaboration. Supported creativity and augmented intelligence can break down barriers and enable artists to explore new ideas and techniques, enriching the creative landscape.

\subsection{Case Study: Biodiversity Monitoring and Data Exchange with Isolated Communities}
Biodiversity monitoring in and around isolated communities is challenging due to limited access and resources. Traditional methods rely on sporadic visits by grant-funded academics, which can introduce biases and lack regular follow-up. Engaging local communities may also introduce incentive structures and biases and may not be sustainable without continuous investment.

We propose an open-source collaboration infrastructure that leverages advanced technologies such as multi-modal large language models (LLMs), satellite communication, and cryptocurrency networks to facilitate sustainable and reliable biodiversity monitoring and data exchange in isolated communities.

\subsubsection{Language Model and Voice Interface}
A specialized multi-modal LLM can be trained on local language, culture, customs, and environmental data such as flora, fauna, biotica, soil pH, and rainfall. This LLM can be accessed through a voice interface by the local community, enabling data entry and knowledge exchange in the local language. The voice interface can help overcome literacy barriers and make the system more accessible to a diverse range of community members.

\subsubsection{Data Collection and Storage}
Photographs and metadata can be logged and collected by a remote team at a later date or uploaded regularly through a satellite link (e.g., Starlink). The data storage system can be designed to be both secure and resilient, ensuring that the collected data remains available and accessible for future analysis and decision-making.

\subsubsection{Live Connection and Model Tuning}
A live connection with the academic team allows for model tuning through prompt engineering, vector database updates, and efficient Lora models, potentially offering timely advice for ecosystem interventions. Real-time communication between the community and academic teams can help identify areas of concern and rapidly adapt the LLM to address emerging challenges.

\subsubsection{Ecosystem Interventions}
The proposed infrastructure would be particularly valuable in areas facing novel disease encroachment, invasive species, active hydrology, shifting aquatic conditions, microplastic hotspots, changing microclimates, or volcanic activity. By providing real-time advice and guidance, the LLM can help communities make informed decisions about ecosystem management and conservation efforts.

\subsubsection{Incentives and Education}
Incentivizing community engagement could be achieved by providing access to the LLM for educational purposes, as demonstrated by the refugee camp e-prize (ref). Local schools and community centers can leverage the LLM as a resource for teaching environmental stewardship and ecological awareness, while also promoting digital literacy and technology skills.

\subsubsection{Monetization and Blockchain Integration}
Monetizing these systems could involve using chaumian mints such as Cashu or Fedimint, under the control of local community leaders, mediated through the global Bitcoin satellite network (Blockstream), enabling digital dollar payments to communities via low-end mobile handsets. By integrating blockchain technology, the proposed infrastructure can ensure secure, transparent, and efficient financial transactions, while also opening up new economic opportunities for isolated communities.

\subsubsection{Visual Training Support Systems}
The infrastructure could be further extended to visual training support systems using low-cost, low-power components. These systems could provide interactive, immersive learning experiences for community members, helping them better understand the local ecosystem and develop skills in environmental monitoring and management.

\subsubsection{Solar Infrastructure}
To minimize the environmental impact and ensure energy sustainability, the proposed infrastructure can be powered by solar energy. This approach will enable the system to operate independently of local power grids, reducing the overall operational costs and maintenance requirements.

\subsubsection{Open-Source Collaboration}
By linking this case study to the open-source collaboration infrastructure discussed earlier, we can create an inclusive, permissionless, federated, and economically empowered system that addresses the challenges of biodiversity monitoring while promoting digital society values such as trust, accessibility, and governance. This collaborative approach can help drive innovation and ensure that the proposed solutions are both scalable and adaptable to the unique needs of different communities and ecosystems.

\subsubsection{Risk Mitigation and Ethical Considerations}
While implementing such an infrastructure, care must be taken to address potential unintended consequences of embedding these inference systems in communities. It is essential to involve the local communities in the development and deployment process, ensuring that their perspectives, values, and traditions are respected and preserved.

Moreover, it is crucial to establish a robust ethical framework for the use of AI technologies, considering potential issues related to privacy, data security, and cultural sensitivity. Regular audits and monitoring can be implemented to ensure that the infrastructure remains transparent, accountable, and aligned with the needs and expectations of the communities it serves.

\subsubsection{Capacity Building and Local Empowerment}
An essential aspect of this initiative is building capacity and empowering local communities to take ownership of their environment and resources. By providing training, resources, and support, the proposed infrastructure can help communities develop the skills and knowledge needed to manage their ecosystems effectively.

Furthermore, the integration of digital tools and technologies can promote digital inclusion and bridge the digital divide, giving isolated communities access to valuable information and resources while fostering a sense of global connectedness and collaboration.

\subsubsection{Future Outlook and Potential Impact}
The proposed open-source collaboration infrastructure for biodiversity monitoring and data exchange has the potential to transform how isolated communities interact with their environment, enabling them to make informed decisions about conservation and ecosystem management.

By leveraging cutting-edge technologies such as LLMs, satellite communication, and blockchain networks, this approach can create a more inclusive, transparent, and accessible system for environmental monitoring and stewardship. The successful implementation of this infrastructure could pave the way for similar initiatives in other regions and ecosystems, promoting global collaboration and innovation in the pursuit of a more sustainable and equitable world.

\section{Overcoming Challenges and Barriers}
\subsection{Addressing Trust, Accessibility, and Governance Issues}
To create a successful open-source Metaverse, it is crucial to address trust, accessibility, and governance challenges. By integrating decentralized and secure technologies such as blockchain and distributed ledger systems, a more transparent and trustworthy infrastructure can be established.

\subsection{Ensuring Safeguarding and Privacy Compliance}
Protecting user privacy and ensuring safeguarding is vital for any digital society platform. The open-source system must be developed in compliance with legislative and cultural norms while maintaining the balance between user privacy and the need for identity verification and data management.

\subsection{Managing Scalability, Performance, and Latency}
As the Metaverse continues to grow, it is crucial to ensure that the open-source system can scale effectively and maintain optimal performance. By using distributed and federated networks, the system can better manage latency and performance issues, ensuring a seamless user experience.

\subsection{Promoting Open Standards and Interoperability}
For the Metaverse to truly thrive, it is essential to promote open standards and interoperability among various platforms and systems. This can be achieved by fostering collaboration between industry stakeholders, encouraging the development of open protocols, APIs, and data standards, and actively supporting the open-source community.

\section{Future Outlook and Potential Developments}
\subsection{Advancements in AI and Generative ML Technologies}
As AI and generative ML technologies continue to evolve, their integration into the Metaverse will further enhance user experiences and create new opportunities for innovation. The release of models like GPT-4 have already prompted debate about general AI \cite{bubeck2023sparks, perez2022discovering} (Figure \ref{fig:rlhf}). It seems unavoidable that this will all impact on the Metaverse and digital society.

\begin{figure*}[ht]\centering 	\includegraphics[width=\linewidth]{rlhf}
	\caption{Models exhibit uncanny behaviours.}
	\label{fig:rlhf}
\end{figure*}


\subsection{Inclusive Digital Society}
By overcoming barriers to entry for emerging markets and less developed nations, a more inclusive digital society can be fostered. This inclusivity will empower new ideas and perspectives, leading to a richer and more diverse digital landscape.

\subsection{Integration of Spatial and Augmented Reality Technologies}
The incorporation of spatial and augmented reality technologies can expand the possibilities within the Metaverse, allowing for more immersive and interactive experiences. These technologies have the potential to reshape digital society and redefine the ways in which people interact with digital environments.

\subsection{Economic Empowerment and Integration of AI Actors}
The creation of an open and economically empowered Metaverse, in which AI actors can mediate governance issues and participate in economic transactions, can lead to a more efficient and dynamic digital ecosystem. This integration will enable new business models and opportunities for all users, both human and AI.

\subsection{Continuous Evolution and Adaptation}
As the digital landscape continues to evolve, the open-source Metaverse system must be flexible and adaptable to meet changing needs and expectations. Continuous innovation and collaboration within the industry will be crucial for the success and longevity of the Metaverse as a transformative digital society platform.

\section{Conclusion and Final Thoughts}
\subsection{Embracing the Open-Source Metaverse Vision}
To create a truly transformative and inclusive digital society, it is essential to embrace the vision of an open-source Metaverse. By fostering collaboration, promoting open standards, and integrating advanced AI and ML technologies, the Metaverse can become a platform that serves societal and business needs.

\subsection{Addressing Challenges and Learning from Past Failures}
Learning from past failures and addressing challenges head-on will be critical to the successful development of an open-source Metaverse. Trust, accessibility, governance, and safeguarding issues must be thoughtfully considered and addressed to build a secure and user-friendly platform.

\subsection{Unlocking New Opportunities and Use Cases}
The integration of AI, ML, and cutting-edge technologies within the Metaverse can unlock new opportunities and use cases across various industries, including education, research, biomedical, and creative fields. By building on a modular open-source system, these opportunities can be explored and realized to their full potential.

\subsection{Fostering Collaboration and Inclusivity}
Creating an inclusive digital society is a key goal for the open-source Metaverse. By breaking down barriers and making the platform accessible to a wider audience, new ideas and perspectives will enrich the digital landscape and drive innovation.

\subsection{Shaping the Future of Digital Society}
As the Metaverse continues to evolve and grow, it will play an increasingly important role in shaping the future of digital society. By embracing an open-source vision, overcoming challenges, and unlocking new opportunities, the Metaverse can become a powerful platform that transforms how people live, work, and interact in the digital world.
\subsection{Industry Conversations and Stakeholder Collaboration}
Continued dialogue and collaboration among industry stakeholders are vital to ensuring the successful development of the open-source Metaverse. By engaging in conversations and understanding the cautious appetite for the ideas presented, the community can work together to shape the future of digital society and overcome the challenges that lie ahead.


\section{Software stack}
This section needs building out to describe the stack and the choices made, but can be seen in Figure \ref{fig:pyramind} and Figure \ref{fig:highlevelstack}.

\begin{figure*}[ht]\centering 	\includegraphics[width=\linewidth]{pyramid}
	\caption{Pyramid showing the components for sats, stablecoins on lightning, asssets, and trust}
	\label{fig:pyramind}
\end{figure*}

\begin{figure*}[ht]\centering 	\includegraphics[width=\linewidth]{highlevelstack}
	\caption{High level overview showing the components for sats, stablecoins on lightning, asssets, and trust}
	\label{fig:highlevelstack}
\end{figure*}

At this time we favour the following component units, with alternatives in brackets.
\begin{itemize}
\item Collaborative space - Vircadia [Omniverse, Open3D foundation, Unreal]
\item Distributed truth - Bitcoin testnet [Main net]
\item Digital Objects - RGB [Ordinals, Pear credits]
\item Messaging and sync - Nostr 
\item Identity - Nostr [Bluesky ION, Slashtags]
\item Fiat money xfer - Taro testnet [Pear credits, RGB, Taro main net]
\item Hardware signing - Seed signer [any hardware wallet]
\item Small group banking - Fediment [chaumian ecash]
\item Local wallet - Mutiny [bitkit, and lightning wallet]
\item Machine learning text - Alpaca [ChatGPT etc]
\item Machine learning image - Stable diffusion [midjourney, Dall-E]
\item Object tracking - Nostr [LnBits accounts]
\end{itemize}

\section{In camera VFX \& telepresence}
Designing open federated metaverse from a 25 year research foundation
There are serious and under discussed natural social constraints on group behaviours, and these translate into social VR. For instance the ideal meeting size is 6, and this is naturally established in work settings. This has not translated into a metaverse setting where dozens of people routinely crash across one another. In the context of supporting a creative “backstage” world where set planning, production shots, etc can be discussed we believe we have solutions which will get the best out of distributed teams of film-makers.
Leveraging the world's most powerful decentralised computing network to create 
scale and security without high cost
The Bitcoin network is more than just a speculative money like asset, it is the most secure distributed computing system ever built. We can jump on the back of this at almost no cost to enable scale for transfer of value, trust, and digital assets of provenance.
Cryptographically assured end points
With the cryptography tools provided through integration of the Bitcoin network we can also use non-blockchain based secure messaging, and identity proofs. 
Micro transactions in collaborative spaces
New tooling the space allows fractions of a pound or dollar to be exchanged between parties across the world. This means that work can be paid “by the second” both inside and outside of the metaverse. This radically improves creative microtask workflows.
World leading open source machine learning and bot architectures
By integrating Stablity AI tools for image generation, video processing, natural language, and speech to text / text to speech we hope to reduce friction within the backstage worlds.
Creating a narrative arrow from a remote director/producer/DP, through a VP screen into a shoot, and back into a persistent metaverse shared with the public
By linking across these new systems with world class telepresence research we hope to use a single digital context to support senior stakeholders, creatives, technical teams, and the wider public.
New paths to monetisation and digital ownership
This unified digital back end is optimised for flows of money, trust, and digital objects. This is a new area for VP.
Current workstreams:
\begin{itemize}
\item Storyboarding with text2img and dreambooth to add talent and costume ideas before meeting up, as demonstrated in this document \cite{ruiz2022dreambooth}.
\item Collaborative, self hosted, high speed, low detail, economically and cryptographically enabled set design spaces, with near instant language translation (speech to text an speech to speech). Micropayment for cheap international labour. Technology agnostic. Use the screen, audio only, compressed video dial-in, headsets, tablet rendering: (this book).
\item High end telepresence \cite{Roberts2015, OHare2018, Fairchild2017, OHare2016} into the studio/shoot from the virtual set, allowing high value stakeholders to be `present` on set as virtual collaborants with spatial descrimination allowing directional queues. This involved real time human capture like moveAI or the expensive rigs with DSLRs.
\item Novel render pipeline for fast turnaround of final look and feel, taking the rough scene and applying img2img ML with the kind of interframe consistency we are starting to see from the video projects \cite{anonymous2023phenaki}.
\item Text to model pipeline for interactively building key elements with senior stakeholders, pushed from post ideation the the  pre-shoot Unreal content creation \cite{poole2022dreamfusion}.
\item All assets switch over to Unreal metaverse and become consistent (optimised) digital set which can be visited by stakeholders, funders, VIPs etc. Public can visit later for a fee? Digital assets can be bought from the set.
\end{itemize}

\section{Accessible metaverse for pre-viz}
Pre-visualization (or "pre-viz") is a process in which a rough simulation of a visual effect or scene is created prior to its actual production. In the context of LED wall virtual production, pre-viz refers to the creation of a 3D representation of a virtual environment, including the placement of cameras, actors, and other elements, that is then used to plan and test the visual effects and lighting for a live-action scene that will eventually be shot in front of an LED wall.\par
The pre-viz process allows filmmakers and visual effects artists to experiment with different camera angles, lighting, and visual effects before committing to a final version. This helps to save time and resources during actual production by reducing the need for multiple takes or re-shoots. Additionally, it allows the filmmakers to see how the final product will look before committing to it, which can help to avoid costly mistakes or changes down the line.\par
The LED wall virtual production process typically involves using a combination of 3D animation software, motion capture technology, and real-time rendering to create a virtual environment that accurately reflects the physical environment in which the scene will be shot. The pre-viz process is then used to plan and test the various visual effects, lighting, and camera angles that will be used in the final production.\par 
Our collaborative software stack is potentially ideally suited to some of this pre-viz work, especially when combined with the power of machine learning, and live linked into Unreal so that changes by stakeholders enter the pre-production pipeline in a seamless way.
\section{Novel VP render pipeline}
Putting the ML image generation on the end of a real-time tracked camera render pipeline might remove the need for detail in set building. To describe how this might work, the set designer, DP, director, etc will be able to ideate in a headset based metaverse of the set design, dropping very basic chairs, windows, light sources whatever. There is -no need- then to create a scene in detail. If the interframe consistency (img2img) can deliver then the output on the VP screen can simply inherit the artistic style from the text prompts, and render production quality from the basic building blocks. Everyone in the set (or just DP/director) could then switch in headset to the final output and ideate (verbally) to create the look and feel (lens, bokeh, light, artistic style etc). This isn’t ready yet as the frames need to generate much faster (100x), but it’s very likely coming in months not years. This ``next level pre-vis'' is being trailed in the Vircadia collaborative environment described in this book, and can be seen illustrated in Figure \ref{fig:vircadiasd}.\par
\begin{figure}[ht]\centering 	\includegraphics[width=\linewidth]{vircadiasd}
	\caption{Top panel is a screen grab from Vircadia and the bottom panel is a quick pass through img2img from Stable Diffusion.}
	\label{fig:vircadiasd}
\end{figure}

This can be done now through the use of camera robots. A scene can be built in basic outline, the camera tracks can be encoded into the robot, and the scene can be rapidly post rendered by Stability with high inter frame consistency.\par
With the help of AI projects such as \href{https://nv-tlabs.github.io/LION/}{LION} it may be possible to pass simple geometry and instructions to ML systems which can create complex textured geometry back into the scene.
\begin{figure}[ht]\centering 	\includegraphics[width=\linewidth]{robotvp}
	\caption{Robot VP}
	\label{fig:robotvp}
\end{figure}

\section{Money in metaverses}
\subsubsection{Global collaboration and remuneration}
In the book ``Ghosts of my life'' \cite{fisher2014ghosts} Fisher asserts that there has been a slowing, even a `cancellation' of creative progress in developed societies, their art, and their media. His contention is that neoliberalism itself is to blame. He says\\
\textit{``It is the contention of this book that 21st-century culture is marked by the same anachronism and inertia which afflicted Sapphire and Steel in their final adventure. But this status has been buried, interred behind a superficial frenzy of ‘newness’, of perpetual movement. The ‘jumbling up of time’, the montaging of earlier eras, has ceased to be worthy of comment; it is now so prevalent that it is no longer even noticed.''}

It is the feeling of the authors of this book that the creative and inspirational efforts of the whole world may be needed to heal these deep wounds. It is possible that by connecting creatives with very different global perspectives, directly into `Western' production pipelines, that we will be able to see the shape of this potential.
\subsection{ML actors and blockchain based bots}
Stablity AI is an open source imitative to bring ML/AL capabilities to the world. This is a hugely exciting emergent area and much more will be developed here.
\subsection{AI economic actors in mixed reality}
AI actors can now be trusted visually \cite{nightingale2022ai}. We have some thinking on this which links the external web to our proposed metaverse. There is work in the community working on economically empowered bots which leverage Nostr and RGB to perform functions within our metaverse, and outside in the WWW, as well as interacting economically through trusted cryptography with other bots, anywhere, and human participants, anywhere. This is incredibly powerful and is assured by the Bitcoin security model. Imagine being able to interact with a bot flower seller representing all the real world florists it had found. In the metaverse you could handle the flowers and take advice and guidance from the bot agent, then it would be able to take your money to buy you flowers to send to a real world address, and later find you to tell you when it's delivered. These possibilities are endless. The AI chat element, the AI translation of images on websites to 3D assets in the Metaverse are difficult but possible challenges, but the secure movement of money from the local context in the metaverse to the real world is within reach using these bots, and they are completely autonomous and distributed.
\section{Our socialisation best practice}
\subsubsection{Identity}
We will base our identity and object management on Nostr public/private key pairs. The public key of these enable lightning based exchange of value globally. %Additional work will be needed to allow objects generated through RGB to be passed between federated worlds.
we plan to operate Nostr in multiple modes. Linking flossverse ``rooms'' will be a Nostr bot to bot system within the private relay mesh. This can also synchronise large amounts of data by leveraging torrents \href{https://iris.to/#/settings}{negotiated by Nostr}. Human to human text chat across and within instances is two 'types' kind of private nostr tag within the private relays mesh. External connectivity to web and nostr apps is just the public relay tags outbound. We don't need to store data external to the flossverse system, though access is obviously possible through the same torrent network.
\subsubsection{Webs of trust}
Webs of trust will be built within worlds using economically costly (but private) social rating systems, between any actor, human or AI. It should be too costly to attack an individual aggressively. This implies an increased weighting for scores issued in short time periods. Poorly behaving AI's will eventually be excluded through lack of funds.
\subsubsection{Integration of 'good' actor AI entities}
Gratitude practice should be encouraged between AI actors to foster trust and wellbeing in human observers. ``It's nice to be nice'' should be incentivised between all parties''. This could include tipping and trust nudging through the social rating system. Great AI behaviour would result in economically powerful entities.
\subsection{Emulation of important social cues}
\href{https://www.cleverclassroomsdesign.co.uk/general-5}{Classroom layout}
\subsubsection{Behaviour incentives, arbitration, and penalties}
Collapses of trust and abuse will trigger flags from ML based oversight, which will create situational records and payloads of involved parties to unlock with their nostr private keys. ML red flagged actors will be finacially penalised but have access to human arbitration using their copy of the data blob. Nothing will be stored except by the end users.
\subsection{Federations of webs of trust and economics}
Nostr is developing fast enough to provide global bridges between metaverse instances.
\section{Security evaluation}
As part of developing our stack we will penetration test the deployment as detailed using \href{https://hexway.io/}{Hexway}

\input{10_otherusecases}
\section{notes for later}
Notes on build-out
The world database in the shared rooms in the metaverse is the global object master,  educational materials, videos,  audio content and branded objects are fungible tokens authentically proved by rgb client side validation between parties,  only validated ones will be persisted in shared rooms like conferences and classes according to the room schema. That allows educators to monetise their content.  That can work on lightning.  NFT objects between parties like content crafted by participants (coursework, homework) are not on lightning and will attract main chain fees but are rarer. User authentication and communication will be through nostr.

\begin{figure*}[ht]\centering % Using \begin{figure*} makes the figure take up the entire width of the page
	\includegraphics[width=\linewidth]{globalclassroom}
	\caption{Functional elements for infrastructure.}
	\label{fig:globalclassroom}
\end{figure*}

\begin{figure*}[ht]\centering % Using \begin{figure*} makes the figure take up the entire width of the page
	\includegraphics[width=\linewidth]{systemc4}
	\caption{Client server C4 diagrams.}
	\label{fig:globalclassroom}
\end{figure*}
