Nonfungible tokens are a whole `class' of digital token, separate and distinct from everything discussed to this point. They are generally \href{https://www.signaturelitigation.com/nfts-recognised-as-property-lavinia-deborah-osbourne-v-1-persons-unknown-2-ozone-networks-inc-trading-as-opensea/}{recognised in law} as property in their own right \cite{moringiello2021property, fairfield2021tokenized}. In the Initial Coin Offering (ICO) and project tokens detailed earlier, and limiting this description to the Ethereum network for now, a project launching an ERC-20 token commits contract code to the blockchain, and this contract then mediates the issuance and management of millions or billions of tokens associated with that project, and it's use case. \href{https://ethereum.org/en/developers/docs/standards/tokens/erc-20/}{ERC-20} is a \href{https://en.wikipedia.org/wiki/Fungibility}{fungible} token issuance. Each of the projects' tokens is interchangeable with any other token. They're all the same from the point of view of the user.\par
Rather than the ERC-20 contract type used for fungible token issuance NTFs predominantly use ERC-721 protocol on Ethereum (just different instructions). It's the case that most NFTs in the 2021/2 hype bubble are algorithmically generated sets of themed art (so called PFP-NFT). Tens of thousands of distinct tokens are `minted', each one being a complex transaction commitment to the Ethereum blockchain, along with it's associated gas fee. These minting events were much hyped social occasions (before the \href{https://www.theguardian.com/technology/2022/jul/02/nft-sales-hit-12-month-low-after-cryptocurrency-crash?}{2022 market crash}), and happened very quickly, with users clamouring to create art with randomly allocated features from the art schema associated with the project. Lucky winners could find themselves with an NFT art piece with more than an average number of `rare' features. If the overall mint becomes more popular, then the secondary market for all of those mints goes up, and because of the liquidity premium they can go up a lot. The perceived rarer mints go up a lot more. This whole process is \href{https://memoakten.medium.com/the-unreasonable-ecological-cost-of-cryptoart-2221d3eb2053}{very energy intensive} on the chain, and the vast majority of these project simply \href{https://www.turing.ac.uk/blog/non-fungible-tokens-can-we-predict-price-theyll-sell}{trend to zero value}. In response to this appalling cost benefit analysis the Ethereum foundation have proposed \href{https://eips.ethereum.org/EIPS/eip-2309}{EIP-2309} to make minting NFTs more efficient. They say ``This standard lets you mint as many as you like in one transaction!''\par
The Ethereum foundation give their somewhat constrained view of \href{https://ethereum.org/en/nft/}{NFTs on their website} and it's a useful primer. On that page they detail some of the use cases, as listed below, with a critique added:
\begin{itemize}
\item Digital content; this is the dominant use case right now. Much more on this later.
\item Gaming items; again more on this later, it's an obvious enough use case but \href{https://climatereplay.org/nfts/nft-digital-ownership-pledge/}{complex politics} in the intersection of games and crypto have stalled the adoption curve.
\item Domain names; this is just starting to reach for applications now, why not a database with the ISP/host?
\item Physical items; seemed like a clear over-reach as transfer of the NFT does not imply transfer of the object, but this is emerging as the growth use case.
\item Investments and collateral; while this was an emergent option in the space, it's likely been a bubble, as owners of the tokens cast around for additional liquidity, and loan businesses chased yield with higher risk. The \href{https://newsletter.banklesshq.com/p/three-arrows-capital-grayscale-maker-lido}{recent implosion} of lenders and funds in the crypto space was partly a function of supposedly world class risk managers accepting jpegs as collateral.
\end{itemize}
Moving away from Ethereum, NFTs can be minted on most of the other level one chains. Solana is a great newcomer example. Sol is a terrible chain with regards to decentralisation, but thanks to that it's far cheaper and faster to mint NFTs on it, and it was becoming a \href{https://markets.businessinsider.com/news/currencies/ethereum-eth-killers-nfts-defi-solana-cardano-wax-crypto-investing-2022-1}{troubling competitor} for Eth before the FTX ponzi scheme collapse destroyed it's market value (Figure \ref{fig:solnfts}).\par 
\begin{figure*}[ht]\centering % Using \begin{figure*} makes the figure take up the entire width of the page
	\includegraphics[width=0.7\linewidth]{solnfts}
	\caption{Solana NFT markets are enjoying growth compared to Opensea on Ethereum, even in the downturn.}
	\label{fig:solnfts}
\end{figure*}
The same might be true for Cardano's ADA, though ADA is struggling to hold onto it's market position despite some technical advances. It's worth reiterating here that the nature of these digital tools likely makes for a `winner take all' market dynamic over time. With fees being central to this generative NFT use case it's possible to see that highly centralised, fast, and cheap chains will capture and eventually dominate the space. Remember that this likely (game theoretic) outcome might as well be a database running without the stark inefficiencies of blockchain. The whole NFT space is a gamble on consumer enthusiasm for spending money continuing to outpace logic.\par 
Astonishingly, according to a JPMorgan insider market report (\href{https://www.coindesk.com/podcasts/the-breakdown-with-nlw/jpmorgan-bitcoin-shows-some-merit-as-a-store-of-value/}{reported on in a podcast}), only around 2 million people have ever actually interacted with NFTs. One analysis suggests that a single entity accounts for 3 of the top 4 holders, having made 32,000 ETH from the NFT boom. This suggests heavy market manipulation and is far from the egalitarian landscape claimed in the hype. Tellingly it's thought around \href{https://uk.finance.yahoo.com/news/three-arrows-wanted-100m-nft-161811450.html}{10\% of the trading volume} on market leading platform `Super Rare' was by the now bankrupt venture capital firm `Three Arrows'. \par
With that said NFTs have clearly allowed \href{https://en.wikipedia.org/wiki/List_of_most_expensive_non-fungible_tokens}{digital and new media artists} to connect with audiences without gatekeepers. Established mediators and curators of art have been caught totally wrongfooted, and NFTs seem to give a way for them to be cut out completely. There are suggestions of applications beyond this initial digital art scope. This is a compounding, and disrupting paradigm change.\par
%Users of NFT markets have \href{https://blog.chainalysis.com/reports/nft-market-report-preview-2021/}{injected around \$30 billion into the tokens during 2021}. 
\section{Key use cases}
\input{07_NFTUmran}
Traditional gamers have pushed back on the seemingly useful idea of integrating NTFs with traditional games. This may be in part because Ethereum mining has kept graphics card prices high for a decade.

\href{https://www.prnewswire.com/news-releases/hbar-foundation-and-ubisoft-partner-to-support-growth-of-gaming-on-hedera-network-301474971.html}{HBAR partnerships}\par
\href{https://finance.yahoo.com/news/epic-games-vp-people-have-lost-interest-in-the-metaverse-200725562.html}{Critique from Marc Petit of Epic and Unreal}.\par
The \href{https://twitter.com/justinkan/status/1491270239967154178}{following text} is from Justin Kan, co-founder of twitch: \textit{``NFTs are a better business model for games. Many gamers seem to be raging hard against game studios selling NFTs. But NFTs are also better for players. Here’s why I think blockchain games will be the predominant business model in gaming in ten years. NFTs are a better business model for funding games . Example: recently I invested in a new web3 game SynCityHQ. They are building a mafia metaverse and raised \$3M in their initial NFT drop.\\ NFTs give studios access to a new capital market for raising capital from the crowd.NFTs can be a better ongoing model for games. Web3 games will open economies, and by building the games on open and programmable assets (tokens + NFTs) they will create far more economic value than they could from any one game. Imagine Fortnite, but other developers can build experiences on top of the V-Bucks and skins. Epic would get a royalty every time any transaction happens. As big as Fortnite is today, Open Fortnite could be much bigger, because it will be a true platform. NFTs are better for gamers Allowing gamers to have ownership of the assets they buy and earn in game allows them to participate in the potential growth of a game. It lets gamers preserve some economic value when they switch to playing something new. But what about the criticisms of NFTs?\\
Here are my thoughts on the common FUDs: "It’s just a money grab on the part of the studios!"\\
Game studios already switched over to the model of selling in-game items, cosmetics, etc to players long ago. But currently the digital stuff players are buying isn’t re-sellable. NFT ownership is strictly better for players. "The games aren’t real games." This reminds me of the criticism of free-to-play in 2008, when the games were Mafia Wars / FarmVille. We haven’t had time for great developers to create incredible experiences yet. Everyone investing in games knows there are great teams building. "Game NFTs aren’t really decentralized because they rely on models / assets inside centralized game clients."
Crypto is as much a movement as it is a technology. Putting items on a blockchain is what gives people trust that they have participatory ownership...which make people willing to buy in to the game. These assets are “backed” by blockchain.
The fact that these item collections are NFTs will make other people willing to build on top of them. "NFTs are bad for the environment." Solana and L2s solve this. NFT games are better for players and for game developers. Like the free-to-play revolution changed gaming, so will blockchain. The games of the future will be fully robust, with open and programmable economies.}''
\section{Broader and metaverse uses}
So far according to a16z NFTs break down into:
\begin{itemize}
\item Profile pictures: These were discussed at the start of the chapter and have felt ubiquitous on Twitter over the last couple of years. The major projects will likely hold value, but the hype cycle will likely lead to all profile NFTs going in and out of fashion. There's potentially a fresh wave of this same kind of low key identity hype possible in the metaverse, and indeed the two plausible both intersect and converge.
\item Art and Music: Art has also been discussed above. Peter Thiel, the billionaire venture capitalist who founded PayPal has invested in expanded NTF use cases. The first is `Royal' which is experimentally \href{https://royal.io/}{selling limited NFT tokens} which contractually entitle the holder to a portion of music artist royalties. Spotify are experimenting with music NFTs (and of course in the metaverse). This is an early adopter area, and again likely converges with our planned uses cases as more complex tooling appears. For instance Tim Exile of \href{https://endlesss.fm/}{Endless.fm} talks about digital assets extending to the building blocks of co-created music, and wished to build a music creator economy which distributes value to creators at the instant of the final value transaction with the consumer.
\item Gaming: As discussed there's pushback from the gaming community, but huge investment from the likes of Lego, Blizzard, Epic, Ubisoft etc.
\item Gig tickets: Not only the straightforward use of \href{https://news.yahoo.com/psg-sells-us-220-000-030927515.html}{transferable tickets for events} as NFTs on a blockchain (which is impossible due to the cost right now) but also onward monetisation of ticket stubs as memorabilia. The NBA is \href{https://deadspin.com/investing-in-nft-ticket-stubs-is-likely-one-of-the-nba-1848991991}{already looking at this}.\\
\textit{``The team sells the ticket for face value many many years ago, but when that stub is being sold now for much more many times over, the team gets none of that money,'' York explained. ``But with an NFT stub that changes. Let’s say a new rookie enters the NBA next season and he turns out to be the next LeBron James. That ticket stub from his first game, as an NFT, the team can put a commission on it — 20 percent or however much, the NBA decides that. In 10 years when it’s worth a lot of money, I or whoever owns that NFT, can sell it for say \$100,000. The NBA can still collect 20 percent of that sale, because it’s all on a smart contract.''}\par
It seems so obvious that this will extend to the virtual events space in the metaverse.
\item Utility: These are broadly `membership' style tokens, and this seems like a sensible fit. Peter Thiel (again) for instance launches a \href{https://www.ztonft.com/}{political funding NFT} from Blake Masters to support his senate ambitions. To be clear, Thiel is a fundamentalist libertarian, and at the very least \href{https://gizmodo.com/peter-thiel-bitcoin-talk-miami-2022-1848764790}{highly eccentric}. This is not necessarily a positive for the technology.
\item Virtual worlds are a huge application for NFTs, and this seems like it would be a natural fit for out metaverse application. In reality the \$2B of sold so far is mostly `allocations' in nascent ecosystems, being sold as highly speculative assets, without even a metaverse to use. The majority of that amount is the hyped `Otherland' plots sold under the Bored Apes brand.
\item ``Full stack'' luxury brands. \href{https://medium.com/@nic__carter/redeem-and-retain-nfts-are-the-future-of-luxury-goods-760f00dbce23}{Nic Carter describes} a mating of physical and virtual luxury goods. His is a useful article on the future direction, and he has also \href{https://medium.com/@nic__carter/why-nfts-are-hard-to-explain-48f0ab0a35bf}{provided a primer on NFTs}. There are many such examples already, such as \href{https://nft.tiffany.com/faq/}{Tiffanys `NFTiff' - cryptopunks} collaboration which will automatically generate royalties for Tiffanys and parent company Louis Vitton in perpetuity. Such products prove provenance, create new aftermarket opportunities, and unlock metaverse applications.
\end{itemize}
It is completely reasonable to assert that these use cases could be accomplished without the use of NFT technology, and is part of the hype bubble.\par
Twitter user Cantino.Eth offers an exhaustive roundup of what they think future uses might be. It's a \href{https://twitter.com/chriscantino/status/1542930648750608387}{thread full of industry insider jargon} but it's indicative of a shift in focus from speculation to `building' as the market conditions change. Some of the more interesting (less arcane) use cases identified in the thread are summarised very briefly below, again with comments as to how this might pertain to our metaverse applications.
\begin{itemize}
\item Hobby tokens, demonstrating interest in an activity. This is potentially a metaverse adaptation of badges on a blazer in the real world, and might serve to drive communities in a metaverse. The same is true for activism and political alighnment. It's a great idea and worth developing.
\item Professional Networks and qualification badges, like a LinkedIn qualification panel, but in the metaverse. A cisco NFT in the metaverse for a CCNA qualification makes intuitive sense. 
\item Badges to indicate membership of distributed projects within a metaverse. This allows users to identify avatars with shared goals in the metaverse.
\item Retail incentives, like brand loyalty stamps or rewards for participation in marketing, or early access programmes. This is a true in a metaverse marketplace as it is in a real world coffee shop.
\item Multiplayer communities with incentives to hit collective milestones. ``Collecting as a team sport''. This again seems like a great and intuitive opportunity, but is perhaps less suitable for our more business focussed space.
User content submission and automatic monetisation when reused by brands, bonded to an NFT contract.
\item Customer Cohort NFTs: early adopters of successful brands would be able to prove the provenance of their enthusiasm for a new product, and this might unlock brand loyalty bonuses. It seems this wouldn't be a transferable NFT, and is more like the ``soulbound'' idea advanced by Meta.
\item Education and Customer Support, think an NFT of a great score on reddit community support forums. A trusted community member badge, but visible in the metaverse. This is somewhat like the web of trust model advanced earlier in the book.
\item NFTs as contracts is far more likely in the metaverse than it has proved to be in real life. This is how `digital land' and objects will be transferred anyway, but with the addition of contractual conditionals with external inputs more subtle products may appear.
\end{itemize}
%Samsung for instance have announced that their TVs will support not only \href{https://news.samsung.com/us/samsung-2022-micro-led-neo-qled-lifestyle-tvs-personalization-options-ces-2022/}{display of NFTs} with artist defined settings in the metadata, but also an integrated marketplace for browsing and purchasing.\par

\section{Objects in our metaverse}
There has been a recent shift away from the `toxic' moniker of NFT and toward `Digital objects', and seem to be judged crucial to metaverse applications. The success of avatar \href{https://medium.com/coinmonks/reddit-nft-success-ca2685163576}{`collectibles' markets} in the Reddit ecosystem, and Meta (ex Facebook) similarly divesting themselves of the NFT term seem to suggest a pivot point in the industry. Meta are encouraging adoption through zero fee incentives but are likely hanging their monetisation of their whole rebrand on taking a huge cut from NFT content creators on their platform. Crucially for the whole concept of NFTs in crypto it looks like they will custody the digital objects within their databases, and allow them to be both bought and sold through interactions with `normal' Fiat money. This completely breaks the model of what an NFT represents, and may in time dilute the technology to the point of being completely meaningless.\par
We have a path to assets and NFTs within the layer 3 elements of our choice (RGB \& Pear Credits), but they're not yet fit for purpose. There are compromise options already available, as below. 
\subsection{Liquid tokens}
We have seen that Liquid from Blockstream is a comparatively mature and battle tested sidechain framework, based upon Bitcoin. It is possible to issue tokens on Liquid, and these have their own hardware wallet available. This makes the technology a strong contender for our uses.
\subsection{Sovryn and RSK}
It's slightly unclear when RSK will support assets at this level. This needs to be revisited.
%\subsubsection{Optimistic rollups}
%\lipsum[50]
%\subsubsection{Zero Knowledge rollups}
%\lipsum[50]
\subsection{Stacks and STX}
There's another possible option is Stacks, without the network effect of Ethereum, but closer to the other design choices made so far. ``Stacks is an open-source network of decentralized apps and smart contracts built on Bitcoin.''\\ 
This novel approach saw the launch of a layer 1 blockchain token called STX, which is used in a similar way to gas in Ethereum. but claims settlement on the Bitcoin network. This is achieved through a novel bridging approach which they call Proof of Transfer (PoX).\\
Stacks users say this hybrid approach is a pragmatic solution which enables dApps, smart contracts, DeFi, NFTs etc without compromising security. In practice the speculative component of the STX tokens which underpin these operations clouds the issue somewhat. It is a potentially useful middle ground solution with a great deal of developer attention.
\subsection{Ethereum}
While it's been discounted elsewhere it's hard to ignore the network effect of Eth NFTs. If the aspiration is to attract the bulk of the `legacy' creator/consumer markets then it will be necessary to support integration of Metamask into any FOSS stack. This isn't a huge technical challenge, nor is it particularly of interest to our use cases at this stage, but it remains a possibility. The main problems remain the slow speed and high expense of the system.
\subsection{Solana}
Solana is both cheap and fast, because it's very highly centralised. It seems unlikely that it's worth this level of compromise. It has also become embroiled with the fallout from the enormous FTX exchange fraud, threatening the existence of the assets (NFTs) issued and stored upon it.
\subsection{Satoshi Ordinals}
Satoshi ordinals \href{https://github.com/casey/ord}{allow tracking of Sats across transactions}, enabling NFT like assignment tracking. This is a hugely exciting development but extremely early.
\subsection{Peerswap}
It may be possible to use ``Peerswap'' to execute rebalancing and submarine swaps into and out of Liquid assets on the sidechain in a single tx. This is anunder explored area at this time.
\subsection{FROST on Bitcoin}
It \textbf{might} be possible to transfer ownership of a UTXO on the Bitcoin base chain using FROST \cite{komlo2020frost}. In this Schnorr \& Taproot based threshold signature system it's possible to \href{https://btctranscripts.com/sydney-bitcoin-meetup/2022-03-29-socratic-seminar/}{add and remove signatories} and thresholds of signing without touching the UTXO itself. In principle (though not yet in practice) this might allow transfer of UTXO ownership. 
\subsection{Spacechains}
It feels like spacechains are almost ready, so this is worth keeping an eye on. It's the `cleanest' way to issue assets using Bitcoin because there's no additional speculative chain. As briefly explained in the earlier section Bitcoin is destroyed to create a new chain which then inherits the security of Bitcoin through onward mining. This new asset or chain is able to accrue value and trade independently based purely on it's value to the buyer, not as a function of a wider speculative bubble attached to a token with multiple use cases.
\subsection{Pear credit}
The outstanding contender at this stage is Pear Credit from Hypercore. This section needs a full explanation later. For now a \href{https://medium.com/@observer1/tether-announced-the-launch-of-pear-credit-8d4f66ccd97b}{blog post on the subject} will have to do.
