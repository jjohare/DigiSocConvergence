\section{Government encroachment into digital society}
Much of the following text is paraphrased from the work of Guy Turner of `The Coin Bureau' and needs more work because of it's critical importance to the book.\par 
Distributed internet protocols are important in the context of government overreach into digital society and people's private lives because they provide a level of decentralization and resilience that can help protect against censorship and surveillance.\par
For example, if a government were to attempt to censor or block access to a centralized internet service, it could potentially do so with relative ease. However, if that same service were distributed across a network of nodes, it would be much more difficult for the government to effectively censor or block access to it.\par
Another advantage of distributed protocols is that they are typically more resilient to attacks or failures. If one node in the network goes offline or is compromised, the others can continue to operate, ensuring that the service remains available. This can be especially important in situations where the internet is being used for critical communication, such as during a natural disaster or political crisis.\par
In addition to their benefits for censorship resistance and resilience, distributed protocols can also help protect people's privacy. Because they do not rely on centralized servers or infrastructure, they can be more difficult for governments or other entities to monitor or track. This can be especially important in countries where government surveillance is prevalent or where individuals may be at risk of persecution for their online activities.\par 
There are a number of distributed protocols that have been developed specifically to address issues of censorship and privacy, and these will be covered in more detail later.\par
It is important to note that distributed protocols are not a silver bullet for censorship or privacy concerns. They can be vulnerable to certain types of attacks, such as those that target the nodes of the network, and they may not always be practical for certain types of applications. However, they do provide an important tool for those seeking to protect their freedom of expression and privacy online. They offer a valuable tool for those seeking to protect their freedom of expression and privacy online, and they will likely continue to play a critical role in the future of the internet.\par
In recent years, several countries have proposed or passed bills that would result in unprecedented levels of online censorship. One such example is Canada's Bill C-11, also known as the Online Streaming Act. This bill was first proposed in November 2020 as Bill C-10, but failed to pass due to its controversial provisions. It was reintroduced in February 2021 as Bill C-11 and was approved by the Canadian House of Commons, the first step in the process of becoming law. If passed, the bill would give the Canadian Radio, Television and Telecommunications Commission (CRTC) the power to decide what content Canadians can view on YouTube and other social media platforms. The CRTC would also have the power to dictate what content creators can produce, with a focus on promoting "Canadian content." Additionally, the bill would require certain broadcasters to contribute to the Canada Media Fund, which is used to fund mainstream media in Canada. The bill is currently being considered by the Canadian Senate, which will vote on it in February. If passed, it will then be debated by the Canadian Parliament. Tech companies such as YouTube have reportedly failed to convince the Senate to exclude user-generated content from the bill, indicating a high likelihood of it becoming law. The potential impact on the internet and free expression in Canada is significant, as the bill would give the government significant control over online content and restrict the ability of individuals to share their views and perspectives.\par
The European Union (EU) has separated its online censorship efforts into two separate bills: the Digital Markets Act and the Digital Services Act. These bills were introduced in December 2020 and are part of the EU's Digital Services package, which aims to be completed by 2030. The Digital Services package is the second phase of the EU's digital agenda, which is being enforced through regulation in the public sector and through ESG investing in the private sector. Both the Digital Markets Act and the Digital Services Act were passed in spring 2022 and went into force in autumn 2022, but will not be enforced until later this year and early next year, depending on the size of the relevant entity. The Digital Markets Act aims to increase the EU's competitiveness in the tech space by imposing massive fines on "gatekeepers," or companies that maintain monopolies by giving preference to their own products and services. This could open the door to innovation in cryptocurrency in the EU, but also requires gatekeepers to provide detailed data about the individuals and institutions using their products and services to the EU. The Digital Services Act, on the other hand, aims to regulate the content that is available online, including user-generated content. It does this by requiring companies to remove illegal content within one hour of it being reported and by imposing fines for non-compliance. The act also requires companies to implement measures to protect users from illegal content and from "other forms of harm," which is defined broadly and could include a wide range of content. The EU is also in the process of passing the Artificial Intelligence Regulation Act, which will be discussed later this year and is reportedly the first of its kind. All five bills in the EU's Digital Services package are regulations, meaning they will override the national laws of EU countries. The potential impact on the internet and free expression in the EU is significant, as the Digital Services Act would give the government significant control over online content and restrict the ability of individuals to share their views and perspectives.\par
In the United States, two significant documents related to online censorship are the Kids Online Safety Act and the Supreme Court case Gonzalez v. Google. The Kids Online Safety Act was introduced in February 2021 and is expected to pass later this year due to bipartisan support. The act requires online services to collect Know Your Customer (KYC) information to ensure that they are not showing harmful content to minors. It also gives the Federal Trade Commission (FTC) the power to decide when children have been made unsafe online and allows parents to sue tech companies if their children have been harmed online. The act has received criticism from both sides of the political spectrum and entities outside of Congress, as it is seen as giving too much power to the government to regulate online content and could lead to increased censorship by tech companies.\par
The Supreme Court case Gonzalez v. Google involves the question of whether Google's algorithmic recommendations supported terrorism and contributed to the 2015 terrorist attacks in Paris. The case has been picked up by the Supreme Court after being passed up by various courts of appeal. It is being heard alongside another case, Twitter v. Tumne, involving the role of Twitter's algorithms in a terrorist attack in Istanbul. There are two potential outcomes for the case. If the Supreme Court sides with Gonzalez, it could increase the liability of social media companies under Section 230 of the Communications Decency Act, which allows them to moderate content to a limited extent without violating the First Amendment. Alternatively, the Supreme Court could declare Section 230 unconstitutional, which would make online censorship illegal but also hinder the use of algorithms on the internet. The ideal outcome, in theory, would be for the Supreme Court to side with Google and for Congress to change Section 230. However, giving Congress the power to change the law could lead to increased censorship and the potential for abuse of power.\par
Our research focuses on business to business use cases for distributed technologies, and will provide mechanisms for verifying who is communicating with whom, to avoid falling foul of these swinging global infringements on privacy.