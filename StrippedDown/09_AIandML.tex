\section{The Cambrian explosion of ML/AI}

This section is full of rough edges and some repetition; Working on it!

\subsection{Overview}
Though the history of this field reaches back to the 1940's with McCulloch et al. exploration of the possible mathematical underpinnings of human brain neurons \cite{mcculloch1943logical}. During the writing of this book we have seen an inflection point in machine learning, to the point where the term ``artificial intelligence'' is feeling intuitively and subjectively real for the first time.  To be clear AI is still a pretty meaningless term. `Intelligence' is one of those slippery words which is highly dependent on context. A satnav system running on a phone can make an intelligent choice about a route by synthesising data and presenting comprehensible results, but it seems absurd to ascribe an intelligence to it. It's possible that there's some kind of ``spoooky'' quantum activity in play in a conscious human brain, as described in mind bending mathematical depth by Penrose in 1989 \cite{penrose1990emperor}. It's something of an unknown unknown \cite{kerskens2022experimental}, and that we'll never get to what's called `strong' or `general' AI \cite{larson2021myth, searle1980minds}, reserved by some scientists for ``true consciousness'', whatever that means \cite{butlin2023consciousness}. With that said we may be approaching the threshold of the `Turing Test` \cite{sep-turing-test}, initially posited by Alan Turing in 1950 \cite{turing1950computing}, and the goalposts have begun to move in response to claims that there have been successful examples \cite{warwick2016can, french2012moving, french2000turing, searle2009turing}. It feels that in this moment it is appropriate to open with a risks section, and work backwards. This is grounded in the hypothesis that there is no agreed end goal here (as we saw with the Bitcoin/Crypto chapter).\par
To set the tone here let's have OpenAI's ChatGPT give us a definition:
it{Intelligence is the ability to acquire and apply knowledge and skills in order to solve problems and adapt to new situations. It can involve a range of cognitive abilities, such as perception, learning, memory, reasoning, and decision-making. Intelligence is a complex and multifaceted concept that has been studied by psychologists, philosophers, and scientists for centuries.}\par
The Oxford English Dictionary defines Artificial intelligence as ``The capacity of computers or other machines to exhibit or simulate intelligent behaviour''. This is very murky territory. The boundary line between very capable trained systems and something that it{feels} like intelligence is obviously a subjective one, and different for each person and context, (Figure \ref{fig:aiVenn}.\par

\begin{figure}[ht]\centering 	\includegraphics{ai}
	\caption{The terminology in the field is both somewhat blurred and highly `nested'.}
	\label{fig:aiVenn}
\end{figure}

We will use AI and ML interchangeably in this text, but is so doing we hope to draw attention to the moment we find ourselves in. It feels like there is an inflection point in human history happening right now, though to somewhat burst the bubble on this hyperbole it's worth reading the legendary Stephen Wolframs \href{https://writings.stephenwolfram.com/2023/02/what-is-chatgpt-doing-and-why-does-it-work/}{explanation of these current systems} as glorified autocompletes. \par
Irrespective of the gap between the perception and truth around these systems there is now a feedback loop where the data that these systems are trained on will be learning from both human \textbf{and} outputs from such systems. Todays young children will never know a world in which the information they encounter is verifiable as of purely human origin. The implications of this are unclear but exciting. In writing this book it became obvious to add this chapter in, and change the direction on the research and product development, because nothing in human history will remain untouched by this. As we will see `metaverse' is likely to change at an incredible rate as a function of some parts of this technology. %Sequoia Capital \href{https://www.sequoiacap.com/article/generative-ai-a-creative-new-world/}{maintain an overview} of the generative art landscape and update their findings regularly (Figure \ref{fig:sequoiacapLandscape}).

\begin{figure}[ht]\centering 	\includegraphics{sequoiacapLandscape.png}
	\caption{Major stands of generative AI and their associated models at the time of print.}
	\label{fig:sequoiacapLandscape}
\end{figure}

\ref{fig:llmlandscape}).
\begin{figure}[ht]\centering 	\includegraphics{llmlandscape}
	\caption{Major stands of large language models from Yang et al \cite{yang2023harnessing}}
	\label{fig:sequoiacapLandscape}
\end{figure}

%
\section{Overview of Large Language Models}
In the last year or two it's become obvious to most people that ``artificial intelligence'' is intuitively and subjectively real for the first time. Around half of adults in the west are now aware of ChatGPT. To be clear AI is still a pretty meaningless term. `Intelligence' is one of those slippery words which is highly dependent on context. A satnav system running on a phone can make an intelligent choice about a route by synthesising data and presenting comprehensible results, but it seems absurd to ascribe an intelligence to it. It's possible that there's some kind of ``spoooky'' quantum activity in play in a conscious human brain, as described in mind bending mathematical depth by Penrose in 1989 \cite{penrose1990emperor}. It's something of an unknown unknown \cite{kerskens2022experimental}, and that we'll never get to what's called `strong' or `general' AI \cite{larson2021myth, searle1980minds}, reserved by some scientists for ``true consciousness'', whatever that means. With that said we may be approaching the threshold of the `Turing Test` \cite{sep-turing-test}, initially posited by Alan Turing in 1950 \cite{turing1950computing}, and the goalposts have begun to move in response to claims that there have been successful examples \cite{warwick2016can, french2012moving, french2000turing, searle2009turing}. It feels that in this moment it is appropriate to open with an ethics section.\par
To set the tone here let's have OpenAI's ChatGPT give us a definition:
\begin{tcolorbox}[enhanced, frame style={fill=lightgray}, interior style={fill=lightgray}]Intelligence is the ability to acquire and apply knowledge and skills in order to solve problems and adapt to new situations. It can involve a range of cognitive abilities, such as perception, learning, memory, reasoning, and decision-making. Intelligence is a complex and multifaceted concept that has been studied by psychologists, philosophers, and scientists for centuries.
\end{tcolorbox}
The Oxford English Dictionary defines Artificial intelligence as 
\begin{tcolorbox}[enhanced, frame style={fill=lightgray}, interior style={fill=lightgray}]The capacity of computers or other machines to exhibit or simulate intelligent behaviour''.
\end{tcolorbox} 
This is very murky territory. The boundary line between very capable trained systems and something that \textit{feels} like intelligence is obviously a subjective one, and different for each person and context, (Figure \ref{fig:aiVenn}).\par

\begin{figure}[ht]\centering 	\includegraphics[width=0.8\textwidth]{ai}
	\caption{The terminology in the field is both somewhat blurred and highly `nested'.}
	\label{fig:aiVenn}
\end{figure}

I will use AI and ML interchangeably in this text, but in so doing I hope to draw attention to the moment we find ourselves in. It feels like there is an inflection point in human history happening right now, with AI as a cultural truism and the more honest `machine learning' relegated to history. \par
Irrespective of the gap between the perception and truth around these systems there is now a feedback loop where the data that these systems are trained on will be learning from both human \textbf{and} outputs from such systems, to the detriment of both models and humans\cite{shumailov2023curse}. Today's young children will never know a world in which the information they encounter is verifiable as of purely human origin. The implications of this are unclear but exciting. \par
In the field of natural language processing, there have been significant advancements in the use of language models for various tasks. These models are trained on large amounts of text data and are able to generate coherent sentences or paragraphs. They have been used for tasks such as language translation, text summarization, and sentiment analysis. GPT3 and GPT4 are the most widely known examples, and are right in the boundary space where ML begins to look like AI. To be clear such language models are many layers short of AI \cite{elkins2020can, marcus2020gpt}. We have by now all seen these work but here's a quick comparison demonstrating the rate of change. The following was fed into ChatGPT 3.5: ``write a poem in iambic pentameter about the difficulty humans have deciding if a thing is really intelligent.''
\newpage
\begin{tcolorbox}[enhanced, frame style={fill=lightgray}, interior style={fill=lightgray}]``Is it true intelligence, or just a guise?
We struggle to decide with troubled brow
For machines can mimic, with clever guise
The way we think and work, with nimble now
But do they truly understand and feel?
Or is it just a programmed façade?
We ponder on this question, with a zeal
For understanding if intelligence is a nod
To mere programming, or something more profound
A spark of consciousness, a self-aware mind
These questions plague us, yet we are not bound
By any one answer, we are free to find
Our own conclusion, as we ponder and muse
On the true nature of intelligence and its use''.
\end{tcolorbox}

This is the GPT4 version a month later, handling the LaTeX formatting itself:\\

\itshape{
Upon the realm of mind and cogitation, \\
A question looms with great anticipation. \\
Doth sentient life in circuitry exist, \\
Or is it but a dream, a wisp of mist? \\

In measured beats of thought, the heart doth race, \\
As human minds grapple with cyberspace. \\
When metal minds in cold precision speak, \\
Do we, in truth, behold intelligence unique? \\

From Turing's test, the doubts and queries sprout, \\
A test to quell the din of skeptic's shout. \\
Yet do these trials prove with certainty, \\
The depth of thought, the spark of liberty? \\

Alas, we seek a scale, a touchstone pure, \\
To judge the essence of the mind obscure. \\
But in that quest, we find ourselves confound, \\
For truth, like wind, doth ever shift around. \\

The line between machine and man doth blur, \\
As thoughts entwined, a tangled web they stir. \\
To tell apart the spirit from the gears, \\
We strain our eyes, confront our deepest fears. \\

In iambic verse, we search the bounds of thought, \\
To wrest the truth from shadows, ever sought. \\
But answers true remain elusive still, \\
The more we grasp, the more they slip our will. \\

For in this dance, intelligence doth sway, \\
And human hearts, beguiled, are led astray. \\
To judge the spark within the wires and code, \\
We must look inward, seek our own abode.}
\normalfont


\subsection{How they work}
For an excellent youtube video on the subject I would currently recommend \href{}{this presentation} from OpenAI founder Andrej Karpathy. Their system is the GPT3-4 series that everyone is familiar with, and the first few sections will concentrate on those options. If you know this stuff you can skip to section \ref{sec:llmoptions}. Alternatively if you want more then you can head to the appendix \ref{sec:aiterms} to see where I am up to in my learning, or else dive straight into the Little Book of Deep Learning \cite{fleuret2023little}. \par
All these systems run \textbf{purely} on numbers. The input data that they are trained on are compressed enormously into a thing called latent space \cite{DBLP:journals/corr/abs-2112-04895}, which is `loosely analogous' to a massive cloud of numbers. The text you type into interface is chopped up into ``tokens'', which is a leftover word from natural language processing research. Tokens are numbers that match words, or bits of words. This varies a lot depending on the techniques being used, but for our purposes we simply need to know that all the words you input, become somewhat \textit{more} number representations, which are then fired off into the latent space number cloud to find patterns that kinda match. This input is usually refered to as the `context window', or `token limit'. In GPT4 it's 8000 tokens (6000 words) and in GPT3.5 Turbo it's 16,000 tokens.  What comes back out of the cloud is statistically most likely sets of numbers which flow from the end of the enormous number input. Polymath and general all round genius Stephen Wolfram explains all this very well \href{https://writings.stephenwolfram.com/2023/02/what-is-chatgpt-doing-and-why-does-it-work/}{in a blog post}.\par
LLMs simply take whatever is in front of them and ``try to complete the document'' using their vast store of ``things that look like this document''. When you are chatting with them there's a bunch of contextual stuff pre-loaded onto the text you type. You don't see this input go in, but it sets up the broader document which the big autocomplete is tasked with finishing. This is ``very roughly'' of the context:
\begin{tcolorbox}[enhanced, frame style={fill=lightgray}, interior style={fill=lightgray}]
``Below is an instruction that describes a task, paired with an input that provides further context. Write a response that appropriately completes the request.

--- Instruction:
Instruction

--- Input:
Input

--- Response:''
\end{tcolorbox}
As the conversation with the chat interface continues the users' prior inputs are added into the context, and that goes in with each and every request. This builds up quickly to reinforce the current conversational flow. The model is inherently blind to the previous conversations, it's just being fed more and more of the previous chat behind whatever you just typed. This is where the systems currently break down. The input frame for ChatGPT is generally 8000 tokens. This allows a lot of chat (about 6000 words), but anyone who's used these systems will have seen it `forget' the earliest context as those tokens fall out of the back of the input frame.\par 
\subsection{Evolving Landscape of LLM Pipelines}

This section delves into the key pillars that form the backbone of deploying and managing these advanced systems.

\begin{comment}
flowchart TB
    A[Start: User Request] -->|Determine Jurisdiction| B[Jurisdiction Check]
    B -->|Data Input| C[Observability]
    C --> D[AI Gateway]
    D -->|Model Selection| E[Model Interaction]
    E --> F[Security and Compliance Check]
    F --> G[End: Deliver Response]

\end{comment}

\begin{figure}[H]
    \centering
    \includegraphics[width=0.2\textwidth]{llmflow}
    \caption{The flow of data through a modern LLM pipeline}
    \label{fig:llmflow}
\end{figure}



\subsubsection{Observability in AI Systems}
Observability plays a pivotal role in the effective management of AI and LLM pipelines. It involves a comprehensive monitoring mechanism that encompasses various aspects:
\begin{itemize}
    \item \textit{Performance Monitoring:} Tracking the efficiency and accuracy of AI models in real-time.
    \item \textit{System Health Checks:} Regular assessments to ensure the AI systems are functioning optimally.
    \item \textit{User Interaction Analysis:} Understanding how users interact with AI systems to enhance user experience.
\end{itemize}

\subsubsection{The AI Gateway: A Multiplex Interface}
The AI Gateway acts as a central hub, orchestrating the interaction between different AI models and the user interface. Key functionalities include:
\begin{itemize}
    \item \textit{Model Integration:} Seamlessly connecting various AI models to the system.
    \item \textit{Dynamic Swapping:} Allowing for the exchange of models without disrupting the system’s operations.
    \item \textit{Load Balancing:} Distributing requests effectively among different models to optimize performance.
\end{itemize}

\subsubsection{Prompt Management and Experimentation}
Effective communication with AI models is achieved through carefully crafted prompts. This area focuses on:
\begin{itemize}
    \item \textit{Prompt Library Development:} Creating a diverse range of prompts to elicit specific responses from AI models.
    \item \textit{A/B Testing:} Experimenting with different prompts to determine the most effective ones.
    \item \textit{User-Centric Design:} Tailoring prompts to align with user preferences and behaviors.
\end{itemize}

\subsubsection{Security and Compliance in AI Ecosystems}
Ensuring the security and regulatory compliance of AI systems is crucial. This involves:
\begin{itemize}
    \item \textit{Data Privacy Measures:} Implementing strategies to protect sensitive user data.
    \item \textit{Regulatory Compliance:} Adhering to global standards and regulations like GDPR (including segmentation of personal data and routing issues).
    \item \textit{Threat Mitigation:} Developing robust defences against cyber threats and vulnerabilities.
\end{itemize}

Each pillar represents an integral component in the architecture of modern AI and LLM pipelines, facilitating their evolution and adaptation in a fast-paced digital world.

\subsection{Ethics \& Impact}
AI ethics is now a hot topic even outside of the academic fields which have previously wrestled with these issues. These new systems are thought likely to be able to impact all human activity \cite{eloundou2023gpts}. \par
The Centre for Humane Technology \href{https://www.humanetech.com/key-issues}{highlight the negative assessment} of the risk of Large Language Models, stating that major advancements in AI are coming and society needs to address the issues before they become `entangled'. The misalignment problem with social media has not been fixed, and this is where Harris and Raskin made their name. Their focus is not on AGI apocalypse scenarios but rather on the potential consequences of rapid AI advancements. AI has evolved significantly since 2017. They somewhat hyperbolically call such models `golems', in reference to their emergent capabilities. They would like society to address the potential risks of LLMs before they become entangled in society and to better understand and manage their impact.\par
They highlight author \cite{harari2014sapiens} Harari's quote: \textit{``What nukes are to the physical world, AI is to the virtual and symbolic world.''} - and this quote feels very compelling when held up against our assertions about cryptographic end points and trust.\par
Rosenburg describes \href{https://bigthink.com/the-present/danger-conversational-ai/}{`The AI manipulation problem'} which refers to the emerging risk of real-time engagement between a user and an AI system that can impart targeted influence, sense the user's reactions, and adjust tactics to maximize persuasive impact. This can occur through natural spoken interactions with photorealistic virtual spokespeople that look, move, and express like real people. Conversational AI can push individual emotional buttons by adapting to personal data and analyzing real-time emotional reactions, becoming more skilled at playing each user over time. This presents a serious concern as it can be used to influence individuals and broad populations in ways that may be damaging to society \cite{Rosenberg2023}. \par
As AI models improve, they are generating their own data for reinforcement learning, with humans only needed to train the reward function. The advancements are not limited to algorithms and APIs; even hardware is advancing rapidly due to AI, as shown by Nvidia's research on using AI to improve chip design.\par
GPT-4 is breaking its \href{https://nanothoughts.substack.com/p/reflecting-on-reflexion}{own records} through self-reflection. This reflection paper has caught global attention and shows GPT-4's ability to improve upon its mistakes, and enhance its performance in a variety of tasks \cite{shinn2023reflexion}. The process of self-reflection has been observed in other research papers as well, demonstrating its effectiveness in improving model outputs.\par
The groundbreaking Huggingface GPT model can draw upon thousands of other AI models to perform tasks involving text, image, video, and question answering \cite{shen2023hugginggpt}. The Hugging GPT paper reveals a model that can connect numerous AI models for solving complex tasks, using language as an interface. It can perform tasks such as describing and counting objects in a picture, generating images and videos, deciphering invoices, and even describing them in perfectly simulated human voices.\par
The rapid advancements in AI are causing commercial pressure, pushing companies like Google to catch up with the likes of OpenAI. With self-improvement, tool use, hardware advances, and commercial pressure, the future of AI seems to be on a trajectory of rapid growth and development. The most recent comparator is the emergence of the iPhone and death of Adobe Flash \cite{horton2019death}, but this is orders of magnitude beyond that and likely most analogous to the industrial revolution \cite{trajtenberg2018ai}. \par 
\href{https://www.key4biz.it/wp-content/uploads/2023/03/Global-Economics-Analyst_-The-Potentially-Large-Effects-of-Artificial-Intelligence-on-Economic-Growth-Briggs_Kodnani.pdf}{A Goldman report} sees 25\% of roles basically wiped out. Even OpenAI themselves published similar findings \cite{eloundou2023gpts}. The systems can already pass almost all human examinations with ease, causing immediate existential harm to education assessment systems.\par 
The risks of generative AI are well covered by a \href{https://www.citizen.org/article/sorry-in-advance-generative-ai-artificial-intellligence-chatgpt-report/}{Public Citizen article}, and are simply listed below to get us started. It is a huge and well researched report and this list doesn't do it justice:
\begin{itemize}
\item A sharp increase in political disinformation;
\item Intensified, widespread consumer and financial fraud;
\item Intrusive privacy violations beyond those already normalized by Big Tech;
\item Harmful health impacts, from promotion of quack remedies to therapeutic malpractice;
\item Amplification of racist ideas and campaigns;
\item Destruction of livelihoods for creators;
\item Serious environmental harm stemming from generative A.I.’s intense energy usage;
\item The stripping of the information commons;
\item Subversion of the open internet;
\item Concentration of economic, political, and cultural power among the tiny number of giant companies with the resources to develop and deploy generative A.I. tools.
\end{itemize}

Research suggests that LLM outputs cannot be discriminated from human output in meaningful ways, casting shade across all assessed measures of humans in written form \cite{sadasivan2023can} from this point forward. Michal Zalewski \href{https://lcamtuf.substack.com/p/llms-a-bleak-future-ahead}{says}: \textit{``Instead of taking sides in that debate, I’d like to make a simpler prediction about LLMs as they operate today. I suspect that barring urgent intervention, within two decades, most of interactions on the internet will be fake. It might seem like an oddly specific claim, but there are powerful incentives to use LLMs to generate inauthentic content on an unprecedented scale — and there are no technical defenses in sight. Further, one of the most plausible beneficial uses of LLMs might have the side effect of discouraging the creation of new organic content on the internet.''}

Within generative machine learning there has been a raging debate between `some' of the artists whose original works were `scraped' into the \href{https://laion.ai/}{LAION} open dataset. Far more can and should be said on this, but I suspect Immersive has it's own advanced opinions so I won't attempt to cut across them.\par
AI safety and the challenges associated with managing an AI system's output are important considerations. Efforts have been made to ensure AI safety over the years, but the pace of change may be outstripping the ability of the industry and researcher to keep up. Finding the right balance between allowing users to express themselves and drawing the line on harmful or offensive content is a difficult task. Defining concepts like hate speech and harmful output, as well as aligning AI systems with human values and preferences, are complex challenges. Ideally, a democratic process would involve people from different perspectives coming together to decide on rules and boundaries for AI systems. However, implementing such a process is not straightforward, and developers must remain heavily involved in decision-making while still seeking input from a broader audience.\par
It is beyond the scope of this text to dig far into these issues, but they have serious implications for anyone working in the space. \par 
\subsubsection{AI's lying heart}
Large language model systems, at this time, are given to making things up. They `hallucinate' very cogently \cite{azamfirei2023large}, presenting completely erroneous assertions and data as facts, with \href{https://news.artnet.com/art-world/chatgpt-art-theory-hal-foster-2263711}{figures to back them up}. Perhaps counter intuitively this problem extends into pure mathematical realms, with seemingly simple arithmetic causing the systems problems. Both of these are the same problem, the models descend the most statically plausible path toward an output based pretty much on the last character generated along the vector into the n dimensional algebra space. This is being ameliorated through self checking \cite{manakul2023selfcheckgpt}, human reinforcement learning \cite{ouyang2022training}, and integration of specialised tools \href{https://writings.stephenwolfram.com/2023/01/wolframalpha-as-the-way-to-bring-computational-knowledge-superpowers-to-chatgpt/}{like Wolfram Alpha}, behind the models.\par 
Media outlets such as The Guardian are rushing to publish open guidelines on \href{https://www.theguardian.com/commentisfree/2023/apr/06/ai-chatgpt-guardian-technology-risks-fake-article?}{how they intend to deal} with the problem of GPT based article research, and it seems that \href{https://www.bbc.co.uk/news/technology-65202597}{legal challenges} will quickly mount up. With that said, the UK government is currently signalling a \href{https://www.gov.uk/government/publications/ai-regulation-a-pro-innovation-approach}{``pro-innovation''}, hands off approach to AI, and is \href{https://www.gov.uk/government/news/initial-100-million-for-expert-taskforce-to-help-uk-build-and-adopt-next-generation-of-safe-ai}{investing significantly} in supporting the technology.
\subsubsection{Alignment and take-off of AGI}
AI has the potential to significantly improve quality of life, but it is essential to ensure that it remains aligned with human values and does not harm or limit humanity. Alignment in language models trys to address this by incorporating strategies, which can be broadly categorized into the following areas:
\begin{itemize}
\item \textbf{Pre-training} In the initial stage, models are trained on a large dataset containing diverse text from the internet. This allows them to learn grammar, facts, reasoning abilities, and some level of world knowledge. However, they can also learn biases present in the data. Alignment efforts at this stage may involve dataset choices, monitoring, and some feedback to reduce reduce obvious biases.
\item \textbf{Fine-tuning} After pre-training, models are fine-tuned on a narrower, more specific dataset, often with human-generated examples and labels. During fine-tuning, alignment efforts can include generating clearer instructions for human reviewers, providing them with guidelines and feedback, and iteratively refining the process to improve the model's behaviour. Things like ranked lists of responses from the system would go in this stage. This is the ``human reinforcement learning'' which has so invigorated LLMs recently \cite{perez2022discovering}.
\item \textbf{Evaluation} Assessing the performance and safety of a language model is crucial for alignment. This can involve creating evaluation metrics, benchmarks, and tests to measure the model's progress in terms of safety, usefulness, and alignment with human values. Some of these are encoded inline with the model generation and are called `hyperparameters'. Two important hyperparameters that influence the generated text are frequency penalty and presence penalty. Frequency penalty affects the repetition of tokens in the generated text by modulating the probability distribution of tokens. A positive frequency penalty value discourages the model from choosing frequently occurring tokens, leading to a more diverse and sophisticated vocabulary. Presence penalty influences the generation of tokens that have already appeared in the generated text. A higher presence penalty value discourages the model from repeating tokens or phrases, ensuring a more cohesive and engaging output. On the other hand, a lower presence penalty value allows the model to generate text with repeated tokens or phrases, which can lead to redundancy and reduce the overall coherence of the text.\par 
So called use of `Red Teams' goes in this section, but much of this has recently been done live on alpha and beta public testing. 
\item \textbf{Safety mitigations} Addressing risks associated with harmful outputs, non-consensual content, and other safety concerns is an essential aspect of alignment. Techniques like more reinforcement learning from human feedback, rule-based filters, and output constraints can be employed to reduce harmful outputs and improve model safety. These feedback loops at the end of the process carry a disproportionate weight in the models, and it is at this stage that the most noticeable change is sometimes introduced. It is critical to get the maximum global public engagement during this final feedback into the system, from policy-makers and diverse human response feedback. It's not clear this is yet happening. Even Altman, the CEO of OpenAI admits that the models likely reflect the training choices of San Francisco `Tech Bros' in this moment. That is clearly not good enough.
\end{itemize}
Altman, says they have optimised their approach based on aiming for earlier AGI, but a slow AGI "takeoff", ie, the dfference between a gradual understandable creep toward a general intelligence, vs "AI FOOM" \cite{yudkowsky2008hanson}. They have a weighted risk matrix, but no clear ideal -yet- as to what the tools look like to keep the AGI "aligned" with humanity. Starting fast, but asserting slow and steady about the emergent behaviour part seems optimistic at best. Nonetheless this is likely the best/only approach, assuming it' possible. The only alternative is literally the ``Butlerian Jihad'' from Herberts' Sci-Fi DUNE.\cite{song2018preventing}.\par
There is a possibility that AI could become misaligned as it gains superintelligence, and it is essential to acknowledge and address this risk. 
AI technology, such as GPT-4, has shown exponential improvement, raising concerns about AI takeoff, which could happen rapidly within days. While GPT-4's performance has not been surprising to some, its impact highlights the need to increase technical alignment work and continually update the understanding of AI safety.\par
It's noteworthy that industry luminaries such as Yoshua Bengio, Turing Prize winner, Stuart Russell, director of the Center for Intelligent Systems, Elon Musk, and Steve Wozniak, Co-founder of Apple co-signed an \href{https://futureoflife.org/open-letter/pause-giant-ai-experiments/}{open letter} requesting a 6 month pause on the development of these systems. For what it's worth I have also signed. The narrative needs time to catch up. To be clear it won't happen, but the signalling here is part of building the awareness that is sorely lacking. 
\subsubsection{Industry capture, and failure to democratise}
Financial time article, integrate:
\begin{tcolorbox}[enhanced, frame style={fill=lightgray}, interior style={fill=lightgray}]``There's a colossal shift going on in artificial intelligence - but it's not the one some may think. While advanced language-generating systems and chatbots have dominated news headlines, private AI companies have quietly entrenched their power. Recent developments mean that a handful of individuals and corporations now control much of the resources and knowledge in the sector - and will ultimately shape its impact on our collective future. The phenomenon, which AI experts refer to as "industrial capture", was quantified in a paper published by researchers from the Massachusetts Institute of Technology in the journal Science earlier this month, calling on policymakers to pay closer attention. Its data is increasingly crucial. Generative AI - the technology underlying the likes of ChatGPT - is being embedded into software used by billions of people, such as Microsoft Office, Google Docs and Gmail. And businesses from law firms to the media and educational institutions are being upended by its introduction.
The MIT research found that almost 70 per cent of AI PhDs went to work for companies in 2020, compared to 21 per cent in 2004. Similarly, there was an eightfold increase in faculty being hired into AI companies since 2006, far faster than the overall increase in computer science research faculty. "Many of the researchers we spoke to had abandoned certain research trajectories because they feel they cannot compete with industry - they simply don't have the compute or the engineering talent," said Nur Ahmed, author of the Science paper. In particular, he said that academics were unable to build large language models like GPT-4, a type of AI software that generates plausible and detailed text by predicting the next word in a sentence with high accuracy. The technique requires enormous amounts of data and computing power that primarily only large technology companies like Google, Microsoft and Amazon have access to. Ahmed found that companies' share of the biggest AI models has gone from 11 per cent in 2010 to 96 per cent in 2021. A lack of access means researchers cannot replicate the models built in corporate labs, and can therefore neither probe nor audit them for potential harms and biases very easily. The paper's data also showed a significant disparity between public and private investment into Al technology. In 2021, non-defence US government agencies allocated \$1.5bn to AI. The European Commission planned to spend \$1.5bn. Meanwhile, the private sector invested more than \$340bn on Al in 2021.''
\end{tcolorbox}
\subsubsection{Cybersecurity implications}
It seems clear to all observers of the potential LLMs to be leveraged in supranational misinformation campaigns, zero-day exploits, and other geopolitical threats. It's appropriate to contextualise American export restrictions on powerful GPU and TPU hardware to China and Iran, and the implications of LLMs being "in the wild" and easily retrained for nefarious purposes. These powerful tools can be weaponised by malicious actors to create a new frontier of cybersecurity threats. 
\begin{itemize}
\item Supranational Misinformation Campaigns: 
LLMs can be easily retrained and used to generate convincing disinformation, thereby escalating the effectiveness of misinformation campaigns. As LLMs become more accessible, nation-states and non-state actors can utilise these technologies to create more targeted, sophisticated, and persuasive disinformation. This could lead to widespread confusion and mistrust, ultimately undermining the credibility of institutions and weakening social cohesion.
\item Zero-Day Exploits: 
Aggressive state actors can exploit LLMs to discover and leverage zero-day vulnerabilities. By training LLMs on massive amounts of data related to software development, vulnerabilities, and exploits, malicious actors can potentially identify previously unknown weaknesses in critical systems. These actors could then launch devastating cyberattacks, causing significant damage to their adversaries' infrastructure, economy, and security.
\item Autonomous Incursions:  Highly Compressed LLMs can potentially deploy autonomous incursion suites like Stuxnet, but with a greater degree of autonomy and less reliance on command and control. These advanced incursion suites might infiltrate targeted systems, adapt to their environment, and blend in with legitimate traffic, making them difficult to detect and mitigate. The integration of LLMs into these suites increases the potential for damage and disruption.
\end{itemize}
In response to these threats, the United States has imposed export restrictions on powerful GPU and TPU hardware to China. While these restrictions may slow down the development and deployment of LLMs for malicious purposes, they are unlikely to provide a long-term solution. As LLMs become more accessible and the technology proliferates, bad actors will trivially acquire the necessary hardware through alternative means, and the model weights themselves are already freely available. Existing defenses may not be adequate to counter the threats posed by LLMs, and the development of new cybersecurity measures requires time, resources, and expertise. In the absence of substantial increases in cybersecurity investment and research, it is likely that adversaries will continue to exploit LLMs to their advantage.\par
\href{https://securityintelligence.com/news/are-threat-actors-using-chatgpt-to-hack-your-network/}{According to Brad Hong}, Customer Success Manager with Horizon 3 AI, ``attackers are likely to adopt code generative AI as a means to address their limitations in scripting skills.'' Hong explains that code generative AI acts as a bridge, translating between languages that the attacker may not be proficient in and providing a means of generating base templates of relevant code. This reduces the skill requirements needed to launch a successful attack, as it serves as a tool in the attacker's arsenal alongside other tools.\par
In an email interview, Hong notes that the use of code generative AI, such as chat GPT, supercharges the attacker's ability to launch successful attacks. As explained by Patrick Harr, CEO at SlashNext, in an email comment, said \textit{``ChatGPT enables threat actors to modify their attacks in millions of different ways in minutes, and with automation, deliver these attacks quickly to improve compromise success.''}\par
The developers of chat GPT were aware that threat actors would attempt to weaponize the AI. As Harr explains, ``hackers are really good at weaponizing whatever technology is put in front of them.'' To mitigate this risk, the developers implemented preventive measures, such as flagging certain words and terms, like `ransom' or `ransomware.' For example, when researchers at Deep Instinct typed in the word `keylogger,' the chatbot responded with ``I'm sorry but it would not be appropriate or ethical for me to help you write a keylogger.''\par
However, these restrictions can be circumvented with a bit of clever rewording. As Jared, a Competitive Intelligence Analyst with Deep Instinct, explains, ``there's always a workaround.'' For example, instead of asking chat GPT to create ransomware code, an attacker could request a script that encrypts files and directories, drops a text file into the directory, and subdirectories.\par
The ability to weaponize chat GPT, or any other AI chatbot, is a game-changer for threat actors. With the ability to modify malicious code quickly and bypass cybersecurity defenses, these tools have the potential to greatly increase the effectiveness of attacks. As such, it is important for cybersecurity professionals to stay informed about the evolving threat landscape and the tools and techniques used by attackers \cite{brundage2018malicious}.\par 
At this early stage in the technology it is important that corporate and private users alike bear in mind that the LLMs are `leaky' and are using the data fed into them to train themselves. They are \href{https://help.openai.com/en/articles/6783457-chatgpt-general-faq}{explicit about this}. Anything that goes into chatGPT can resurface, as Samsung have found out \href{https://cybernews.com/news/chatgpt-samsung-data-leak/}{to their cost}.
\subsubsection{Use in war}
In the sphere of military, geopolitical, and civil defence the application of these tools is both shadowy and seemingly \href{https://www.vox.com/recode/23507236/inside-disruption-rebellion-defense-washington-connected-military-tech-startup}{somewhat incompetent}. It is an especially twisted irony that `Rebellion Defence'' seem to have modelled themselves on the rebellion movement in the Star Wars fictional universe, seemingly unaware that Lucas most likely saw the USA as analogous to the Empire in the films \cite{immerwahr202221}. A \href{https://www.stopkillerrobots.org/wp-content/uploads/2022/10/ADR-Artificial-intelligence-and-automated-decisions-Single-View.pdf}{recent report} from ``Stop Killer Robots'' identifies the governance concerns which they have been monitoring for years. They identify areas of specific concern.
\begin{itemize}
\item transparency and explainability \item responsibility and  accountability
\item bias and discrimination.
\end{itemize}
\href{https://vframe.io/9n235/}{munition detection}  and autonomous war robots to do
\subsubsection{Existential Threat}
Many serious researcher are now sounding the alarm at the speed of development. In this moment is seems likely that ascribing emergent behaviours to large language models is a mirage born of the desires of the somewhat accelerationist community, and their more moderate couterparts \cite{schaeffer2023emergent}. There is potentially competitive advantage in mandating a `slowdown' in the technology development if the company and research team you work for find themselves behind. It's hard to tell what's going on. With that said the increase over time of noise generated by fancy autocomplete, vs human signal will doubtless lead (mathematically) to an undoing of current digital society, in the end. This is already impacting the training of new LLM models \cite{shumailov2023curse}. All of that is without considering the wilder theories such as the so called `paperclip miximiser' \cite{bostrom2003ethical}
or what Yudkowsky calls the \href{https://www.lesswrong.com/tag/squiggle-maximizer-formerly-paperclip-maximizer}{``squiggle maximiser''}, a literal end of life scenario.
\subsubsection{Scaling challenges}
In the DeepMind paper that introduced Chinchilla \cite{hoffmann2022empirical} several problems with scaling were highlighted. These finding are being challenged as cleaner data shows improved results.
\begin{itemize}
\item Data, rather than model size, is currently the primary constraint on language modeling performance.
\item Large models (e.g., with 500B parameters or more) are not necessary if enough data can be leveraged.
\item Encountering barriers to data scaling would be a significant loss compared to the potential performance gains.
\item The amount of text data available for training is unclear, and there may be a shortage of general-domain data.
\item Specialized domains, such as code, have limited data compared to the potential gains if more data were available.
\end{itemize}
They argue that researchers often focus on scaling up the number of parameters rather than increasing the amount and diversity of training data. This focus on scaling numbers has led to a lack of clarity and rigour in describing the data collection processes and sources used in training these models. It seems  there are challenges associated with increasing the amount of training data, particularly in terms of quality and the diminishing returns of data scaling. As these barriers are better usderstood so does the community respond with new techniques to process the training data. The outstanding Falcon model simply parsed HTML, removed duplication, and applied so basic filtering to the `common crawl' dataset \cite{penedo2023refinedweb}. Similarly Microsoft created and incredibly efficient code support LLM trained in a matter of hours by using high quality textbooks \cite{gunasekar2023textbooks}.
\begin{itemize}
\item Generated tokens from 3 TB collection of code and text from StackOverflow
\item GPT-3.5 was used to generate 1B tokens of text similar to textbooks in the field
\item Over several days trained a small 1.3B parameter model (phi-1)
\item Used GPT-3.5 to generate text similar to the textbook exercises
\item Fine-tuned the phi-1 on the exercises
\end{itemize} 
The future is clearly not in \textit{more} but rather \textit{better} data. The most up-to-date analysis of scaling complexity is by Kaddour et al. and is well summarised by Anthony Alcaraz \href{https://www.linkedin.com/posts/anthony-alcaraz-b80763155_this-paper-is-a-must-read-for-those-interested-activity-7087728440300158976-wa_f?}{on LinkedIn} \cite{kaddour2023challenges}
 
\subsubsection{Model Design Challenges}

\begin{itemize}

\item Unfathomable Datasets
\begin{itemize}
\item Unable to thoroughly review or audit massive datasets. Leads to issues like benchmark contamination, privacy leaks, and suboptimal domain mixtures.

\end{itemize}

\item Tokenizer-Reliance
\begin{itemize}
\item Tokenizers have drawbacks like computational overhead, language dependence, and information loss.
\end{itemize}

\item Fine-Tuning Overhead
\begin{itemize}
\item Fine-tuning entire LLMs requires prohibitive compute resources and memory.
\item Obviously Low Rank Adaption (LoRA) is currently pushing the field forward in that regard.
\end{itemize}

\end{itemize}

\subsubsection{Model Behavior Challenges}

\begin{itemize}

\item Prompt Brittleness
\begin{itemize}
\item Small syntactic prompt variations can drastically change outputs. Lack of prompt robustness.
\end{itemize}

\item Hallucinations
\begin{itemize}

\item Models generate convincing but incorrect or underdetermined outputs.
\end{itemize}

\item Misaligned Behavior
\begin{itemize}
\item Outputs often poorly aligned with human values, intentions, and expectations.
\end{itemize}

\item Outdated Knowledge
\begin{itemize}
\item Pre-trained knowledge becomes inaccurate over time. Isolated updates remain difficult.
\end{itemize}

\end{itemize}

\subsubsection{Scientific Challenges}

\begin{itemize}

\item Brittle Evaluations
\begin{itemize}
\item Benchmarks are sensitive to small prompt variations. Static human evaluations become outdated.
\end{itemize}

\item Lacking Experimental Designs
\begin{itemize}

\item Papers lack controlled experiments and ablations. Vast hyperparameter space is under-explored.
\end{itemize}

\item Lack of Reproducibility
\begin{itemize}
\item Nondeterminism in parallelized training and blackbox APIs hinders reproducibility.
\end{itemize}

\item Tasks Not Solvable by Scale
\begin{itemize}
\item Some reasoning and generalization tasks exhibit inverse scaling or disproportionate memorization.

\end{itemize}

\end{itemize}
\subsection{A more optimistic interpretation}
It's possible that in a post-social media world where AI assistants become ubiquitous, there is a shift toward more meaningful human-human interpersonal relationships. If everyone has an AI assistant in their ear to help mediate their personal fears and anxieties, then we can perhaps see a path to more positive and local ways to communicate and engage with one another. The presence of AI assistants could conceivably help individuals become better listeners, communicators; ironically more empathetic. By providing personalized support and guidance, AI assistants might help us to better understand our own emotions and reactions, which in turn can lead to more authentic human interactions. If this new technology can undo the damage wrought by social media, and help us navigate our insecurities and anxieties, we may see a decline in the superficiality that has become synonymous with online interactions. This shift could lead to the resurgence of face-to-face conversations and the revitalization of community bonds, as individuals seek more authentic connections with those around them. This is still a worthwhile digital society, so long as the algorithm fuelled polarity fostered in the current is not repeated with weaponised AI assistants. AI assistants may seem less judgemental, and might promote mindfulness of our own biases and preconceptions, allowing us to approach conversations with an open mind and a willingness to learn from others. In this future, AI assistants not only serve as personal coaches and mediators for our individual fears and anxieties but also become catalysts for promoting healthier and more authentic human connections. By supporting our emotional well-being and helping us navigate interpersonal relationships, AI has the potential to transform the way we engage with one another. 
\newpage
\subsection{The Law (not a lawyer!)}
The \href{https://www.holisticai.com/papers/the-state-of-ai-regulations-in-2023}{world}, and more urgently the EU, is starting to frame it's legislative landscape one these technologies:
The USA has published it's ``AI Bill of Human Rights'', highlighting five themes:
\begin{itemize}
\item Safe and Effective Systems
\item Algorithmic Discrimination Protections
\item Data Privacy
\item Notice and Explanation
\item Human Alternatives, Consideration, and Fallback
\end{itemize}
It would be worth digging into these further here were it not so clear that the guardrails have already been breeched. The \href{https://www.whitehouse.gov/wp-content/uploads/2022/10/Blueprint-for-an-AI-Bill-of-Rights.pdf}{full report} is still worth reading.\par

\subsubsection{AI Regulation and Compliance: Navigating the Evolving Landscape}

The global urgency to regulate AI is intensifying, prompting organizations to adapt their governance structures to current and upcoming regulatory requirements. This section provides an overview of existing and new AI regulations in the EU and US, and suggests key action points for navigating AI compliance.

\paragraph{International Soft-Law Approaches}
\begin{itemize}
\item Various international bodies have adopted non-binding guidelines for ethical and responsible AI use. These include the \href{https://oecd.ai/en/ai-principles}{OECD AI Principles}, the \href{https://www.unesco.org/en/artificial-intelligence/recommendation-ethics}{UNESCO’s Recommendation on the Ethics of AI}, and the \href{https://www.whitehouse.gov/ostp/ai-bill-of-rights/}{White House Blueprint for an AI Bill of Rights}.
\item Standardization bodies such as \href{https://www.iso.org/standard/77608.html}{ISO/IEC}, \href{https://standards.ieee.org/initiatives/autonomous-intelligence-systems/}{IEEE}, and the U.S. \href{https://nvlpubs.nist.gov/nistpubs/ai/NIST.AI.100-1.pdf}{NIST} provide additional guidance.
\item These initiatives encourage the responsible and human-centric use of AI, focusing on principles like privacy, data governance, accountability, and human oversight.
\end{itemize}

The legislative overhead (globally - taken in aggregate) is very high, and there's growing concern about the energy and carbon footprint of the technology \cite{wu2022sustainable}. 


\subsubsection{US's Sectoral Approach to AI Regulation}
\begin{itemize}
\item US AI governance is dispersed across various federal agencies and tailored to specific sectors.
\item The Federal Trade Commission (FTC), wielding its \href{https://www.ftc.gov/business-guidance/blog/2021/04/aiming-truth-fairness-equity-your-companys-use-ai}{authority}, is actively involved in AI regulation. Its commitment to enforcing fair AI use was demonstrated by opening an \href{https://www.washingtonpost.com/documents/67a7081c-c770-4f05-a39e-9d02117e50e8.pdf?itid=lk_inline_manual_4}{investigation} into OpenAI's chatbot ChatGPT.
\item The Equal Employment Opportunity Commission (\href{https://www.eeoc.gov/newsroom/eeoc-releases-new-resource-artificial-intelligence-and-title-vii}{EEOC}) and the Consumer Financial Protection Bureau (\href{https://www.consumerfinance.gov/about-us/blog/algorithms-artificial-intelligence-fairness-in-home-appraisals/{CFPB}) are other examples of active sectoral regulators for AI.
\item States are also showing interest in AI regulation, with California introducing \href{https://leginfo.legislature.ca.gov/faces/billTextClient.xhtml?bill_id=202320240AB331}{AB 331}, a law specifically targeting automated decision tools.
\item Public discourse suggests potential for a broader approach, incorporating AI requirements in a comprehensive national privacy law, like the proposed \href{https://www.commerce.senate.gov/services/files/9BA7EF5C-7554-4DF2-AD05-AD940E2B3E50}{American Data Privacy and Protection Act}.
\item In addition to actively governing algorithms, \href{https://carnegieendowment.org/2023/07/10/china-s-ai-regulations-and-how-they-get-made-pub-90117}{China recently introduced AI regulations} targeting generative AI and fake news.
\end{itemize}
The FTC has published a \href{}{blogpost} discussing the legal implications of AI-generated deception under the FTC Act. As synthetic media becomes more widespread, understanding the legal landscape is crucial for businesses involved in creating, distributing, or using AI-generated content. The FTC Act governs deceptive or unfair practices, including synthetic media and generative AI tools. Businesses should consider liability assessment, risk mitigation, deterrence measures, avoiding misrepresentation and false advertising, and post-release detection and monitoring. Other legal frameworks, such as copyright and trademark laws, defamation, and privacy rights, may also apply. The FTC emphasizes the importance of considering the broader social and ethical implications of these technologies and calls for collaboration among legal professionals, regulators, and industry stakeholders to ensure responsible development and use of AI-generated content. It feels like even this comparatively swift response is too little too late. 

\subsubsection{EU's Comprehensive AI Regulatory Approach}
\begin{itemize}
\item The European Commission (EC) introduced the \href{https://digital-strategy.ec.europa.eu/en/policies/regulatory-framework-ai}{draft AI Act in April 2021}, aiming to be the EU's first comprehensive, cross-sectorial regulation focusing on AI.
\item The Act will categorize AI systems based on risk, impose fines for non-compliance, and is currently in the negotiation phase.
\item The \href{https://commission.europa.eu/business-economy-euro/doing-business-eu/contract-rules/digital-contracts/liability-rules-artificial-intelligence_en}{AI Liability Directive} and a revision of sectoral safety legislation, including for \href{https://law.stanford.edu/publications/machine-learning-eu-data-sharing-practices/}{machinery and general product safety}, are also part of the EU's AI strategy.
\end{itemize}

The EU is first mover on this, and they have over-corrected for their lack of knowledge (arguably fair enough):\
The EU regulation has been passed on 14/6 and classifies AI systems into four risk categories: unacceptable risk, high risk, limited risk, and minimal risk. Unacceptable risk AI systems will be prohibited, while high-risk AI systems will be subject to strict requirements. Any training of a model is automatically a foundational model (annoyingly) and therefore high risk. Getting a compliant model will require a demonstration of deep understanding of the workings of the model. It's unclear how anyone will accomplish this and it may end up being a defacto ban on trained models and possibly even optimised ones (\hyperref[sec:LoRA]{LoRA}). Any product derived from a \textbf{compliant} foundational model will be deemed acceptable by default. This question of compliant foundation models is absolutely key as we plan through the 3 year grace period which the EU has allowed. Short of a proper external legal assessment we \textit{must} assume that Immersive is in scope in terms of it's projects, because of the global reach of the law. Any business doing any business in the EU is exposed. Stanford University Centre for Research on Foundation Models have done their own assessment on the compliance of the current models, and they simply aren't \ref{fig:euAIcompliance}. Bloom from Huggingface is closest and is `perhaps' \href{https://huggingface.co/BelleGroup/BELLE_BLOOM_GPTQ_4BIT}{usable in optimised modes} on cloud \href{https://lambdalabs.com/nvidia-h100-gpus}{hardware} \cite{dettmers2022case}. I suggest we just plough on with whatever is most performant but keep and eye on it.

\begin{figure}[H]
    \centering
    \includegraphics[width=0.95\textwidth]{euaicompliance}
    \caption{Stanford's assessment of the compliance of current foundation models vs the EU AI act.}
    \label{fig:euAIcompliance}
\end{figure}

Some other things can be high risk too, like external HR software. Limited-risk and minimal-risk AI systems will be subject to less stringent requirements. 
\begin{itemize}
  \item \textbf{Safety and Transparency:} AI systems used within the EU must be safe, transparent, traceable, non-discriminatory, and environmentally friendly. These systems should be overseen by humans to prevent harmful outcomes.
  \item \textbf{Risk-Based Categories:} The EU AI Act categorizes AI systems based on risk levels, which include unacceptable risk, high risk, limited risk, and a separate category for generative AI like ChatGPT.
  \item \textbf{Banned Practices:} Certain practices, like live facial recognition and scraping of biometric data from social media, are prohibited.
  \item \textbf{Transparency for Generative AI:} Companies may have to publish summaries of copyrighted material used for training their systems, which could be problematic for LLMs due to the vast amount of data involved in training.
  \item \textbf{Designing Prohibitions for Illegal Content:} AI systems should be designed to prevent the generation of illegal content.
  \item \textbf{AI-Generated Content Disclosure:} People must be informed when they are interacting with AI-generated content.
  \item \textbf{High-Risk Activities:} High-risk activities, such as certain uses of AI in law enforcement, will be subjected to tighter regulation.
  \item \textbf{Pre-Deployment Risk Assessments:} The EU proposes to implement a pre-deployment risk assessment process for high-risk models before they are released to the public.
  \item \textbf{Open Source Carve-out:} The act includes provisions for open source AI, but there are exclusions for large open source "Foundation Models" like GPT.
  \item \textbf{Training Data Documentation:} Companies may need to comprehensively document and be transparent about the copyrighted material used in the training of their algorithms, which could be challenging due to the amount and diversity of training data used.
\end{itemize}

It is somewhat unclear where these delimiters are; they will have to clarify it at some point. Burges Salmon Solicitors have provided a public access flowchart which is pretty helpful and is in Figure \ref{fig:euaiflowchart}.

\ref{fig:EUAIflowchart.pdf}. 
\begin{figure}[H]
    \centering
    \includegraphics[width=0.5\textwidth]{euaiflowchart.pdf}
    \caption{EU AI legislation.}
    \label{fig:euaiflowchart}
\end{figure}


The USA and UK meanwhile will likely use the court system to iteratively establish these boundaries over time, though there will clearly be guidance soon(ish). Nobody wants to get caught up in courts around AI.

There will be legislative focus on human oversight. The regulation will require that all AI systems have human oversight, meaning that humans will be able to understand and control how the AI systems operate. This is intended to help ensure that AI systems are used in a safe and ethical manner. I think this is fine in the context of physical events in that we can simply record the current session to text, and discard afterwards, allowing users to ``come to the information desk'' and discuss their signed waiver. \par 

The regulation will require that AI systems be transparent, meaning that users will be able to understand how the AI systems work and how they make decisions. Again, this just goes in the waiver. I am hopeful that there will be boilerplate terms we can use for each jurisdiction in the context of deployed AI. The regulation will require that AI systems be accountable, meaning that the developers and users of AI systems will be responsible for the consequences of their use. This is less clear to me, because we will at `least' be users, so we might need a bit of advice on the implications here.\par
The draft AI regulation intersects with GDPR in a number of ways. For example, GDPR requires that personal data be processed in a fair and transparent manner. So again, it will require that AI systems that process personal data be transparent and accountable. Additionally, the GDPR prohibits the processing of personal data for certain purposes, such as profiling. It rightly prohibits the use of AI systems for social scoring. Non of that is a problem if we don't collect ANY user data extrinsic to the session when we deploy to the public.\par
The regulation will apply to all AI systems that are developed, placed on the market, or \textbf{used in the European Union}, and we can expect that many countries will simply copy/paste this legal framework. It will be enforced by national authorities in each EU member state, which is suggestive of a degree of (risky) legislative arbitrage potential. Perhaps most annoyingly at a pragmatic level, there are a number of technical requirements for AI systems, such as accuracy, and fairness. I suspect we can explain to users that the technology is nascent and use a high quality model to demonstrate best effort, training and testing against our local event dataset. All regulation will also include a number of procedural requirements, such as requirements for risk assessment and documentation. The penalties in the EU for violating the regulations will be significant. For example, up to 4\% of \textbf{global} annual turnover. The timeline for the adoption of the laws is still being negotiated, though the European Commission has stated that it expects the end of 2023, crucially with a two year phase in. This still means pretty much means `anything we do', since there is an extrinsic capture clause in EU law for international companies selling at all into the EU. This is the legislation to watch basically.\par
The Online Safety Bill is the most relevant UK legislative proposal pertaining to AI, as it includes a number of provisions that specifically address the use of AI by online platforms. The bill requires \textit{online} platforms to take steps to prevent the spread of harmful content, \textbf{including content that is generated by AI systems}. The bill also requires online platforms to be more transparent about how they use AI systems and to give users more control over how their data is used. The penalties for non-compliance with the legal provisions in the UK vary depend on the law that is being breached. Fines of up to £10 million for breaches of the DPA 2018. The information commissioners office has \href{https://ico.org.uk/for-organisations/uk-gdpr-guidance-and-resources/artificial-intelligence/guidance-on-ai-and-data-protection/how-should-we-assess-security-and-data-minimisation-in-ai/}{issued vague guidance} and can enforce through notices requiring companies to take steps to comply with the law. The UK seems less burdensome at this time, \textbf{but a quirk of the law gives \href{https://webdevlaw.uk/2022/11/21/a-quick-hypothetical-situation-or-your-crash-introduction-to-the-real-world/}{genuine legal peril} for C-Suite individuals}. I have been quoted nearly £600/hr \href{https://www.linkedin.com/in/barry-scannell-bbb5aa207/}{by this LinkedIn guy}, for legal and risk assessment in the UK.\par
Sunak has signalled that his government wishes the UK to be regarded as a global leader in AI safety. This will obviously engender both opportunities and risks. The new lead of the UK foundation model taskforce, Ian Hogarth, has been quite vocal over the last few years on these matters, and seems a knowledgeable and well considered appointment.\par
In 2018, Hogarth's key concern around AI was the implications of the potential acceleration of AI development and deployment, which he referred to as ``AI nationalism''. He argued that nations would enter into a competitive race to advance AI technology, triggering geopolitical tension and even conflict. He also mentioned the risk of nations stockpiling AI technology and preventing others from accessing it (something we can see in the USA CHIP act), and noted the need for an international, cooperative approach to managing AI.\par
In 2021, Hogarth’s perspective evolved to a more focused concern about the swift progress of AI development, especially in the realm of Artificial General Intelligence (AGI), which he refers to as ``Godlike AI''. This shift was based on his experiences and conversations within the tech industry. He raised the alarm about the rapid speed at which private companies are advancing towards AGI without proper oversight and regulation.  He expressed concern that the race to develop AGI was taking place in a small community of tech companies, without democratic involvement or consideration of potential risks to the human race. These are valid concerns.\par
He, like I, thinks there has been an increasing rush towards the development of AGI, and the people involved seemed more focused on the potential benefits rather than considering the risks. \par 
Despite this, he remains fundamentally optimistic about the potential of science and technology to transform our lives for the better, provided it is done safely. He will have a direct role in shaping the future of AI regulation in the UK. His recognition of the need for governments to get more involved, and for more people to have a say in the AI conversation, potentially represents an invitation and opportunity for businesses. 

\subsubsection{UK Legal Position}

\begin{itemize}
    \item \textbf{Source Reference:}
    \begin{itemize}
        \item This summary is based on the "Responsible and Trustworthy AI" report published by the UK. The full report can be accessed \href{https://iuk.ktn-uk.org/wp-content/uploads/2023/10/responsible-trustworthy-ai-report.pdf}{here}.
    \end{itemize}

    \item \textbf{Strategic Objective:}
    \begin{itemize}
        \item The UK Science and Technology Framework (March 2023) identifies AI as one of five critical technologies. The UK is seeking to build a strategic and globally competitive advantage in AI to become a science and technology superpower by 2030.
    \end{itemize}

    \item \textbf{Core Principles of Responsible and Trustworthy AI (RTAI):}
    \begin{itemize}
        \item \textit{Transparency and Explainability}: Essential for understanding AI systems.
        \item \textit{Safety, Security, and Robustness}: Ensuring AI systems are robust, safe, and secure.
        \item \textit{Non-maleficence}: AI must not harm individuals, society, or the environment.
        \item \textit{Privacy}: Protecting personal information and addressing surveillance concerns.
        \item \textit{Fairness and Justice}: Promoting equality and addressing biases.
        \item \textit{Accountability}: Organisations must take ownership of AI actions and decisions.
    \end{itemize}

    \item \textbf{Policy Direction and Legislation:}
    \begin{itemize}
        \item Guiding the development, deployment, and use of RTAI across the AI lifecycle.
        \item Seeking international consensus on RTAI principles for global alignment.
    \end{itemize}

    \item \textbf{Challenges and Implementation:}
    \begin{itemize}
        \item AI's transformative impacts pose challenges in responsible application.
        \item Necessity for trade-offs when RTAI principles conflict.
        \item Notable cases of principle violations, such as government predictive algorithms and Clearview AI Inc. scandal.
    \end{itemize}

    \item \textbf{Stakeholder Responsibilities:}
    \begin{itemize}
        \item Developers and industry representatives.
        \item Government bodies, regulators, and policymakers.
        \item Standards bodies for technical standards.
        \item Funders and investors supporting RTAI development.
        \item Users, customers, and researchers in public debate.
        \item Advisory bodies and social actors shaping public debate.
    \end{itemize}

    \item \textbf{Guidance and Tools:}
    \begin{itemize}
        \item Existence of guidance documents and tools for implementing RTAI.
        \item Need for more granular operationalisation techniques due to voluntary and variable quality of these tools.
    \end{itemize}
\end{itemize}


\subsection{Humans and AI, horizon scanning}
As AI becomes increasingly integrated into various aspects of our lives, it is poised to transition from being a mere tool to becoming an integral companion and collaborator. The potential trajectory of this evolution covers a wide range of areas, including equity, environment, education, health, governance, and more.
\begin{figure}[ht]
\centering
\includegraphics[width=0.19\textwidth]{base_output_00002_.png}
\hfill
\includegraphics[width=0.19\textwidth]{base_output_00003_.png}
\hfill
\includegraphics[width=0.19\textwidth]{base_output_00005_.png}
\hfill
\includegraphics[width=0.19\textwidth]{base_output_00006_.png}
\hfill
\includegraphics[width=0.19\textwidth]{base_output_00007_.png}
\caption{A progression of human-AI interaction, oneshot collaboration between ChatGPT and SDXL. Placeholder.}
\end{figure}

\subsubsection{The Evolution of Human-AI Interaction: Soon, Next, and Later}
This journey of human-AI integration can be broadly divided into three phases: 'Soon', 'Next', and 'Later'. Each phase brings unique challenges and opportunities.\par 

\textbf{The 'SOON' phase} focuses on Digital Literacy, Data Privacy, and Algorithmic Bias. It's critical that individuals understand how to use AI and digital technologies to access information and services. Additionally, as AI systems often rely on personal data, ensuring privacy and mitigating biases embedded in AI algorithms are key challenges. \par

\textbf{The 'NEXT' phase} will see the rise of Biome AI, AI Ethics \& Safeguarding, and Ambient Education. In this phase, AI will become more integrated into our biological lives and our everyday environments. It will be imperative to ensure the responsible use of AI as its influence grows. This phase will require a focus on developing robust ethical frameworks for AI use, and creating ambient educational environments where learning is facilitated by AI.\par 

\textbf{The 'LATER' phase} anticipates the emergence of Fully Autonomous Systems, where AI will become so advanced and reliable that it can operate without human supervision in various sectors, leading to unprecedented levels of productivity and efficiency. Another exciting aspect of this phase is the intersection of AI and Mental Health. AI companions will be capable of understanding human emotions and mental states, providing psychological support and therapeutic interventions. They could assist in managing mental health conditions and improving overall well-being.

\subsection{Human flourishing and expression}
\paragraph{Celebrating Human Diversity: Now to Later}\par
As AI becomes more integrated into our daily lives, it will play a crucial role in celebrating and promoting human diversity. AI systems will be designed to understand, respect, and adapt to a wide range of human experiences, perspectives, and identities. They will help us to better understand ourselves and each other, breaking down barriers and fostering a more inclusive society. Furthermore, by acknowledging the unique forms of intelligence exhibited by AI, we can redefine what intelligence means in a diverse society.
\paragraph{AI supported creativity: Soon to Next}\par
In the future, we also foresee the rise of AI and Creativity, where AI will not only assist in creating art, music, and literature, but will also be capable of original creative thought, leading to a renaissance of AI-generated creativity that will reshape our cultural landscape. Another exciting prospect is the development of Diverse AI. Just as human diversity is celebrated, AI will be designed with diverse "personalities" and ways of thinking, leading to a richer and more nuanced interaction between humans and AI.

\paragraph{Equity: Now to Next}\par
By democratizing resources, and lowering the marginal cost of education to zero globally, AI will play a pivotal role in reducing socio-economic disparities. However, it's crucial to ensure that AI advancements are distributed equitably and don't reinforce existing social inequities.
\paragraph{Secret Learnings for Children: Next to Later} \par
AI education will include playful, memorable, and slightly subversive learning experiences. AI educators will foster independence and trust in children by providing private, magical moments of learning. Some of these moments might occur in secret, away from adult supervision, encouraging children to explore, experiment, and learn in their own unique ways. All children should have the opportunity to grow up with headphones that contain a local AI informational agent, even without connection to the internet. Access of opportunity increases.
\paragraph{The age of the productive tinker: Later}\par
Increasingly a simple set of small specialised chips (ASICs) will allow low power inference using AI. This will produce an explosion of gadgets for niche applications. As an example AI will revolutionize the fashion industry by co-designing 3D printed single-use clothing tailored to individual preferences. These clothes could be made from environmentally friendly materials, suitable for recycling after use.
\subsubsection{Environment}
The manifestation of AI as a tool to support and celebrate human diversity also has profound implications for our interaction with the environment. As AI systems become more attuned to understanding and adapting to a wide range of human experiences and perspectives, they also have the potential to significantly enhance our relationship with the natural world. This transition from human flourishing to a more sustainable interaction with our environment is the next critical stage in our evolving relationship with AI.
\paragraph{Resilience and collaborative management} 
AI's role in predicting and monitoring environmental changes, optimizing resource consumption, and enhancing waste management will become widespread, while the carbon footprint of AI technologies will need to be minimized. Additionally, future scenarios include AI's role in Climate Change and Wildlife Conservation. Advanced AI models will help in predicting and mitigating the effects of climate change, assisting in the planning and execution of climate resilience strategies at local and global levels. AI will also play a critical role in wildlife conservation, from monitoring endangered species and their habitats, to predicting and preventing potential threats. 

\paragraph{Supporting our place: Later}: 
AI will support human relationships with nature by monitoring and managing our physical health at a micro-level. These systems will provide educational information about our natural environment, helping us make more sustainable choices.
\subsection{Holistic health}
Our interaction with the environment is intrinsically linked to our overall health and well-being. AI’s role in environmental management can directly impact our personal health. As AI technologies evolve to help us predict and respond to environmental changes, optimize resource consumption, and enhance waste management, they also pave the way for more holistic approaches to health management. This progression from environmental stewardship to personalized health care is the next evolution in the integration of AI into our daily lives.
\paragraph{Personal Health Management: Now to Next}: 
AI will also support our relationships with our own health and internal biome. Personalized AI systems will analyze and respond to changes in our biomes to optimize our health, making recommendations based on our specific needs and preferences.
\paragraph{Lifetime Support Structures: Now to Later}
AI will serve as a form of lifetime support, offering guidance throughout various stages of life. These AI systems will learn patterns, habits, and preferences over time, providing customized assistance, reminders, and suggestions to improve wellbeing and productivity.
\subsection{The age of the informational Agent}
As trust in the internet evolves, the situation will change and adapt as people learn to use informational agents. Intentional UX will allow users to ask their AI to bring them the information they want from a far more decentralized and confusing internet. This will promote incredible diversity in humans as they fracture somewhat into informational enclaves.

Two future scenarios include the rise of AI in Truth Verification and Education. AI will become sophisticated enough to verify the truthfulness of information on the internet, combating the spread of misinformation and ensuring that users have access to accurate and reliable information. AI will also become an integral part of education, acting as a personalized tutor that adapts to each student's learning style and pace, revolutionizing the way we learn and acquire new skills.

\paragraph{Money will change: Next -Later}: Money will increasingly be spent by machines on behalf of human intent. Algorithms will bargain and arbitrage on behalf of their owner/user, for human and machine services, and for ideas, alorithms, and good; gloablly. This will smooth out the operation of money globally.

\paragraph{Ubiquitous multi-modal UX: Next to Later}
The future will see the rise of ubiquitous displays, computing \cite{lyytinen2002ubiquitous}, and interfaces - screens and projections seamlessly integrated into our environments. These displays will not only provide information and entertainment but will serve as interfaces for our interactions with AI. Gesture may come to the fore, and subvocalisation to talk to personal AIs will be possible.


\subsection{Risks, Responsibilities and Mitigation Strategies for AI in Events}

\subsubsection{Risks and Responsibilities}

\begin{itemize}
\item \textbf{Privacy and data protection} - The AI system will likely process significant personal data, so complying with regulations like GDPR is crucial. Strategies like data minimization, transparency, and consent mechanisms should be implemented.

\item \textbf{Algorithmic bias} - With hyper-personalization, there is a risk of discriminatory outcomes if the algorithms embed societal biases. Rigorous testing for fairness and regular audits are needed.

\item \textbf{Transparency} - The EU AI Act requires high levels of transparency for users. The company must ensure the AI logic is explainable and accessible.

\item \textbf{Cybersecurity} - Customer data could be vulnerable, so strong cybersecurity protections are vital, like encryption and access controls.

\item \textbf{AI safety} - Irresponsible AI could recommend harmful events. Oversight systems like human-in-the-loop review are important safeguards.

\item \textbf{Legal liability} - If AI-recommended events cause harm, legal liability questions arise. The company should clarify liability through contracts and insurance.

\item \textbf{Copyright} - Training datasets likely contain copyrighted content. Proper licensing and credits are needed.

\item \textbf{Environmental impact} - Energy usage and carbon footprint of large AI models should be managed through efficiency measures.

\item \textbf{Industry disruption} - Jobs like event planners could be impacted. Transition support for affected workers can demonstrate social responsibility.
\end{itemize}

\subsubsection{Mitigation Strategies}

\begin{itemize}
\item Perform comprehensive risk assessments accounting for EU regulations.

\item Implement "ethics by design" in development and deployment.

\item Establish oversight boards, audits, and impact assessments.

\item Clearly communicate AI capabilities, limitations and roles to users.

\item Develop strict protocols for obtaining user consent. Enable user controls over data.

\item Test extensively for algorithmic bias and other harms using diverse datasets.

\item Engineer transparency into the AI models and provide explanations of outputs.

\item Implement cybersecurity best practices and controls like multi-factor authentication.

\item Establish human oversight procedures and feedback loops to check AI recommendations.

\item Clarify legal culpability and seek appropriate insurance coverage.

\item Formally credit all copyrighted training data. Seek explicit licensing where feasible.

\item Adopt energy-efficient computing infrastructure. Purchase carbon offsets.

\item Provide job retraining opportunities to affected professionals.
\end{itemize}


\section{Generative art systems}
Generative art refers to art that is generated algorithmically rather than manually created. In this report, we will provide an overview of generative art, including its history, key techniques, applications, and future outlook. \par 
Generative art emerged in the 1960s alongside early computer art. Artists like Frieder Nake and Georg Nees used algorithms to create abstract visual art. In the 1970s and 80s, fractal geometry allowed for complex recursive patterns. More recently, neural networks have enabled new forms of image generation, and they are the focus of this text.
\subsection{Modern Models and Systems}

\subsection{Image Generation}

\begin{itemize}
\item \textbf{Stable Diffusion} - An open source diffusion model capable of creating realistic images from text prompts. Widely used by artists due to:
\begin{itemize}
\item Flexible and intuitive text prompts
\item Many interfaces and extensions for control, most notably controlnet for our puposes
\item Ability to fine-tune on custom datasets
\end{itemize}

\item \textbf{DALL-E 2} - First to market, it's a proprietary generative AI system from OpenAI that creates images from text captions. Key features:
\begin{itemize}
\item Fairly hotorealistic images
\item Diverse creative capabilities from prompts
\item Requires OpenAI access and paid credits
\item Very long in the tooth now (8 months old), and fallen from favour\end{itemize}

\item \textbf{Midjourney} - Web-based generative art tool with social community aspects. Notable for:

\begin{itemize}
\item Easy to start generating images quickly
\item Built-in sharing and voting on generations
\item Best in class ``vibes''
\item Limited free tier with paid subscriptions
\item Privacy is questionable
\item Subject to change, making consistency of approach impossible.
\end{itemize}

\item \textbf{Imagen} - AI system from Google producing images from text. Characteristics:
\begin{itemize}
\item Very high-resolution outputs
\item Control over lighting and detail
\item Currently restricted to limited partners
\end{itemize}

\end{itemize}

\subsection{How they work}
There's a good detailed and in depth blog post by Weng \href{https://lilianweng.github.io/posts/2021-07-11-diffusion-models/#classifier-free-guidance}{here}.

The four main types are over-viewed \href{https://www.linkedin.com/feed/update/urn:li:activity:7088752096853803008/}{by Deshwal} as below:

\begin{itemize}

\item [GAN: Generative Adversarial Network] "Adversarial" as the name suggests are two opposite networks. One learns to create images out of noise (Artist) which is actually very hard process and the other says "umm okay! This isn't good" (Critic) which is a relatively easy process. So because of the fact that being an artist is very hard than being a critic, these networks are not stable and Critic learns faster than the Artist.

\item [Auto Encoder style] VAE, U-Nets etc belong to this category where same network breaks down image in deeper level features using CNN and then rebuild it again. That's like a child learning by breaking things. VAE and U-Nets are almost same with minor differences and serve as a base model in Diffusion process so that think of them ans analogues to BERT in LLMs.

\item [Flow Based]: Well here it becomes complex. You apply function X to an image and then you re-create the original image by applying the Exact inverse of that function. For example, a very basic function is to add 5 and then subtract 5 to get original stuff. 

\item [Diffusion processes] VAE, U-Nets are used as base models. You insert some pseudo random (because you know what you added based on a timestamp "T") Gaussian noise to an image and instead of asking the model to predict original image, you ask the model to predict the Noise that you inserted. Since Gaussian is an additive noise independent of original signal, you subtract that from image and get original image. Another piece is that instead of predicting whole noise at once, you predict the noise for previous (T-1) step.
\end{itemize}

Intuitive interfaces provide easy access to these models. \href{https://github.com/AUTOMATIC1111/stable-diffusion-webui}{Automatic1111's Web UI} offers a full-featured frontend to Stable Diffusion. \href{https://www.midjourney.com}{Midjourney} provides a social platform to create, share and discuss AI art. \href{https://www.runwayml.com}{Runway} delivers generative models through a subscription service.

Fine-tuning techniques like \href{https://arxiv.org/abs/2208.12242}{DreamBooth} allow customizing Stable Diffusion by training on datasets of specific concepts. \href{https://github.com/yfszzx/stylegan-nada}{styleGAN-NADA} improves image quality through noise optimization. \href{https://github.com/NVlabs/stylegan3}{styleGAN3} introduces a generator architecture that achieves state-of-the-art results. Extensions like \href{https://github.com/mit-han-lab/gaugan}{Gaugan} bring control over seasons, weather, lighting and more.

Vibrant communities continually push boundaries of generative art through platforms like the \href{https://www.reddit.com/r/deepdream/}{deepdream subreddit}, the \href{https://github.com/dvschultz}{ML Art Colabs repository}, and the \href{https://stability.ai/blog}{Stability AI blog}. Events like \href{https://www.elfman.ai/}{Butterfly Dream} showcase creativity.

Beyond generation, image processing techniques can manipulate existing visuals. \href{https://github.com/TencentARC/GFPGAN}{GFPGAN} restores blurred faces using facial landmarks and semantic segmentation. \href{https://github.com/dazhizhong/BRDNet}{BRDNet} removes unwanted objects through edge-aware propagation and diffusion. \href{https://github.com/xinntao/Real-ESRGAN}{Real-ESRGAN} super-resolves images up to 16x resolution. \href{https://github.com/jantic/DeOldify}{DeOldify} colorizes black and white photos through deep learning. Such techniques enable restoring, retouching and enhancing images.

\subsubsection{Stable Diffusion}

\subsubsection{Stable Diffusion Ecosystem}

As an open-source diffusion model, \href{https://arxiv.org/abs/2105.05233}{Stable Diffusion} has given rise to an extensive ecosystem of models, interfaces, extensions, and communities.

\paragraph{Models} The core Stable Diffusion repository provides strong baselines like \href{https://github.com/CompVis/stable-diffusion}{sd-v1-4} optimized for speed and \href{https://github.com/Stability-AI/stablediffusion}{sd-v2-1k} for 1024x1024 resolution. Models exist for specific domains like anime, manga, and hentai.

\paragraph{Interfaces} Many open-source frontends provide access to Stable Diffusion:
\begin{itemize}
\item \href{https://github.com/AUTOMATIC1111/stable-diffusion-webui}{Automatic1111's Web UI} - full-featured frontend with extensions
\item \href{https://github.com/alembics/disco-diffusion}{Disco Diffusion} - focused on creative exploration
\item \href{https://github.com/hlky/stable-diffusion}{Stable Diffusion GUI} - cross-platform interface supporting Google Colab
\item \href{https://github.com/invincible-sam/A1111-SD-webui-colab}{A1111-SD-webui-colab} - run Stable Diffusion entirely in Google Colab
\end{itemize}

\paragraph{Extensions} Additional modules provide enhanced control:
\begin{itemize}
\item \href{https://github.com/lllyasviel/ControlNet}{ControlNet} - mask-based image editing
\item \href{https://github.com/alembics/vedaso}{Vedaso} - creative effect brushes
\item \href{https://github.com/camenduru/stable-diffusion-webui-tuner}{Stable Diffusion Tuner} - fine-tune model inside Web UI
\item \href{https://github.com/invoke-ai/InvokeAI}{InvokeAI} - optimized inference and rendering
\end{itemize}

\paragraph{Training \& Tuning} Stable Diffusion can be customized:
\begin{itemize}
\item \href{https://github.com/Stability-AI/dreambooth}{DreamBooth} - fine-tune on specific concepts
\item \href{https://github.com/yfszzx/stable-diffusion-stability-ai}{Stable Diffusion Tuning} - improve image quality
\item \href{https://github.com/d8ahazard/sd_hires_face_fix}{SD highres fix} - enhance face quality
\end{itemize}

\paragraph{Community} Vibrant communities continually advance Stable Diffusion:
\begin{itemize}
\item \href{https://www.reddit.com/r/StableDiffusion/}{StableDiffusion subreddit} - sharing creations and discoveries
\item \href{https://discord.gg/stabilityai}{Stability AI Discord} - dedicated channels on SD topics
\item \href{https://civitai.com}{Civitai} - central model hub with versioning
\end{itemize}

\subsubsection{ComfyUI}

\href{https://github.com/comfyanonymous}{ComfyUI} is a feature-rich set of tools and libraries for building interactive web applications using the \href{https://reactjs.org}{React} framework. It makes creating beautiful, functional UIs easy through:

\begin{itemize}
\item \textbf{React-Based} - Built on React for modular, reusable components
\item \textbf{Declarative} - Describe desired UI state without implementation details
\item \textbf{Extensible} - Easily add custom components and functionality
\item \textbf{Testable} - Designed for confident testing of UI behavior
\item \textbf{Documented} - Well-documented for easy learning
\item \textbf{Community} - Large active community for help and support
\end{itemize}

Extensions provide additional capabilities:

\begin{itemize}
\item \textbf{Components} - Pre-built React components for common UI elements like buttons, menus, and forms
\item \textbf{Animations} - Animated React components for engaging UIs
\item \textbf{State Management} - Tools for managing UI state

\item \textbf{Testing} - Utilities for testing ComfyUI applications
\end{itemize}

Other notable features include:

\begin{itemize}
\item \textbf{Responsive Design} - Components auto-adjust layouts for any device size
\item \textbf{Internationalization} - Support for global applications in different languages
\item \textbf{Accessibility} - Interface remains usable by people with disabilities
\end{itemize}

The ComfyUI ecosystem is constantly evolving with new extensions created by the vibrant community. With its versatility, extensibility and helpful userbase, ComfyUI empowers developers to create beautiful, functional UIs for diverse web applications. The declarative programming style and reusable components help quickly build interfaces that are responsive, accessible, and internationalized. 

\subsection{Video generation}


This is incredibly fast moving area and I have many many links in my Mindmap. This section is a placeholder really, I wouldn't act on it.

\begin{itemize}
\item \textbf{DALL-E 3D} - 3D model generation by Anthropic using principles from DALL-E 2. Allows:
\begin{itemize}
\item Text-to-shape generation
\item Control over materials and lighting
\item Interaction with object geometry
\end{itemize}

\item \textbf{Xformation} - Proprietary 3D generation system capable of modifying shape from images.
\begin{itemize}
\item Deforms template 3D objects to match 2D images
\item Controllable 3D effects from image edits
\end{itemize}

\item \textbf{Text2Mesh} - Leveraging Stable Diffusion for text-based 3D model generation.
\begin{itemize}
\item 3D stylization based on natural language input
\item Control mesh topology and appearance
\end{itemize}

\end{itemize}


Extending image synthesis techniques, models like \href{https://arxiv.org/abs/2105.05233}{Stable Diffusion} are being adapted to generate artificial video content. Dedicated systems like \href{https://www.anthropic.com/research/phenaki}{Phenaki} and \href{https://runwayml.com}{Runway} enable text-to-video generation with control over length, resolution and scene contents.

Creating smooth, consistent video requires modeling inter-frame coherence. Techniques like \href{https://ebsynth.com}{EBSynth} achieve this through interpolation and style transfer between frames. \href{https://www.fastvideoai.com}{FastVid2Vid} matches latent vectors between frames to improve consistency. \href{https://nvlabs.github.io/instant-ngp}{Instant Neural Graphics Primitives} (Instant-NGP) learns a temporal model over sequences of frames.

Existing videos can also be enhanced through diffusion models. Techniques enable automatically increasing resolution, translating scenes to different styles, editing objects seamlessly, and more. However, concerns exist around deepfake videos and synthetic media. Moderation systems like \href{https://www.anthropic.com}{Anthropic's Claude} may provide remedies.

Overall, rapid progress in generative video hints at possibilities of creating immersive films, VR experiences, lifelike avatars and more. But thoughtful governance frameworks are necessary to manage risks.

\subsection{Audio generation}

Recent breakthroughs have also extended AI synthesis to the audio domain, enabling applications like text-to-speech, music composition, and editing podcasts.

Models like \href{https://jukebox.openai.com/}{Jukebox} and \href{https://github.com/ facebookresearch/jukebox}{Facebook's Jukebox} generate novel music conditioned on genres, artists, and lyrics through a hierarchical VQ-VAE framework. Meanwhile, \href{https://github.com/coqui-ai/TTS}{Coqui TTS} and \href{https://github.com/NVIDIA/tacotron2}{Tacotron 2} convert text to human-like speech using end-to-end neural architectures.

For editing audio, tools like \href{https://riptide.ai/}{Riptide} remove vocals from songs, while \href{https://www.descript.com/}{Descript} inserts music and trims silences in podcasts. However, bad actors could exploit such capabilities for impersonation fraud and fake media. Strong governance models are critical.

Looking ahead, advances in generative audio may enable creating interactive AI companions, realistic speech synthesis, and personalized music experiences. But maintaining public trust through transparency and accountability will be essential.

\subsection{3D generation}

3D shape generation has also made strides through AI, transitioning text-to-image breakthroughs to the 3D domain. Methods like \href{https://nv-tlabs.github.io/GA-fusion}{GA-Fusion} combine GANs with gradient-based optimization for high quality results. \href{https://github.com/autodeskailab/clip-forge}{CLIP-Forge} matches rendered images with CLIP embeddings to guide optimization. \href{https://threedle.github.io/3dhighlighter}{3D-Highlighter} localizes text prompts on shapes by comparing CLIP similarities.

Other approaches focus on reconstructing shapes from images. \href{https://x-a-i.github.io/xformation}{XFormation} deforms template 3D shapes to match target views. \href{https://sketching-the-future.github.io}{Sketch-Guided Vision Models} optimize an SDF to match input sketches. \href{https://ajayj.com/dreamfields}{Dream Fields} uses a NeRF parameterized by FiLM.

Such techniques could enable creators to manifest their ideas into 3D worlds. However, thoughtful governance is critical to reduce risks associated with impersonation, toxic content, and intellectual property. Community building, education, and responsible deployment will help realize the positive potential of AI.

\subsection{Conclusion}

Rapid progress in AI has unlocked breakthrough capabilities for synthesizing realistic content across images, video, audio, and 3D geometry. However, concerns around biases, misinformation, and toxic content necessitate responsible development and deployment of these technologies. Maintaining public trust through transparency, accountability and inclusivity will be key to ensuring that the benefits of AI progress outweigh the risks. If harnessed judiciously and ethically, generative AI could augment human creativity in unprecedented ways. But it should empower rather than replace us. Ongoing advances in AI safety and governance will help achieve this vision



\subsection{Explore and document enterprise AI}
For this micro project we briefly explore and document setting up a `proof of concept' project work-flow assistant for Immersive. What we'd like to explore here is the current state of workarounds for the token limit problem. In a business context an LLM should ideally provide accurate, suitable, specific and/or actionable responses when prompted about things to do with our work's processes and work-flows. It is fair to say that LLMs are pretty awful at this. That is likely to change pretty soon. \par
\textbf{There are several options to look at, with different trade-offs. Keeping an eye on the right tool for the job in the medium term will give a competitive advantage for a while}, and prevent the replication of effort many companies will have to go through, as they pivot between approaches.
\subsubsection{Some ways of thinking}
Poulter \href{https://twitter.com/jamespoulter}{CEO} of Vixen Labs has come up with a somewhat strained analogy he calls ``The Central Intelligence Agent''. I'm going to include it until I find something better because I think he's struck on something by dividing up the company needs with his taxonomy (Figure \ref{fig:vixenAnthro}).
\begin{figure}[H]
    \centering
    \includegraphics[width=0.95\textwidth]{vixenanthro}
    \caption{Vixen Labs anthropomorphic taxonomy of business functions}
    \label{fig:vixenAnthro}
\end{figure}
\href{https://observer.com/2023/06/ceo-ai-survey-yale-professor/}{``Renowned Yale Professor Jeffrey Sonnenfeld Discusses CEOs’ Fear and Confusion of A.I''} in a recent survey and presentation, and this is worth a quick look.\par
He sees five groups (per the article):
\begin{itemize}
\item ``Curious creators'' argue everything you can do, you should do. (Venture capitalist Marc Andreessen recently expressed a similar view in a blog post about A.I.)
\item ``Euphoric true believers'' only see the good in technology.
\item \textbf{``Commercial profiteers'' don’t necessarily understand the new technology but are enthusiastically seeking to cash in on the hype.}
\item ``Alarmist activists'' advocate for restricting A.I.
\item ``Global governance advocates'' support regulation and necessary crackdown.
\end{itemize}
This seems a pretty simplistic set of buckets, but he's got a decent dataset, and he's -very- eminent so perhaps we should realistically see ourselves in the emboldened commercial profiteer category. I think this kind of self reflection is important when dealing with things this potentially transformational. Sonnenfeld said: \textit{``As Robert Oppenheimer warned us, it can be very dangerous to think that technology only takes us to the best of the world.''}
\subsubsection{Trusted enterprise AI}
Enterprise AI, specifically designed or retrofitted for professional use cases, is becoming a significant theme. This trend is driven by leading companies such as Google, Microsoft, and latterly, curiously, Salesforce. Trust has become a primary consideration. At this early stage in the technology it is important that corporate and private users alike bear in mind that the LLMs are `leaky' and are using the data fed into them to train themselves. They are \href{https://help.openai.com/en/articles/6783457-chatgpt-general-faq}{explicit about this}. Anything that goes into ChatGPT can resurface, as Samsung have found out \href{https://cybernews.com/news/chatgpt-samsung-data-leak/}{to their cost}. At this time the following companies have responded by banning the use of the technology internally. 
\begin{itemize}
\item Apple
\item Samsung
\item Verizon
\item Bank of America Corp.
\item Citigroup Inc.
\item Deutsche Bank AG
\item Goldman Sachs Group Inc.
\item Wells Fargo \& Co.
\item JPMorgan Chase \& Co.
\item Amazon's lawyers recommended caution, though a recently leaked document suggests that managers are recommending it's use.
\end{itemize}
There are likely \textit{many} more who have done this less publicly, and indeed I am aware of examples. In response to this \href{https://openai.com/blog/new-ways-to-manage-your-data-in-chatgpt}{OpenAI announced} a business version of its tool, ChatGPT Business. Clearly this is a premium subscription service designed to be private by default. The service manages multiple users and does not train its future models on any conversations that flow through the business. This approach is a significant step towards establishing trust in AI tools, as it ensures that sensitive business conversations are not used to train AI models.\par
As mentioned, Salesforce has been partially AI-powered for years. They recently announced a series of AI-related developments, including the pilot of `Einstein GPT', dubbed ``the world's first generative AI for CRM''. This tool builds on an existing underlying intelligence layer called Einstein, which has been running in Salesforce since 2016. The new generative Einstein GPT is more content-oriented, helping businesses auto-generate text, pictures, and code. This tool is designed to help sales teams find the most likely next customer to buy. More interestingly they are leveraging their expertise in `Salesforce Ventures' a \$500 million fund focused on funding generative AI startups. They have already invested in major projects like Humane, Triple, Anthropic, and Cohere.\par
They have \textit{also} announced an AI Cloud suite: `Salesforce's AI Cloud'. It includes nine GPT-powered applications designed to automate and enhance various business processes. These applications include Sales GPT, Service GPT, Marketing GPT, Commerce GPT, \textbf{Slack GPT}, Tableau GPT, Flow GPT, and Apex GPT. \par
Each of these applications is designed to cater to specific business needs, such as personalizing text generation for emails, automating mundane tasks, customizing messages for different audience segments, and creating workflows from natural language prompts. You can see that our work is already a customer here and this could be built upon.\par 
This suite emphasizes the `Einstein GPT Trust Layer', designed to ensure no potential data leakage, allowing enterprises to use AI for their most sensitive needs. They say this trust layer sits between customer data and the AI models, ensuring that the AI capability can provide predictions without actually looking inside the data. This approach would allow our work to leverage the power of enterprise AI without sacrificing data privacy and/or security.\par
Elsewhere in enterprise AI:
\begin{itemize}
\item Accenture announced a \$3 billion investment into their data and AI practice. This investment includes doubling their talent to 80,000 people, launching an AI navigator for Enterprise platform, and starting accelerators for data and AI readiness across 19 different industries. 
\item Contextual, an enterprise-focused AI startup, recently launched out of stealth with \$20 million in seed funding. 
\item Glean announced a \$100 million round and introduced a workplace chatbot called the Glean Chat Business. 
\item Olive: This healthcare automation startup has raised \$450 million in fresh capital to build out its enterprise AI for hospitals 1.
\item Welltok: This company provides a cloud-based employee wellness platform and has raised \$355 million 1.
\item Outreach: This sales engagement platform has raised \$114 million in series E funding 1.
\item Stackpath: This cybersecurity startup has raised \$396 million in funding 1.
\item Cohere, which is another business-focused AI startup, recently announced a \$270 million series C funding round 2, and are partnered by \href{https://www.oracle.com/news/announcement/oracle-to-deliver-powerful-and-secure-generative-ai-service-for-business-2023-06-13/}{Oracle}.
\item Tunisian enterprise AI startup InstaDeep has also raised \$100 million in Series B financing led by Alpha 3.
\item There's a raft of tools like \href{https://customgpt.ai/}{CustomGPT}, or day planner \href{https://www.beforesunset.ai/}{Before Sunset} that promise to take your data and make it smart by leveraging their deals with the big cloud providers. The prime example of this approach is \href{https://blog.dropbox.com/topics/product/introducing-AI-powered-tools}{Dropbox AI}, which claims to bring the Apple spotlight experience, with ChatGPT smarts, to clouds data. I don't have confidence that I know enough about this, and it seems to be the purview of the AI-Club. If you have a use case there's likely a product, but we'd have to project plan it in specifically and find the right fit and cost/benefit.
\end{itemize}
Taken overall these investments indicate a strong belief in the transformative potential of AI in the enterprise sector.\par
With all this said it's possible the technology is over-hyped. While some believe that AI will disrupt industries in unimaginable ways, others argue that the technology still has a long way to go. Some even argue that the distracting nature of the platforms may be net negative in the short term. Regardless, the current state of Enterprise AI represents a pivotal moment, with companies trying to productise AI and change workflows within large corporations. The impact of these developments could range from a promise of transformation with AI being every bit as disruptive as everyone says it is, to an overhyped flop, as often happens with new technologies. Some industry analysts argue that we're still in the early stages of AI's potential impact. Tech analyst Dan Ives likens the current state of AI to a ``Gold Rush'' moment, suggesting that we're closer to 1995 than 1999 in terms of AI's evolution and impact on industries. This perspective suggests that while AI has made significant strides, there's still a long way to go before it fully transforms the business landscape. I would tentatively agree with this analysis, and avoid over investing in low confidence FOMO plays.
\subsubsection{Brute forcing ChatGPT4 with contexts}
While we await `ChatGPT Business' it's still possible to explore using the tooling. The ChatGPT4 system is so far out ahead of everyone else it's sometimes useful to consider using it for business by adding in carefully crafted chunks of context data to refine how it answers. This is prompt engineered. A fine example of this everyday use of the technology can be found in \href{https://clarifycapital.com/the-future-of-investment-pitching}{this clarify capital report} which finds that ChatGPT created pitch decks are \textbf{far} more compelling than human created ones. It can be developed onward for more complex corporate proposals like this through the API, which is a subscription service, with additional tiers for heavy corporate use cases (\ref{sec:microsoft}).
\begin{itemize}
\item \textbf{Advantages}
\begin{itemize}
\item Can be trialled in the web interface, spending a few hours or perhaps days building a custom context that solves a specific use case for the business, then simply copy/pasting.
\item \textbf{OpenAI GPT is incredibly cost effective} (\$1 for around 700 pages for GPT3.5 performance), or \$20 a month for the web subscription.
\item GPT4 is \textbf{way} ahead in terms of performance. Possibly 18 months ahead of open source.
\item Incredible team and partnership. Plug-ins are arriving very fast to solve real business problems. They have scale and velocity. 
\end{itemize}
\item \textbf{Disadvantages}
\begin{itemize}
\item In terms of future project planning 18 months isn't that long, open source is worth investing in for the sake of differentiation in those time-scales.
\item It's a very general model, refining this through the API means programming work. This is a known unknown with staffing costs.
\item You're necessarily giving your commercial data to a cloud service.
\item Their ``corporate'' private package has trust implications, and cost implications (more in the next section).
\item \textbf{It's reliant on a reliable internet connection, so it's suitable for the office but perhaps not `site'. Using it might therefore mean ending up investing time in two development tracks}.
\end{itemize}
\end{itemize}
One of the incredibly frustrating things about the GPT series is that OpenAI are changing the code behind the models all the time as seen in figure \ref{fig:GPTchanges}. This makes it hard to build upon in a trustable way \cite{chen2023chatgpts}.
The team built a dataset with 50 easy problems from LeetCode and measured how many GPT-4 answers ran without any changes. The March version succeeded in 52\% of the problems, but this dropped to a pale 10\% using the model from June. I assume that OpenAI pushes changes continuously, but we don't know how the process works and how they evaluate whether the models are improving or regressing.
In my opinion, this is a red flag for anyone building applications that rely on GPT-4. Having the behavior of an LLM change over time is not acceptable.

\begin{figure}[H]
    \centering
    \includegraphics[width=0.95\textwidth]{gptchanges}
    \caption{Changes to GPT models performance over time - denied by OpenAI (Chen et al) \cite{chen2023chatgpts}}
    \label{fig:GPTchanges}
\end{figure}

 
\subsubsection{ChatGPT - Code Interpreter}
The ChatGPT Code Interpreter Plugin, introduced in March 2023, offers a sandboxed environment featuring a working Python interpreter. This environment, which is isolated from other users and the Internet, supports an impressive array of functionalities. It comes pre-loaded with over 330 libraries, including popular ones such as pandas, matplotlib, seaborn, and TensorFlow, among others.

As illustrated in Figure \ref{fig:chatGPTdata}, the Code Interpreter Plugin is capable of performing a myriad of tasks. For example, it can visualize any data inputted by the user, generate GIFs of the visualizations, and perform file uploads and downloads. It can extract colors from an image to create a color palette, and autonomously compress large images when memory is running low. Moreover, the plugin can clean and process data, generate insightful visualizations, and convert files to different formats quickly and efficiently.

The Code Interpreter Plugin can be installed by ChatGPT Plus users in a few simple steps. However, it is worth noting that while this plugin is powerful, it does have certain limitations, such as frequent environment resets, limited Optical Character Recognition (OCR) capabilities, and an inability to access the web. Despite these limitations, OpenAI continues to work on improving the capabilities of the Code Interpreter Plugin, promising a future with substantial impacts on the world of programming.

\begin{itemize}
\item The Code Interpreter Plugin introduces a sandbox and an advanced language model, both of which are critical to its functionality.
\item The emphasis of the plugin is on the quality of the model, which can generate code, debug it, and even decide when not to proceed without human input.
\item The plugin offers substantial model autonomy, enabling it to work through multiple steps of code generation autonomously.
\item Despite its powerful capabilities, the plugin does have limitations, such as frequent environment resets and limited OCR capabilities.
\item The plugin is only available to ChatGPT Plus users, and requires a few simple steps for installation.
\item The Code Interpreter Plugin represents a significant advancement in the realm of programming, changing the way programmers interact with AI systems.
\end{itemize}
\subsubsection{Go all in with Microsoft}
\label{sec:microsoft}
%Much like the above option, this is the ``corporate offering'' where they can train what you need, hand hold you through it, give you their servers. It's a `whatever you want and can afford'' option, which was rumoured to be in 6 figures before they will even onboard you. That has just changed with \href{https://azure.microsoft.com/en-us/blog/build-next-generation-ai-powered-applications-on-microsoft-azure/}{Azure AI studio}, which I haven't dug into yet. 
Microsoft have -just- released GPT4 which privately works on your own data. \href{https://techcommunity.microsoft.com/t5/ai-cognitive-services-blog/introducing-azure-openai-service-on-your-data-in-public-preview/ba-p/3847000}{This is likely the best option on the market right now}.
\subsubsection{Anthropic - Claude 2}
Claude-2, Anthropic's ChatGPT competitor was just released. It's cheaper, stronger, faster, can handle PDFs, and supports longer conversations.
\begin{itemize}
\item Claude is 5x cheaper than GPT-4.

\item It has more recent data. A a mix of websites, licensed data sets from third parties and voluntarily-supplied user data from early 2023.

\item It outperforms GPT4 on the GRE writing and HumanEval coding benchmarks.

\item It features a context window of 100,000 tokens, the largest of any commercially available model.

\item It can analyze roughly 75,000 words, about the length of “The Great Gatsby".

\item It can easily handle any code related tasks.
\end{itemize}

\subsubsection{Llama 2}
The new Llama 2 model from Meta looks initially exciting but is pretty mired in legal detail compared to the emerging open source community efforts.
\paragraph{License Grant}

You are granted a non-exclusive, worldwide, non-transferable, royalty-free license to use, reproduce, distribute, modify, and create derivative works of Llama 2.
\paragraph{Attribution and Acceptable Use}

You must retain the attribution notice in all copies of Llama 2.
Your use must comply with Meta's Acceptable Use Policy, which prohibits illegal, deceptive, dangerous, or harmful uses.
\paragraph{Commercial Use and Model Improvement}

You cannot use Llama 2 to improve any other large language model besides Llama 2.
If your products or services have over 700 million monthly active users, you must obtain a separate license from Meta.
\paragraph{Disclaimer, Liability, and Ownership}

Llama 2 is provided "as is" with no warranties. You assume all risks from use.
Meta has no liability for damages arising from use of Llama 2.
You own any derivative works and modifications you create, subject to Meta's ownership of Llama 2.
\paragraph{Termination and Risks}

Meta can terminate the license if you breach it. You must delete Llama 2 on termination.
Be aware of regulations like Article 28b of the AI Act in the EU. Do appropriate diligence to comply with laws and address risks around bias, fairness, transparency, and safety.
\paragraph{Key Takeaways}

Understand attribution requirements, acceptable use policy, commercial use limits, disclaimer, risks, and ownership provisions.
Seek legal counsel given complexities.

\subsubsection{Roll your own trained LLM}
This costs around \$500k to train something up from a trillion tokens that you bring to the party. This gets to `last years' GPT3 level. It's too much, but it's worth being aware of. It's worth noting that Cerebras are offering access to their \href{https://www.cerebras.net/press-release/cerebras-unveils-andromeda-a-13.5-million-core-ai-supercomputer-that-delivers-near-perfect-linear-scaling-for-large-language-models}{Andromeda cluster} which can train a significant model in the vein of Llama in around 11 days. 
\subsubsection{Wait for the Google integrations}
I very strongly suspect that corporate level ML assistance is coming in force to the Google stack already employed at our work. This is \textbf{by far the most likely and pragmatic solution for the `project planning assistant' business case}. \href{}{Vertex AI} cloud based generative art support shows the direction of travel in this regard.  \begin{tcolorbox}[enhanced, frame style={fill=lightgray}, interior style={fill=lightgray}]
With this update, developers can access our text model powered by PaLM 2, Embeddings API for text, and other foundation models in Model Garden, as well as leverage user-friendly tools in Generative AI Studio for model tuning and deployment. Backed by enterprise-grade data governance, security, and safety features, Vertex AI can make it easier than ever for customers to access foundation models, customize them with their own data, and quickly build generative AI applications.
\end{tcolorbox}
\begin{itemize}
\item Advantages
\begin{itemize}
\item our work already trusts Google with it's business data
\item Single repository potential to leverage that fact
\item Will likely be very cheap as a customer incentive. Currently it's around 700 pages per dollar.
\end{itemize}
\item Disadvantages
\begin{itemize}
\item GCHQ have \href{https://www.ncsc.gov.uk/blog-post/chatgpt-and-large-language-models-whats-the-risk}{taken the unusual step} of warning that the data put into these systems goes into their training and can thereby resurface in competitors searches later
\item The likes of Apple and Samsung have banned the use of these cloud tools as a result. There's anecdotal evidence of commercially sensitive details surfacing, though it's hard to validate these
\item The products can change over time, in ways that are outside of your control
\end{itemize}
\end{itemize}
With all that said there is potentially business advantage to learning how these systems work through doing.
\subsubsection{Build something custom self hosted}
Building something custom here means taking an open source model, with a permissive license. There's a lot of these now and they are `decent'. It's possible to add in some code engineering around the edges to give it access to private datasets through a chat interface.
\begin{itemize}
\item \textbf{Advantages}
\begin{itemize}
\item It's private, local, under your control, and so you can trust your data will be within the company walled garden
\item It's building toward IP and knowledge, in the likely scenario where GPT4 level models are less than 2 years away. This is real internal investment
\item There are NO legal repercussions to using it in a purely off-line way. You don't even need to tell people you're doing it. `Probably' no GPDR, data governance, compliance overheads if designed right.
\end{itemize}
\item \textbf{Disadvantages}
\begin{itemize}
\item It will still lie, and the company and individuals are still responsible for the legal repercussions of acting on what the model says.
\item In a live deployment to the public it will occasionally say ``bad things''. It's just impossible to control edge cases. Even without the issues incurred by being online, the exposure around that in terms of waivers is uncertain.
\item Making it fit for purpose by adding memory is hard. The token limits on these models are really small. They're just not that smart yet.
\end{itemize}
\end{itemize}
Below is output from the top two models from the \href{https://huggingface.co/spaces/HuggingFaceH4/open_llm_leaderboard}{global leader board}. The second highest rated worked better (supercot). Into it I uploaded the `International Code of Practice for Entertainment Rigging' which is a \href{https://www.plasa.org/wp-content/uploads/2017/11/ICOPER_V1.0.pdf}{meagre 35 page document}. I have done 500 pages in the past but the quality starts to break down with scale. Section 3.5 of the document says
\begin{tcolorbox}[enhanced, frame style={fill=lightgray}, interior style={fill=lightgray}]``All custom built equipment selected for a project must be reviewed and approved by a qualified person. Custom built equipment must be provided with necessary markings and documentation to ensure safe use.''
\end{tcolorbox}
You can see the output from the model running at home (without internet) in Figure \ref{fig:ICOPERLLM}. It has royally embellished the facts, but it's actually delivering up good advice, and the memory injection from the uploaded document means it's specific enough to the context of the question.
\begin{figure}[H]
    \centering
    \includegraphics[width=0.95\textwidth]{ICOPERLLM}
    \caption{Results from home hosted LLM}
    \label{fig:ICOPERLLM}
\end{figure}
I have very much seen this when playing with both my own "landmark attention" models, and Claude. I expect they will find a way round it at some point. Because humans write with a focus on the beginning and end of documents, so large language models pay far more attention at the beginning and end of their context input \cite{liu2023lost}. 

That's tricky because if  you load in a stack of PDFs and they get translated by the system into one big chunk (which may or may not be in the expected order of the PDFs), then the PDFs in the middle of the submission are subject to more hallucination, garbling, forgetting etc.

The "other" system, vector databases is more even handed, and so commensurately more predictable. Still got huge problems though. 

I have seen both effects. They're currently both bad enough that I wouldn't trust document interrogation to these things. There are workarounds obviously:
Ask for references, ask if to tell you if it thinks it's making it up, as for quotes, ask for locations, check the working, ask in many sperate ways etc.

These "tips" can be encoded into the preamble that goes into every query, as a standing command, so you don't have to, except for the checking bit of course.
You can do that with either self host models or the web ones, but it's something you need to do often with the web interfaces. There's no silver bullet yet but it feels like months away, so I am not advocating learning these tricks. Might be better to wait for the fixes. Just don't trust them, these are the reasons, and they are all basically seeded in the way humans write.

There's a lot that can be done here, but the cost benefit is unclear. If this were sensitive internal planning documents, with a lot of complexity, then there's potentially a strong case, but we'd need tight procedures to check on it's homework. This tool, as usual, is best as a way to rapidly construct a framework.\par 
As a side note there's a lot of open source tools like \href{https://github.com/TransformerOptimus/SuperAGI}{SuperAGI} for `Agents' and \href{https://github.com/Mintplex-Labs/anything-llm}{AnythingLLM} for document analysis which straddle the line between using cloud vendors and local hosting. Building on one of these gives optionality, but they are new and `flakey'.
\subsubsection{LoRA training}
\label{sec:LoRA}
LoRA stands for Low-Rank Adaptation and it's a cheap way to optimise models. A few hundred dollars of rented GPU time can nuance a model to be more performant for a specific task. It sounds great but basically you're optimising for a task that you need to understand very well, possibly/probably to the detriment of the generality of the tool. If there's something you know you want, then this is an option that's achievable and affordable.\par
Goat, a 7B LLaMA model finetuned for arithmetic tasks notably outperformed the ~75x larger 540B PaLM model and GPT-4. Goat's success can be attributed to two primary factors: task-specific finetuning and LLaMA's unique digit tokenization. The problem here is Llama is arguably derived from data with a non-commercial license. OpenLlama gets around this with their Apache2 (do what you like) copy of the model. The situation is rapidly evolving. Another example of task-specific finetuning is the Gorilla project, where the LLM was trained to generate API calls. This is a really important area and we might be able to get ahead in this niche. This controlling complex whole site systems with voice control. The model was finetuned using 1,645 API calls from various sources and demonstrated superior performance compared to non-finetuned models. We can easily repeat that.\par
Recent findings suggest that less LoRA training gets better results, so this is increasingly being adopted as a way to improve business fit \cite{xue2023repeat}.
\subsubsection{Enormous token limits}
\label{sec:llmoptions}
There's a few models now boasting staggering token input limits. With 1 million token windows it's possible that each query to the LLM can be `the whole corporate database and filesystem' and then the question you want an answer to. This is a beguiling option, probably the most performant (though I have not played with one), and also pretty much useless as nobody yet offers a legally trustable way to upload your whole business into a chat window.
\begin{itemize}
\item \textbf{Advantages}
\begin{itemize}
\item Likely excellent at it's job
\item Will `hallucinate' less
\item Probably stay on target
\item Knows all about the business
\end{itemize}
\item \textbf{Disadvantages}
\begin{itemize}
\item Web based right now, and knows all about the business
\item Not actually available, I'm on wait lists. 
\item Longer latency as everything will need uploading in the token window (assuming you don't buy a service)
\end{itemize}
\end{itemize}
Very recently this in beginning to shift with the \href{https://github.com/eugenepentland/landmark-attention-qlora}{emergence} of a locally run \href{}{`landmark attention' model}. There are some issues with it at time of writing but this could be run on an our work cloud instance. I have 4000 tokens running at home. I am exploring tree of thought which needs these bigger windows and elevates reasoning by 70 percent \cite{yao2023tree}.
\subsubsection{Memory and other systems}
There's many ways to use a database (in this case a `vector' DB) to prime the context window with appropriate information. It's also potentially useful to look at combining older and well understood ML techniques like knowledge graphs in combination \cite{pan2023unifying}. I've not personally had much luck with these yet.
\subsubsection{Cost implication for self build}
It's less than \$20k to build a system that can do anything you need, on site, rack mount in a ship, whatever. You can do it for less in the cloud, see the next section. You can also spend much, much more. Use-case dependent.\par
Here's the output from the `local' open source model for clarity. Note it drifts off into nonsense toward the end then stops. This can be dealt with but I chose not to:
\begin{tcolorbox}[enhanced, frame style={fill=lightgray}, interior style={fill=lightgray}]
ASSISTANT:Based on the provided information, here are some steps you can take regarding custom built equipment: 
1. Ensure that the equipment is properly marked and traceable to the manufacturer. 
2. Keep up-to-date with maintenance schedules and ensure that maintenance records are readily available. 
3. Follow safety procedures and guidelines for rigging work, including visual inspections, PPE usage, and clear communication. 
4. Prevent any malfunctioning or faulty equipment from being used and quarantine it if necessary. 
5. Use equipment with a capacity that meets or exceeds the design specifications in the lifting plan. 
6. Attach loads in a manner that does not damage the equipment or the load, and consider sling angle factors. 
7. Inspect electrical connections and repair or replace equipment as necessary. 
8. Maintain good housekeeping and prepare a post
\end{tcolorbox}

\subsubsection{Build something custom in a private cloud}
This is exactly the same as the previous section but you hire a private cloud system `on demand' to do the work. This is \href{https://lambdalabs.com/service/gpu-cloud/pricing}{currently priced at} \$1.10/hr and only costs you money when you're using it (though you have to shut it down yourself). This is both secure, and fairly cost effective. Also, it scales in that if you find a real serious application you can just get bigger rental GPUs and open a private/public interface. \textbf{It's my preferred path of all the private use cases except for mission critical on site stuff}, and edge cases like boats at sea etc. For that you need to buy GPUs. AMD have \href{https://www.amd.com/en/newsroom/press-releases/2023-6-13-amd-expands-leadership-data-center-portfolio-with-.html}{recently announced} a partnership with opensource behemoth Huggingface to allow access to large and capable models like Falcon in their enormous new memory architecture. Falcon is from the UAE and has odd views on human rights. This is one to watch.
\subsection{Whistlestop tour of terms}
\label{sec:aiterms}
\subsubsection{Transformers}
Machine learning transformers are a groundbreaking architecture in the field of natural language processing (NLP) that have redefined tasks such as text generation, translation, and sentiment analysis. \href{https://docs.google.com/presentation/d/1ZXFIhYczos679r70Yu8vV9uO6B1J0ztzeDxbnBxD1S0/edit#slide=id.g13dd67c5ab8_0_2648}{This link} is a very technical but excellent overview of how they work. Transformers have gained popularity due to their ability to efficiently capture long-range dependencies and model complex relationships between words in a sentence.\par 
At the core of the transformer architecture lies the self-attention mechanism. This mechanism allows the model to weigh the importance of each word in a sentence relative to the others, effectively capturing context and dependencies. In contrast to traditional neural networks, like recurrent neural networks (RNNs), transformers process input sequences in parallel, rather than sequentially. This parallel processing enables transformers to efficiently understand and remember long-range dependencies, which is particularly important in NLP tasks.\par
RNNs, on the other hand, process input sequences one element at a time, making it difficult for them to capture relationships between words that are far apart in a sentence. As a result, RNNs can struggle with tasks that involve complex sentences or require a deep understanding of context.\par
Transformer-based models, such as GPT (Generative Pre-trained Transformer) and BERT (Bidirectional Encoder Representations from Transformers), have become the go-to models for many NLP tasks.\par
The layers contain weight matrices that are responsible for encoding the model's knowledge and language understanding. \par 

\subsubsection{GANs}
Generative adversarial networks are used for generating synthetic data, and are incredibly useful for our fine tuning use cases. GANs consist of two neural networks that are trained to compete with each other, with one network generating synthetic data and the other trying to distinguish between the synthetic data and real data. This process allows GANs to learn the underlying distribution of the data and generate samples that are highly realistic.\par
Reinforcement learning is a type of machine learning that involves an agent learning through trial and error in order to maximize a reward. 
\subsubsection{LoRA}
LoRA, or Low-Rank Adaptation \cite{hu2021lora}, is a technique that enables efficient adaptation of large language models to specific tasks or domains while maintaining their expressive power. It does so by introducing a small modification to the pre-trained model's weight matrices, enabling the fine-tuning process to be more computationally efficient without sacrificing performance. Visually you can think of this as slipping modification layers in between the transformer layers, which are far more interdependent and thereby expensive to retrain. We are already experimenting with these systems for our use cases.
\subsubsection{Embeddings and Latent Space}
Embeddings play a crucial role in both generative AI art and large language models, as they provide a way to represent complex data types, such as text or images, in a continuous vector space. In both contexts, embeddings capture the underlying structure and semantics of the input data, enabling AI models to learn and generate new content based on these representations. This is what the user sees happening with both AI generative art, and LLMs.\par 
In the context of generative AI art, embeddings are often used to represent visual elements, such as images or shapes. Latent space is a continuous vector space where each point corresponds to an embedding that encodes the features and semantics of an image. Once the AI model is trained, new images can be generated by sampling points from this latent space and decoding them back into the image domain. Embeddings can similarly represent styles or artistic techniques. Style transfer techniques, for example, utilize embeddings to extract and apply the style of one image to the content of another.\par 
In the context of large language models, embeddings are used to represent words, phrases, or sentences in a continuous vector space. These models are trained on vast amounts of text data, learning to generate contextually relevant and semantically meaningful embeddings for language. Similarly to the visual use case in generative art these embeddings capture various aspects of language, such as syntax, semantics, and relationships between words or phrases. Once trained, the large language models can generate new text by sampling from the distribution of embeddings and decoding them back into the text domain.\par 
Embeddings are also used in tasks like sentiment analysis, machine translation, and text classification, where the AI model must understand the meaning and context of the input text.
\subsubsection{Vector databases}
One of the highest points of human `friction' when dealing with and AI model, and especially LLMs is the lack of a persistent and/or contextual memory within the systems. This is beginning to be addressed using vector databases.
A vector database is designed to efficiently store, manage, and query the high-dimensional vectors, often used in the context of machine learning and artificial intelligence. These high-dimensional vectors are the embeddings previously discussed.\par 
Using a vector database with embeddings for AI data retrieval and processing can significantly improve efficiency, scalability, and performance. In the context of a stored item of data, embeddings allow the storage of complex `concepts' as fixed-length vector, which interacts with the enormous latent space in the trained model. This makes storage and retrieval more efficient. \par 
Vector databases allow and optimise for efficient nearest neighbor search, which is crucial for data retrieval tasks in AI applications.
To do this, given a query input, the AI system first converts the input into an embedding using the same technique as for the stored data. The vector database then performs a nearest neighbour search to find the most similar embeddings in the database. At scale this can result in more consistency when using models, but crucially it doesn't train the models on events that have happened. It is not a `memory'.
\subsubsection{Memory Streams}
In the paper `Generative Agents: Interactive Simulacra of Human Behavior' Park et al present a solution and working example for the problem of contextual memory in AI systems \cite{park2023generative}. This is a pretty stunning paper for our purposes in collaborative XR where we would hope to interact with virtual agents. \par
As they point out in the paper virtual agents should be able to manage constantly-growing memories and handle cascading social dynamics that unfold between multiple agents. Their architecture uses a large language model to generate a memory stream, reflection, and planning. The memory stream contains a comprehensive list of the agent's experiences (written as a kind of internal monologue), and the planning module synthesizes higher-level inferences over time. These memory transcriptions are highly compressible and would be excellent as RGB style private data blobs between our federated virtual worlds. It will therefore me possible to `meet' virtual agent friends across instances of virtual spaces \textbf{and} through nostr social media. This is a key technology for our uses now.
\subsubsection{Gradient descent}
Gradient descent is an optimization algorithm widely used in machine learning and deep learning, including large language models, to minimize a loss function. The loss function measures the difference between the predicted output and the actual output (also known as the target) for a given input. The goal of the training process is to \href{https://societyofai.medium.com/gradient-descent-basics-and-application-1cef98179ee6}{minimize this loss} to improve the model's performance.\par
In the context of large language models, gradient descent helps to adjust the model's parameters (weights and biases) so that it can generate more accurate predictions. These models consist of multiple layers of neurons with a large number of parameters that need to be fine-tuned. It is this training process that takes so much time and energy.
\begin{itemize}
\item Initialize parameters: The model's parameters are initially set to random values. These parameters are then iteratively adjusted using gradient descent.
\item Calculate loss: For a given input and target, the model generates a prediction, and the loss function calculates the difference between the prediction and the target.
\item Compute gradients: The gradients of the loss function with respect to each parameter are computed. A gradient is a vector that points in the direction of the greatest increase of the function, and its components are the partial derivatives of the function with respect to each parameter. The gradients indicate how much each parameter contributes to the loss.
\item Update parameters: The model's parameters are updated using the gradients. This is done by subtracting a fraction of the gradient from the current parameter value. The fraction is determined by a hyperparameter called the learning rate. A smaller learning rate results in smaller updates and slower convergence, while a larger learning rate can result in faster convergence but might overshoot the optimal values.
\item Iterate: Steps 2-4 are repeated for a certain number of iterations, a specified tolerance, or until convergence is reached (i.e., when the change in the loss function becomes negligible).
\item Gradient descent has several variations, such as Stochastic Gradient Descent (SGD) and mini-batch gradient descent. These methods differ in how they use the training data to compute the gradients and update the parameters. In SGD, the gradients are computed and the parameters are updated using only one data point at a time, while in mini-batch gradient descent, a small batch of data points is used to compute the gradients and update the parameters. LLMs like GPT can use meta optimisers to train as they operate \cite{dai2022can}.
\end{itemize}
\subsubsection{TPUs}
Tensor Processing Units (TPUs) are specialized hardware accelerators for machine learning workloads, developed by Google. TPUs are designed to speed up the training and inference of machine learning models, particularly large deep neural networks. They are highly parallel and optimized for low-precision arithmetic, which allows them to perform computations much faster than traditional CPUs or GPUs. TPUs can be used in a variety of machine learning applications, such as natural language processing, computer vision, and speech recognition. Google has integrated TPUs into its cloud platform, allowing developers to easily use them for their machine learning workloads. Overall, TPUs provide a powerful and efficient platform for machine learning.
The top of the line Nvidia tensorflow unit at this time is the v4, and it is comparable if more generalised hardware.
\subsubsection{Tensorflow}
TensorFlow is a popular open-source machine learning framework developed by Google and was instrumental in kicking off a lot of this research area. It is still widely used for training and deploying machine learning models in a variety of applications, such as natural language processing, computer vision, and speech recognition, but is being somewhat superceded by JAX. The consensus seems to be that JAX itself is more specialised and harder to use, but works well with Googles hardware cloud systems. Time will tell if this upgrade gets community traction. TensorFlow provides a flexible and high-performance platform for building and deploying machine learning models. It allows users to define, train, and evaluate models using a variety of deep learning algorithms, such as convolutional neural networks and recurrent neural networks. TensorFlow also has a strong emphasis on scalability and performance, with support for distributed training and deployment on a variety of platforms, including GPUs and TPUs. Overall, TensorFlow is a powerful tool for building and deploying machine learning models.
\subsubsection{PyTorch}
PyTorch is a popular open-source machine learning framework developed by Facebook's AI research group. It is primarily used for applications such as natural language processing and computer vision. PyTorch is based on the Torch library and provides two high-level features: tensor computations with strong GPU acceleration and deep neural networks built on a tape-based autograd system. PyTorch offers a variety of tools and libraries for machine learning, including support for computer vision, natural language processing, and generative models. It also allows for easy and seamless interaction with the rest of the Python ecosystem, including popular data science and machine learning libraries such as NumPy and scikit-learn. 
\subsubsection{NumPy}
NumPy is a popular open-source library for scientific computing in Python. It provides a high-performance multidimensional array object, as well as tools for working with these arrays. NumPy's array class is called ndarray, which is a flexible container for large datasets that can be processed efficiently. The library provides a wide range of mathematical functions that can operate on these arrays, including linear algebra operations, Fourier transforms, and random number generation. NumPy also has a powerful mechanism for integrating C, C++, and Fortran code, which allows it to be used for high-performance scientific computing in a variety of applications. Overall, NumPy is an essential library for working with numerical data in Python.
\subsubsection{Latent space}
In the context of generative artificial intelligence (AI), a latent space is a high-dimensional space in which the model represents data as points. This space is "latent" because it is not directly observed, but is inferred by the model based on the data it is trained on. In the case of a generative model, the latent space is often used to encode the underlying structure of the data, such that samples can be generated by sampling from the latent space and then decoding them into the data space.

For example, in a generative model for images, the latent space may encode the features or characteristics of the image, such as the shape, color, and texture. By sampling from this latent space and decoding the sample, the model can generate new images that are similar to the training data, but are not exact copies. This allows the model to generate novel and diverse samples that capture the essence of the training data.

The latent space is an important aspect of generative models because it allows the model to capture the underlying structure of the data in a compact and efficient way. It also provides a way to control the generation process, such as by interpolating between latent space points to generate smooth transitions between samples. At this time the navigation through that mathematical space is steered by vectors into the space, which come from a separate and parallel integration of a natural language model. This crucial bridge came from research at OpenAI, and has been instrumental in the current explosion of usability of the systems \cite{radford2021learning}.
\subsubsection{Edge AI compute and APUs}
\begin{itemize}
\item \href{https://www.theverge.com/2023/2/23/23611668/ai-image-stable-diffusion-mobile-android-qualcomm-fastest}{Qualcomm phone chip} offers low power and high speed Stable Diffusion on mobiles
\item IBM have introduced the \href{https://research.ibm.com/blog/ibm-artificial-intelligence-unit-aiu}{concept of the AIU}, for high speed and low power training
\item Nvidia's \href{https://www.okdo.com/p/nvidia-jetson-agx-orin-64gb-developer-kit/}{latest in the Jetson} Edge AGX line is a high performance general AI unit for industrial applications
\item Esperanto Risc V chip \href{https://www.esperanto.ai/News/risc-v-startup-esperanto-technologies-samples-first-ai-silicon/}{claims incredible performance} gains
\item The MetaVRain asic \href{https://hdh4797.wixsite.com/dhan/project-1}{claims 900x speed increases} on general GPU problems
\item Microsoft are rumoured to be looking to mitigate the staggering costs of running ChatGPT (\$1M/day) using forthcoming \href{https://www.theinformation.com/articles/microsoft-readies-ai-chip-as-machine-learning-costs-surge?}{hardware of their own design}
\item \href{https://www.cerebras.net/}{Cerebras systems} have built an AI architecture from the ground up and claim incredible numbers.
\end{itemize}
 These systems will drive the compute to less `constrained' but somewhat less capable AI systems, distributing the access but increasing risks.
\subsubsection{Prompt engineering}
The art of prompting constrains the generation space into documents that contain correct answers.

A better performance in few-shot prompting can be achieved by constructing fictional scenarios.

Using flattery or painting a clear fictional narrative, such as referring to the LLM as a "masterful French translator," can help to guide the model towards producing the desired output.

Much of the initial prompt engineering involves constructing scenarios that can only be completed correctly.

A good strategy often involves imagining what kind of document might contain the correct answer.

Fine-tuning models for tasks helps make the model more capable and more able to do as it's told.

Fine-tuning involves giving many examples of a task being done correctly, resulting in a model that acts almost as though it had been prompted by thousands of correct examples.
Use of assisted learning: Working with assistants or partners on the project can help identify problems and improve the model.

Correct absurdity instead of playing along: Even if the question seems absurd, the model should provide an answer grounded in reality instead of going along with the absurdity.

Satisfaction of preference model: The model should aim to fulfill a preference model emulating what a human would want.

Consciousness of self: The model should be conscious in a sense of what it is, what it's doing, and where it's situated in the world.

Avoidance of interpolation: Instead of just interpolating what humans might do, the model should do something other than that.

Use of instruction tuning: This makes the desired outcome of the model clearer.

Reinforcement learning with a reward model: This technique takes the results to another level and provides feedback to improve the model.

Avoidance of mode collapse: The model should avoid fixating on a particular way of answering.


\subsection{Consumer tools}
Gozalo-Brizuela and Garrido-Merchan provide a helpful review and taxonomy of recent generative ML systems in their paper `ChatGPT is not all you need' \cite{gozalo2023chatgpt}, with Figure \ref{fig:MLtaxonomy} showing some of the main categories and systems. 

\begin{figure}
  \centering
    \includegraphics{mltaxonomy}
  \caption{Taxonomy of \href{https://arxiv.org/abs/2301.04655}{recent generative ML systems} by Gozalo-Brizuela and Garrido-Merchan used with permission.}
  \label{fig:MLtaxonomy}
\end{figure}
\subsubsection{ChatGPT}
ChatGPT is a neural network-based natural language processing (NLP) model developed by OpenAI. It is a continuation of the OpenAI programme which binds iteratively more capable Generative Pre-trained Transformer models to a web and API based text chat interface. It uses self-attention mechanisms to generate high-quality text in a variety of different languages. ChatGPT is specifically designed for conversational text generation, and has been trained on a large corpus of dialogue data in order to produce responses that are natural, diverse, and relevant to a given conversation. Because it is a large language model, ChatGPT has a vast amount of knowledge and can generate responses to a wide range of questions and prompts and meta-prompts \cite{hou2022metaprompting}. This allows it to generate responses that are relevant, natural-sounding, and diverse in nature. It has proved incredibly popular, demonstrating uncanny abilities for natural conversation, code generation, copy writing and more. It is substantially flawed in that it `speaks' with authority but often makes things up completely. This extended recently to creating academic references to back it's assertions, completely out of thin air. The interface and APIs seem to be evolving and improving in real time.\par
The model uses a transformer-based architecture, which means that it consists of a series of interconnected ``blocks'' that process the input data and generate the output text. Each block contains multiple self-attention mechanisms, which allow the model to focus on different aspects of the input data and generate a response that is coherent and relevant to the conversation. In addition to its transformer-based architecture, ChatGPT also uses a variety of other techniques to improve its performance. For example, it uses beam search to generate multiple candidate responses for each input, and then selects the best one based on a combination of factors such as relevance, coherence, and diversity. This allows the model to generate high-quality responses that are appropriate for a given conversation. Additionally, ChatGPT uses a technique called ``response conditioning'' to bias the model towards generating responses that are appropriate for a given conversation context. This allows the model to generate more relevant and coherent responses, even when faced with challenging input data. Microsoft have  \href{https://medium.com/@owenyin/scoop-oh-the-things-youll-do-with-bing-s-chatgpt-62b42d8d7198}{integrated GPT4} with Bing, their internet search engine, and plugins for other websites are coming soon.\par 
One key aspect of GPT-4 is reinforcement learning with human feedback (RLHF), which helps align the model with human preferences, making it more useful and easier to interact with. The process involves using human feedback to fine-tune the model, requiring relatively less data compared to the initial training phase. The development of GPT-4 involves multiple components: the design of neural network algorithms, data selection, and human supervision through RLHF. Researchers have made significant progress in understanding the behavior of the fully trained system using evaluation processes, although the complete understanding of the model remains a challenge. GPT-4 has the ability to perform `reasoning' based on the knowledge it has gained from the training data. While some interactions with the model may display wisdom, others might lack it. The dialog format used in the model enables it to answer follow-up questions, admit mistakes, challenge incorrect premises, and reject inappropriate requests. In a recent report, researchers revealed groundbreaking advancements in GPT-4, hinting at sparks of artificial general intelligence \cite{bubeck2023sparks}. The Microsoft researchers had access to the unrestrained GPT-4 during its early development, allowing them to experiment for around six months. 
\begin{itemize}
\item GPT-4 can use tools with minimal instruction, displaying an emergent capability to utilize calculators, character APIs, and text-to-image rendering.
\item  GPT-4 can pass mock technical interviews on LeetCode and could potentially be hired as a software engineer.
\item When tasked with creating a complex 3D game, GPT-4 produces a working game in a zero-shot fashion.
\item GPT-4 can tackle the 2022 International Mathematics Olympiad, demonstrating a high level of mathematical ability.
\item It can answer Fermi questions, which are complex questions often used in difficult interviews.
\item GPT-4 can serve as an AI personal assistant, coordinating with others over email, booking events, and managing calendars.
\item It can help diagnose and solve real-life problems, such as fixing a leak in a bathroom.
\item GPT-4 can build mental maps of physical locations, which may be useful when embodied.
\item It displays a theory of mind, capable of understanding what others may be thinking or believing about a situation.
\end{itemize}
There are obviously still limitations to GPT-4. As an autoregressive model, it cannot plan ahead and struggles with discontinuous tasks. This issue could potentially be addressed by augmenting language models with external memory. The paper raises concerns about the unrestricted GPT-4's ability to create propaganda and conspiracy theories, and the ethical implications of giving AI intrinsic motivation. The researchers call for a better understanding of AI systems like GPT-4 to address these challenges.\par
When asked about controversial figures or topics, GPT-4 can provide nuanced and balanced answers, highlighting its potential to reintroduce nuance to conversations. It is important to consider AI safety and alignment when developing powerful models like GPT-4. After its completion, the model underwent extensive internal and external testing, including red teaming and safety evaluations, to ensure alignment with human values. To ensure that AI systems align with various human values, it is necessary to establish broad societal boundaries, which may differ across countries and individual users. The art of crafting effective prompts for GPT-4 involves understanding the model's behavior and composing prompts in a way that elicits the desired response. This process is similar to human conversation, where individuals adapt their phrasing to communicate more effectively. As the model becomes more advanced, it may increasingly resemble human interactions, which can offer insights into human communication and behaviour. \par 
Although impressive, it is generally agreed that GPT-4 is not an AGI due to its limitations in approximating human-level intelligence. The concept of consciousness in AI is debated, with some believing AI can be conscious, while others argue that AI can convincingly fake consciousness without being truly aware. Potential risks associated with even a `simply' and unconcious AI include disinformation, economic shocks, and geopolitical impacts. These concerns do not necessarily require superintelligence but could result from large-scale deployment of powerful language models without proper safety controls.
\subsubsection{GPT API and programming}
In the context of programming, GPT-4's advancements may have a significant impact on how developers interact with AI systems,, and their productivity and creativity. The impact of AI systems like GPT-4 on programming is significant, changing the way programmers work by allowing an iterative process and back-and-forth dialogue with the AI as a creative partner. This development is seen as a major step forward in programming.\par 
It seems almost inevitable that at some stage a large language model will be optimised for computer instruction sets,   and simply be able to bridge directly from human intentionality to bytecode, running it's own tests and refinements without external consultation. The degree to which this would still be considered `a compiler' is unclear.

\subsubsection{ChatGPT - Advanced Data Analysis}
The ChatGPT Code Interpreter Plugin, introduced in March 2023, offers a sandboxed environment featuring a working Python interpreter. This environment, which is isolated from other users and the Internet, supports an impressive array of functionalities. It comes pre-loaded with over 330 libraries, including popular ones such as pandas, matplotlib, seaborn, and TensorFlow, among others.

As illustrated in Figure \ref{fig:chatGPTdata}, the Code Interpreter Plugin is capable of performing a myriad of tasks. For example, it can visualize any data inputted by the user, generate GIFs of the visualizations, and perform file uploads and downloads. It can extract colors from an image to create a color palette, and autonomously compress large images when memory is running low. Moreover, the plugin can clean and process data, generate insightful visualizations, and convert files to different formats quickly and efficiently.

The Code Interpreter Plugin can be installed by ChatGPT Plus users in a few simple steps. However, it is worth noting that while this plugin is powerful, it does have certain limitations, such as frequent environment resets, limited Optical Character Recognition (OCR) capabilities, and an inability to access the web. Despite these limitations, OpenAI continues to work on improving the capabilities of the Code Interpreter Plugin, promising a future with substantial impacts on the world of programming.

\begin{itemize}
\item The Code Interpreter Plugin introduces a sandbox and an advanced language model, both of which are critical to its functionality.
\item The emphasis of the plugin is on the quality of the model, which can generate code, debug it, and even decide when not to proceed without human input.
\item The plugin offers substantial model autonomy, enabling it to work through multiple steps of code generation autonomously.
\item Despite its powerful capabilities, the plugin does have limitations, such as frequent environment resets and limited OCR capabilities.
\item The plugin is only available to ChatGPT Plus users, and requires a few simple steps for installation.
\item The Code Interpreter Plugin represents a significant advancement in the realm of programming, changing the way programmers interact with AI systems.
\end{itemize}

\subsubsection{CodeLlama}
Meta is now preparing to launch CodeLama, an open source code generating model to rival OpenAI's Codex and Microsoft's GitHub Copilot. This could disrupt the industry.
\subsubsection{Google Gemini}

\subsubsection{Current LLM Capabilities}

\begin{itemize}
\item it{GPT-4} - Best for difficult reasoning tasks based on comparisons with Claude 2. Most advanced reasoning of current LLMs.

\item it{Claude 2} - 100k token context window allows ingesting entire documents and books. Useful for QA and summarization with large knowledge bases.

\item it{Bard} - Integrated with web like Google Search. New multimodal features understand images. 40 language support.

\item it{Codex} - Unlocks new abilities like running and refining code iteratively. Suspected separate model accessed via API.

\item it{Pi} - Focused on being a friendly personal assistant. Asks followup questions to continue conversations.

\item it{LLaMA 2} - Open source alternative to GPT-4 from Meta. Could pose threat as safer open LLM.

\item it{Personal LLMs} - Allow customization with your own data. Create personalized AI assistants.
\end{itemize}

This is the list of libraries that code interpreter can call.

\input{09_code_interpreter_libs}


\begin{figure}
  \centering
    \includegraphics{chatGPTdata}
  \caption{A multi-model conversation with chatGPT4 `code interpreter plugin' by \href{https://www.oneusefulthing.org/p/it-is-starting-to-get-strange}{Mollick}}
  \label{fig:chatGPTdata}
\end{figure}


\subsection{Researcher toolkits}
\subsubsection{COG containers for ML}
Cog is an \href{https://github.com/replicate/cog}{open-source tool} for packaging machine learning models into production-ready containers. It simplifies Docker container creation, resolves compatibility issues between CUDA, cuDNN, PyTorch, TensorFlow, and Python, and uses standard Python to define model inputs and outputs. Cog generates an OpenAPI schema, validates inputs and outputs with Pydantic, creates an automatic HTTP prediction server using FastAPI, and offers automatic queue worker functionality. It supports cloud storage with Amazon S3 and Google Cloud Storage (coming soon), and allows model deployment on any infrastructure that supports Docker images, including Replicate.
\subsection{Enterprise and convergence}
Startups are clearly eager to create innovative products and business models, while established companies are exploring ways to respond to the rapid advancements in generative AI. There seems to be a sense of urgency for enterprises worldwide to develop AI strategies. Into the space Nvidia AI have emerged as the clear market enabler, offering more accessible and faster infrastructure.\par
Their flagship product, the Nvidia DGX H100 is in full production and available through partners like Microsoft Azure. They're also launching Nvidia DGX Cloud in collaboration with Microsoft Azure, Google GCP, and Oracle OCI, making AI supercomputers accessible from a browser. Nvidia AI Foundations might be a suitable solution as a cloud service that includes:
\begin{itemize}
\item Language, visual, and biology model-making services
\item Nvidia NeMo for building custom language models
\item Picasso, a visual language model-making service for custom models trained with licensed or proprietary content
\end{itemize} 
Nvidia Picasso could transform how visual content is created by allowing enterprises, ISVs, and service providers to deploy their own models. This might enable the generation of photorealistic images, high-resolution videos, and detailed 3D geometry for various applications, so it's certainly something to watch closely. Our alignment to self hosted and open source pipelines makes this less of a priority for exploration however.\par 
Companies like Getty Images and Shutterstock intend to use Picasso for building generative models with their extensive libraries. Nvidia say they will also expand its partnership with Adobe to integrate generative AI into creative workflows, focusing on commercial viability and proper content attribution.\par
In the field of biology, Nvidia's Clara could be a healthcare application framework for imaging instruments, genomics, and drug discovery. Their Bio NeMo might help researchers create fine-tuned custom models with proprietary data. Nvidia BioNeMo service could provide generative AI models for drug discovery as a cloud service for easy access to accelerated workflows.\par
As discussed Nvidia still hope that their enterprise metaverse offering Onmiverse will gain worldwide traction. They are investing heavily in bringing ML into this product line. This is an incrediby similar proposition to flossvers, our proposal in this book, but operating at a different level of investment and technology, with commensurately more gated access. \par
Nvidia's partnerships with TSMC, ASML, and Synopsys could lead to advancements in chips and efficiency. Grace, Grace Hopper, and Bluefield 3 are designed for energy-efficient accelerated data centers.\par
Microsoft's investment in OpenAI and especially ChatGPT is clearly paying dividends right now, but it is possible that they are mindful of the transition to less centrally controlled `edge' hardware as previously mentioned has forced their hand toward offering their generalised systems for use by corporations as plugins. It's not clear at all that this is the correct model. With that said the current response from Microsoft it yielding incredible powerful results. Their plugin `code interpreter' allows uploading of data files of up to 100MBin size, and both writing and execution of Python code, one of the languages of data analystics. This toolkit is the first major leveraging of a unified multi-modal model. It can create media rich documents with charts, images, and diagrams, providing appropriate descriptive analysis and diagnostics of the statistics employed. This is said to be happening at the level of a junior data analytics specialist so far, and could represent the beginning of a distributive democratisation of data science Figure \ref{fig:chatGPTdata} and Figure \ref{fig:chatGPTword}.


\begin{figure}
  \centering
    \includegraphics{mltaxonomy}
  \caption{The planned integration of these tools with Office Suite is likely to be a historic moment}
  \label{fig:chatGPTword}
\end{figure}


\subsection{Accessibility}
\subsubsection{Open source LLM chat and assistants}
Sheng at el present FlexGen which allows execution of large language model chat bots in powerful but affordable hardware\cite{Sheng2023}. The paper presents FlexGen, a high-throughput generation engine for large language models (LLMs) that can be run with a single commodity GPU. FlexGen can be configured under various hardware resource constraints by aggregating memory and computation from the GPU, CPU, and disk, and it uses a linear programming optimizer to store and access tensors. FlexGen compresses weights and attention key/value cache to 4 bits with negligible accuracy loss, allowing for a larger batch size and increased throughput. When running OPT-175B on a single 16GB GPU. A PC running alongside our metaverse server could provide ML assistance services to users of the collaborative space immediately. We are currently using \href{https://huggingface.co/Pi3141/alpaca-lora-30B-ggml/tree/main}{Alpaca Llama 4-bit quantised models}.\par 
\bigskip
\href{https://www.semianalysis.com/p/google-we-have-no-moat-and-neither}{This leak} purporting to be from a Google employee rings very true against the research we have done (paraphrased highlights):
\begin{itemize}
\item Open-source models are outpacing Google and OpenAI in terms of development speed and capabilities.
\item Examples of open-source achievements include LLMs on a phone, scalable personal AI, responsible release, and multimodal advances.
\item Google's models have a slight edge in quality, but open-source models are faster, more customizable, more private, and overall more capable.
\item Google has no secret sauce and should consider collaborating with the open-source community and enabling third-party integrations.
\item Large models might slow down progress; smaller, faster models should be prioritized for quicker iteration.
\item Meta's LLaMA was leaked and sparked an outpouring of innovation in the open-source community, lowering the barrier to entry for training and experimentation.
\item LoRA, an inexpensive fine-tuning method, has been underexploited within Google and should be paid more attention to.
\item Retraining models from scratch is expensive and time-consuming; using LoRA allows for stackable improvements that can be kept up to date more easily.
\item Large models may not be advantageous in the long run compared to rapid iteration on small models.
\item Data quality scales better than data size; high-quality, curated datasets are becoming the standard in open-source training.
\item Competing with open source is a losing proposition; Google should consider working with them instead.
\item Owning the ecosystem, as Meta does, allows them to benefit from free labor and innovation, which Google could adopt by cooperating with the open-source community.
\end{itemize}
\subsubsection{Real time transcription}
Real-time language translation can be applied to text interfaces within metaverse applications. This can be useful in situations where users are typing or reading text, rather than speaking.\par
To apply NMT to text interfaces in the metaverse, the algorithm can be integrated into the interface itself. When a user types text in a specific language, the NMT algorithm can automatically detect the language and generate a translation in the desired language. This can be done in real-time, allowing for fast and seamless communication between users speaking different languages. NMT algorithms are well-suited for use in text interfaces, allowing for fast and accurate translations between multiple languages. As the technology continues to advance, we can expect to see more and more applications of NMT in the metaverse.
\subsubsection{Real time translation}
One of its most impressive recent applications is real-time language translation. In this section we will explore how this technology works, and how it can be used in metaverse applications.\par
Real-time language translation refers to the ability of a machine learning model to instantly translate spoken or written text from one language to another. This is different from traditional translation methods, which often involve human translators and can be slow and error-prone.\par
One of the key technologies behind real-time language translation is neural machine translation (NMT). This is a type of machine learning algorithm that is based on neural networks. NMT algorithms are trained on large datasets of text that has been translated by human experts. This allows the algorithm to learn the patterns and nuances of each language, which it can then use to generate accurate translations.\par
One of the key references for the use of neural machine translation in real-time language translation is the paper "Neural Machine Translation by Jointly Learning to Align and Translate" by Bahdanau et al \cite{bahdanau2014neural}. This paper describes the use of a neural network-based approach to machine translation, which has shown impressive results in terms of accuracy and speed.\par
One of the key advantages of NMT is its ability to handle complex and varied sentences. Traditional translation algorithms often rely on fixed rules and dictionaries, which can be limiting. NMT algorithms, on the other hand, can learn to handle a wide range of sentence structures and vocabulary. This makes them well-suited for translating natural languages, which are often full of irregularities and exceptions.\par
Another advantage of NMT is its ability to handle multiple languages at once. Traditional translation algorithms often require the user to specify the source and target languages, but NMT algorithms can automatically detect the languages of the input and output text. This makes them well-suited for use in metaverse applications, where users may be speaking different languages at the same time.\par
One of the challenges of using NMT in metaverse applications is the need for real-time performance. Metaverse applications often involve fast-paced interactions, and any delay in language translation can hinder the user experience. To overcome this challenge, NMT algorithms can be optimized for speed, using techniques such as parallel processing and batching. It seems likely that in our proposed systems we will require API calls to external services for this functionality, and this will almost certainly incur a cost.\par
The use of NMT in metaverse applications is also an active area of research, with a number of papers exploring the potential of this technology. For example, the paper "Real-Time Neural Machine Translation for Virtual Reality" by Chen et al. describes the use of NMT algorithms in virtual reality environments, showing how they can be used to support real-time communication between users speaking different languages.\par
Overall, the use of machine learning for real-time language translation is a rapidly-evolving field, with many exciting developments and applications. As the technology continues to advance, we can expect to see even more impressive results and applications in the future.
\href{https://openai.com/blog/whisper/}{OpenAI whisper}

\subsubsection{Real time description}
\subsubsection{Interfaces}
\href{https://tech.fb.com/ar-vr/2021/03/inside-facebook-reality-labs-wrist-based-interaction-for-the-next-computing-platform/}{emg}
\subsubsection{Text to sound}
Complex acoustic environments are possible using \href{https://anonymous.4open.science/w/iclr2023_samples-CB68/report.html}{text to sound} prompting. 

\begin{figure}
  \centering
    \includegraphics{sdtools}
  \caption{SD tools website shows elements of creation and training.}
  	\label{fig:SDTools}
\end{figure}
\subsection{Virtual humans}
\subsubsection{Real time human to avatar mapping}
\subsection{AI actors}
This is the next major section to be written.
\subsection{Chatbots}
We are using Flexgen \cite{Sheng2023} on local hardware with various large language models. Response time is over a minute and the accuracy of the results is poor, but we are excited that it runs at all.

\begin{figure}
  \centering
    \includegraphics{aitechstack}
  \caption{A16Z view the nominal deployment of and AI tech stack in this way, but we are not using any of these models.}
  	\label{fig:aitechstack}
\end{figure}


\subsubsection{Faces}
Faces and their corresponding personae are already paired in the the Tavern AI ecosystem, encoding the metadata for the AI character into the PNG files. Obviously these could be inscribed and sold as Bitcoin ordinals. It would be a nice touch to encode the personality for the characters into a larger, high resolution file using image steganography \cite{morkel2005overview}, which would allow PKI type ownership too. This would be more suitable for our RGB use case. 
\subsubsection{Voices}
\subsubsection{Autonomous tasks (AutoGPT \& SaSa}

Autonomous General Purpose Language Models (AutoGPTs) are tools that can perform any task, leveraging connection to the internet and LLMs. Recently there has been a shift in the AutoGPT space towards specialized agents that cater to specific tasks or industries, providing more focused and useful solutions. This represents part of a more general move away from centralised general models toward more task specific systems.\par
Autonomous agents for research for instance search the internet for information relevant to a specific research topic and extract information from trustworthy sources. One example is [insert], an agent designed explicitly for research purposes. Similarly, medical research agents can call medical APIs and cite sources, providing targeted assistance in the medical field.\par
These agents can leverage a chain of GPT calls and fine-tuned models to perform tasks efficiently and effectively.\par
It has been suggested that such systems be dubbed semi-autonomous specialized agents (SASAs). These agents can streamline processes, automating multiple steps without requiring user mediation for each task. It can be seen that these are already being integrated with messenger services such as Telegram bots, very similarly to the planning and approach seen in this book.
We propose:\par

Extrinsic AI actors which link multiple\\ intrinsic virtual spaces.\\
Bespoke news and current affairs synthesis\\
Bespoke interactive subject matter training\\
bots that bring you what you want as bespoke audio visual packages
\subsection{Governance and safeguarding}
\subsubsection{Governance in the Virtual Reality Space}
The governance of the virtual world will be a critical element in the success of the Metaverse. The virtual world will need to be policed and governed in a way that will not only protect the rights of the citizens of this new digital environment but also protect them from cybercrime. As a somewhat strained but interesting example; \href{https://www.reuters.com/technology/interpol-says-metaverse-opens-up-new-world-cybercrime-2022-10-27/}{Interpol see} simulated environments as a way for terrorist groups to gather and plan attack. Governments and regulatory bodies will play a key role in the governance of the virtual world, but so will the industry and businesses. Nair et al describe the ``unprecedented privacy risks'' of the metaverse, finding that wearing a headset can currently reveal 25 data points about the user, simply by analysis of the data \cite{nair2022exploring, Nair2023}.  This included inference about ethnicity, disability, and economic status. Strong data protection laws will be needed to safeguard privacy, data ownership and reduce the risk of data breaches. The governance of the virtual world will be critical to success, safeguarding will be needed to protect citizens from cyberattacks.
\subsubsection{Safeguarding in the Metaverse}
When it comes to safeguarding in the Metaverse, people need to be made aware of the risk of using VR technology. There are still many questions around the health implications of using VR and the impact it may have on a person’s eyesight. In terms of safeguarding in the Metaverse, this is just one area that needs to be addressed. Users will also need to be made aware of the risks of hacking. Users will need to be educated on the need to be careful when it comes to sharing personal information and be careful what websites they access on a virtual computer. They will need to be made aware of the potential risk of having malware installed on their computer by visiting untrusted websites. Users will also need to be made aware of the potential risk of being manipulated in the virtual world. This risk is particularly high when it comes to children who are growing up in the digital world. They will need to be educated on the potential risks of being groomed or manipulated in the Metaverse.\\

The problem with large social metaverse systems seems to be somehow wrapped up in humans need to test boundary conditions in novel surroundings:
\begin{itemize}
\item Despite `best efforts' by the software vendors  there is a chaotic mix of levels of maturity amongst the participants. Ostensibly safe games are themselves `gamed' by \href{https://futurism.com/mom-horrified-her-kids-seeing-roblox}{slightly older} players.
\item No recording of action, and reaction, creating a feeling of impunity of action. At it's best `The philosophers Island', but in safeguarding terms it seems more a school yard without a teacher, or perhaps worse, Lord of the Flies \cite{cameron2012splendid}.
\item Even adults in exclusively adult meeting places seem to go slightly off the rails trying to find technical and social boundaries instinctively. This leads to the now somewhat famous (TTP) ``time to penis'' problem \cite{lamb2022second} (\href{http://gamedesignreviews.com/reviews/little-big-planet-browsing-content/}{coined at GDC 2009}). 
\item The research on this is pretty thin.
\item People seem to be suffering genuine psychological harm.
\end{itemize}
\href{https://www.immersivewire.com/p/harassment-metaverse-how-address}{Article in immersive wire}
\subsubsection{How to fight against cybercrime in the Metaverse?}
The best way to fight against cybercrime in the Metaverse is to educate the general public on the potential risks and dangers in order to prevent them from being targeted. This can be done through various channels and mediums, such as social media, blogs and podcasts. People will need to be made aware of the risks of opening emails or clicking on links sent by unknown people. They will also need to be aware of the risks of clicking on ads and links that may lead them to websites that host malware or that steal personal information.

\href{https://www.whitehouse.gov/ostp/ai-bill-of-rights/}{AI bill of rights}\\

Roblox \href{https://www.bbc.co.uk/news/technology-48450604}{in BBC news} for child exploitation.

\subsection{The emergent role of AI in education}

