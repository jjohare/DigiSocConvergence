\section{Application case studies}
As we have seen in the `collaborative mixed reality' chapter, these tools are best deployed where some human conversational cues (pointing, looking etc) are required in the context of a shared task, which is mostly visual in nature. This is a surprisingly small amount of tasks, though we have seen that the emergence of AI means that increasingly natural language AI can streamline communication, while visual generative ML can suggest design alternatives or improvements based on existing data and user preferences. This is very likely to expand the use space and this section will attempt to explain how as the case studies are explained.\par 
We will employ the acronym for collaborative virtual environment (CVE) from this stage, and it's going to come up a lot. There will be far less references in this section for brevity.
\subsection{Classic use cases}
Small teams working on product, architectural, or industrial design can benefit from CVEs that allow them to visualize, modify, and iterate on 3D models in real-time. 
\subsection{Virtual training and simulation}
CVEs can facilitate skill development and training in various industries, such as healthcare, military, aviation, and emergency response. Trainees can practice procedures in a virtual environment, with natural language AI providing instructions, explanations, or feedback, and visual generative ML potentially customizing scenarios to adapt to each user's learning curve.
\subsection{Remote teleconferencing}
In situations where face-to-face communication is not feasible, CVEs can enable remote teams to work together on shared visual tasks like planning events, brainstorming ideas, or reviewing documents. Natural language AI can transcribe and analyse spoken conversations, providing real-time translations or summaries, while visual generative ML can create visual aids or dynamically update shared documents. This may especially be useful in complex multinational legal and/or negotiation applications, though very clearly the risks of using assisting ML tooling increases. 
\subsection{Virtual art \& media collaboration}
Artists, animators, and multimedia professionals can collaborate in CVEs to create and develop their projects, such as films, animations, or video games. Natural language AI can help in storyboarding, scriptwriting, or character development, while visual generative ML can generate new visuals or adapt existing assets based on user input and style preferences.
\subsection{Data visualization and analysis}
Small teams working with large datasets can use CVEs to visually explore and analyze data in a more intuitive and engaging way. Natural language AI can help users query and interact with the data using conversational interfaces, while visual generative ML can generate new visualizations based on patterns and trends identified in the data.
\subsection{Education and virtual classrooms}
    Educators can leverage CVEs to create immersive learning experiences that engage students in collaborative activities, such as group projects, problem-solving, or scientific experiments. Natural language AI can facilitate communication, provide personalized tutoring, or assess student progress, while visual generative ML can create customized educational content based on individual needs and interests.
\subsection{Virtual labs and scientific research}
Researchers can use CVEs to conduct experiments, visualize complex data, or simulate real-world conditions in a controlled environment. Natural language AI can assist in interpreting results, automating lab protocols, or identifying research gaps, while visual generative ML can generate predictions or models based on existing data to support hypothesis testing and decision-making.


\subsection{Media and entertainment}


\subsection{Biomedical}
Collaborative Virtual Environments (CVEs) have immense potential in the fields of chemical and medical molecular modeling. By incorporating natural language AI and visual generative machine learning, these environments can revolutionize the way scientists and researchers approach complex chemical and biological problems. Here are some specific use cases:

    Drug design and discovery:
    CVEs can enable researchers to collaboratively visualize and manipulate 3D molecular structures in real-time, identifying potential drug candidates and understanding protein-ligand interactions. Natural language AI can help users interact with the molecular data, while visual generative ML can predict potential binding sites, energetics, or toxicity profiles based on existing knowledge.

    Protein structure prediction and modeling:
    Small teams can work together to predict protein structures, visualize folding patterns, and model protein-protein or protein-nucleic acid interactions. Natural language AI can assist in annotating and explaining the structural features, while visual generative ML can generate new structural hypotheses based on sequence alignments, homology modeling, and experimental data.

    Molecular dynamics simulations:
    CVEs can facilitate collaboration on complex molecular dynamics simulations, allowing researchers to analyze and visualize trajectories, energetics, and conformational changes. Natural language AI can help users navigate through simulation data and identify relevant patterns, while visual generative ML can create new conformations or predict the effects of mutations on protein stability and function.

    Cheminformatics and QSAR modeling:
    Researchers can leverage CVEs to develop and validate Quantitative Structure-Activity Relationship (QSAR) models, which predict the biological activity of chemical compounds based on their structural properties. Natural language AI can facilitate the exploration and interpretation of chemical descriptors, while visual generative ML can suggest new compounds with desired properties or optimize existing molecular scaffolds.

    Metabolic pathway modeling:
    Small teams can work together to build and analyze metabolic pathways, integrating experimental data and computational models to understand the underlying mechanisms and predict metabolic fluxes. Natural language AI can assist in annotating and explaining pathway components, while visual generative ML can generate new pathway hypotheses or predict the effects of genetic or environmental perturbations.

    Biomolecular visualization and virtual reality:
    CVEs can offer immersive, interactive experiences for exploring biomolecular structures and dynamics, enhancing researchers' understanding of complex biological systems. Natural language AI can provide contextual information or guide users through molecular landscapes, while visual generative ML can create new visualizations or adapt existing ones based on user preferences and insights.

    Collaborative molecular docking and virtual screening:
    Small teams can use CVEs to perform collaborative molecular docking and virtual screening, which involve predicting the binding of small molecules to target proteins. Natural language AI can help users refine docking parameters and analyze results, while visual generative ML can generate alternative poses or suggest new compounds for screening based on user feedback and existing data.
    Choose a suitable mixed reality platform: Select a platform that allows the creation of simple, accessible shared mixed reality environments. Consider open-source options like Mozilla Hubs or JanusVR, which offer customizable and collaborative virtual spaces.

    Integrate open-source biomed software: Incorporate open-source biomed software such as PyMOL, Chimera, or VMD for molecular visualization and analysis. These tools can be integrated into the mixed reality environment for real-time interaction, allowing students and instructors to collaboratively visualize and manipulate molecular structures.

    Leverage AI and machine learning: Integrate AI and ML algorithms like those found in DeepChem, RDKit, or Open Babel to aid in the discovery and optimization of novel compounds. These tools can help predict molecular properties, perform virtual screening, and optimize lead compounds for drug development. By incorporating AI and ML, students can learn how to apply these cutting-edge techniques to real-world problems in biomedicine.

    Establish a distributed proof system: Utilize a distributed proof system like the Nostr protocol to federate the small virtual classroom environments. This will allow for seamless collaboration among students and faculty while maintaining security and data integrity.

    Create digital objects for interaction: Use digital objects such as 3D molecular models, virtual lab equipment, and interactive simulations to create an immersive learning experience. These digital objects can be shared and manipulated in real-time, promoting collaborative learning and problem-solving.

    Implement accessible interfaces: Ensure that the virtual classroom environment is accessible to all students, including those with disabilities. Utilize AI-driven tools like StabilityAI to help with language barriers, safeguarding, and governance, enabling a more inclusive learning experience.

    Foster collaboration and communication: Encourage students and faculty to collaborate on projects, share ideas, and ask questions in real-time using voice chat, text chat, or other communication tools integrated into the mixed reality environment.

    Provide training and support: Offer training sessions and support materials to help students and faculty become familiar with the mixed reality environment, the integrated biomed software, and AI/ML tools.

    Monitor progress and adjust as needed: Regularly review student progress, gather feedback, and adjust the virtual classroom environment as needed to ensure an effective and engaging learning experience.
\subsection{Collaborative Design and Prototyping}
Utilizing open-source systems and AI-assisted tools can enable more efficient and creative collaboration in design and prototyping processes. Teams from diverse cultural backgrounds can work together seamlessly, creating a rich pool of ideas and innovations.

\subsection{Training, Simulation, and Education}
The modular open-source system can be applied to various training, simulation, and education scenarios. By integrating AI and generative ML technologies, these tools can provide personalized learning experiences and create realistic simulations that cater to different learning styles and requirements.

\subsection{Remote Collaboration and Teleconferencing}
As remote work becomes more prevalent, the Metaverse can provide a more engaging and immersive platform for collaboration and teleconferencing. The open-source system can be adapted to serve various industries, making remote collaboration more efficient and inclusive.

\subsection{Chemical and Medical Molecular Modeling}
In fields like chemical and medical molecular modeling, the integration of AI and generative ML technologies can significantly improve collaboration and innovation. Teams can work together in immersive environments to visualize complex molecular structures, benefiting from real-time AI-generated visuals and natural language processing.

\subsection{Creative Industries and Generative Art}
The combination of AI, ML, and open-source systems can revolutionize the creative industries by offering new avenues for generative art, content creation, and collaboration. Supported creativity and augmented intelligence can break down barriers and enable artists to explore new ideas and techniques, enriching the creative landscape.

\subsection{Case Study: Biodiversity Monitoring and Data Exchange with Isolated Communities}
Biodiversity monitoring in and around isolated communities is challenging due to limited access and resources. Traditional methods rely on sporadic visits by grant-funded academics, which can introduce biases and lack regular follow-up. Engaging local communities may also introduce incentive structures and biases and may not be sustainable without continuous investment.

We propose an open-source collaboration infrastructure that leverages advanced technologies such as multi-modal large language models (LLMs), satellite communication, and cryptocurrency networks to facilitate sustainable and reliable biodiversity monitoring and data exchange in isolated communities.

\subsubsection{Language Model and Voice Interface}
A specialized multi-modal LLM can be trained on local language, culture, customs, and environmental data such as flora, fauna, biotica, soil pH, and rainfall. This LLM can be accessed through a voice interface by the local community, enabling data entry and knowledge exchange in the local language. The voice interface can help overcome literacy barriers and make the system more accessible to a diverse range of community members.

\subsubsection{Data Collection and Storage}
Photographs and metadata can be logged and collected by a remote team at a later date or uploaded regularly through a satellite link (e.g., Starlink). The data storage system can be designed to be both secure and resilient, ensuring that the collected data remains available and accessible for future analysis and decision-making.

\subsubsection{Live Connection and Model Tuning}
A live connection with the academic team allows for model tuning through prompt engineering, vector database updates, and efficient Lora models, potentially offering timely advice for ecosystem interventions. Real-time communication between the community and academic teams can help identify areas of concern and rapidly adapt the LLM to address emerging challenges.

\subsubsection{Ecosystem Interventions}
The proposed infrastructure would be particularly valuable in areas facing novel disease encroachment, invasive species, active hydrology, shifting aquatic conditions, microplastic hotspots, changing microclimates, or volcanic activity. By providing real-time advice and guidance, the LLM can help communities make informed decisions about ecosystem management and conservation efforts.

\subsubsection{Incentives and Education}
Incentivizing community engagement could be achieved by providing access to the LLM for educational purposes, as demonstrated by the refugee camp e-prize (ref). Local schools and community centers can leverage the LLM as a resource for teaching environmental stewardship and ecological awareness, while also promoting digital literacy and technology skills.

\subsubsection{Monetization and Blockchain Integration}
Monetizing these systems could involve using chaumian mints such as Cashu or Fedimint, under the control of local community leaders, mediated through the global Bitcoin satellite network (Blockstream), enabling digital dollar payments to communities via low-end mobile handsets. By integrating blockchain technology, the proposed infrastructure can ensure secure, transparent, and efficient financial transactions, while also opening up new economic opportunities for isolated communities.

\subsubsection{Visual Training Support Systems}
The infrastructure could be further extended to visual training support systems using low-cost, low-power components. These systems could provide interactive, immersive learning experiences for community members, helping them better understand the local ecosystem and develop skills in environmental monitoring and management.

\subsubsection{Solar Infrastructure}
To minimize the environmental impact and ensure energy sustainability, the proposed infrastructure can be powered by solar energy. This approach will enable the system to operate independently of local power grids, reducing the overall operational costs and maintenance requirements.

\subsubsection{Open-Source Collaboration}
By linking this case study to the open-source collaboration infrastructure discussed earlier, we can create an inclusive, permissionless, federated, and economically empowered system that addresses the challenges of biodiversity monitoring while promoting digital society values such as trust, accessibility, and governance. This collaborative approach can help drive innovation and ensure that the proposed solutions are both scalable and adaptable to the unique needs of different communities and ecosystems.

\subsubsection{Risk Mitigation and Ethical Considerations}
While implementing such an infrastructure, care must be taken to address potential unintended consequences of embedding these inference systems in communities. It is essential to involve the local communities in the development and deployment process, ensuring that their perspectives, values, and traditions are respected and preserved.

Moreover, it is crucial to establish a robust ethical framework for the use of AI technologies, considering potential issues related to privacy, data security, and cultural sensitivity. Regular audits and monitoring can be implemented to ensure that the infrastructure remains transparent, accountable, and aligned with the needs and expectations of the communities it serves.

\subsubsection{Capacity Building and Local Empowerment}
An essential aspect of this initiative is building capacity and empowering local communities to take ownership of their environment and resources. By providing training, resources, and support, the proposed infrastructure can help communities develop the skills and knowledge needed to manage their ecosystems effectively.

Furthermore, the integration of digital tools and technologies can promote digital inclusion and bridge the digital divide, giving isolated communities access to valuable information and resources while fostering a sense of global connectedness and collaboration.

\subsubsection{Future Outlook and Potential Impact}
The proposed open-source collaboration infrastructure for biodiversity monitoring and data exchange has the potential to transform how isolated communities interact with their environment, enabling them to make informed decisions about conservation and ecosystem management.

By leveraging cutting-edge technologies such as LLMs, satellite communication, and blockchain networks, this approach can create a more inclusive, transparent, and accessible system for environmental monitoring and stewardship. The successful implementation of this infrastructure could pave the way for similar initiatives in other regions and ecosystems, promoting global collaboration and innovation in the pursuit of a more sustainable and equitable world.